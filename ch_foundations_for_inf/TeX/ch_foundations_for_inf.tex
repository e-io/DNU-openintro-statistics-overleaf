\begin{chapterpage}{Foundations for inference}
  \chaptertitle{Foundations for inference}
  \label{foundationsForInference}
  \label{ch_foundations_for_inf}
  \chaptersection{pointEstimates}
  \chaptersection{confidenceIntervals}
  \chaptersection{hypothesisTesting}
\end{chapterpage}
\renewcommand{\chapterfolder}{ch_foundations_for_inf}

\chapterintro{Statistical inference is primarily
  concerned with understanding and quantifying the
  uncertainty of parameter estimates.
  While the equations and details change
  depending on the setting, the foundations for inference
  are the same throughout all of statistics. \\

  \noindent%
  We start with a familiar topic:
  the idea of using a sample proportion to estimate
  a population proportion.
  Next, we create what's called a
  \emph{\hiddenterm{confidence interval}}, which is a range
  of plausible values where we may find the true population
  value.
  Finally, we introduce the
  \emph{hypothesis testing framework},
  which allows us to formally evaluate claims about the
  population, such as whether a survey provides strong
  evidence that a candidate has the support of a majority
  of the voting population.}



%__________________
\section{Point estimates and sampling variability}
\label{pointEstimates}

\index{data!solar survey|(}

Companies such as Pew Research frequently conduct
polls as a way to understand the state of public opinion
or knowledge on many topics, including politics,
scientific understanding, brand recognition, and more.
%These polls typically reach a sample of 300 to
%10,000 people.
The ultimate goal in taking a poll is generally to use
the responses to estimate the opinion or knowledge of the
broader population.

%These polls are often based on 500 to 5000 people,
%and a polling company such as Pew would use this sample
%to estimate the opinions of the broader population.
%For example, Pew frequently conducts a poll on about
%1000 adults about their feelings about the direction
%of their country.
%In early 2019, they found that 
%Through this and future sections,
%we'll use some new notation and terminology:
%\begin{itemize}
%\item
%    For all inference problems concerning proportions,
%    the population proportion will be written as $p$.
%    When discussing a population summary such as $p$,
%    it is common to refer to the value as a population
%    \term{parameter}.
%    In the solar survey,
%    $p$ represents the proportion of \emph{all}
%    American adults who support solar energy.
%\item Using Pew Research sample, we can estimate that the proportion
%    of American adults who support expanding solar energy is
%    somewhere near \pewsolarpollpercent{}.
%    This is called the \term{sample proportion},
%    and it gets a special label of $\hat{p}$
%    (spoken as \emph{p-hat}).
%\item The size of a sample will generally
%    be denoted by $n$. In the case of this Pew Research poll,
%    the \term{sample size} is $n = \pewsolarpollsize{}$.
%\end{itemize}



%In the United States, those 1000 adults would be used
%to generalize out to a population of about \emph{250 million}
%American adults.
%A~natural question arises:
%\begin{quote}
%\em
%If the poll was based on only a thousand people,
%how reliable is it?
%\end{quote}
%For instance, if we took another poll,
%we wouldn't get the exact same answer,
%so how trustworthy is the result?
%This is the topic of this first inference section,
%where we hope to understand how variable estimates
%are from one sample to the next,
%which will give us an idea of how much trust we should
%(or shouldn't) put into such polls.


\subsection{Point estimates and error}

\index{point estimate|(}

Suppose a poll suggested the US President's approval
rating is 45\%.
We would consider 45\% to be a
\term{point estimate}\index{estimate} of the approval
rating we might see if we collected responses from the
entire population.
%\footnote{When we collect responses from the
%  entire population, it is called a \term{census}.
%  It is often expensive to conduct a census,
%  which is why we often instead take a sample.}
This entire-population response proportion is
generally referred to as the \term{parameter}
of interest.
When the parameter is a proportion,
it is often denoted by $p$,
%We typically estimate the parameter by collecting
%information from a sample of the population;
%we compute the observed proportion in the sample;
%also called a \term{point estimate},
and we often refer to the sample proportion as $\hat{p}$
(pronounced \emph{p-hat}\footnote{Not to be confused with
  \emph{phat}, the slang term used for something cool,
  like this book.}).
Unless we collect responses from every individual in the population,
$p$ remains unknown, and we use $\hat{p}$ as our estimate of~$p$.
The difference we observe from the poll versus
the parameter is called the \term{error} in the estimate.
%There are other considerations that can influence
%the error in a sample's estimate can be influenced
%by other factors, too.
%it is not the complete story.
%For this reason, we will also find it convenient to track
%the \term{sample size}, which is generally referred to using
%the letter $n$.
Generally, the error consists of two aspects:
sampling error and bias.
%Throughout the rest of this section,
%we discuss what a point estimate like
%\pewsolarpollpercent{} represents
%and the sampling uncertainty associated with such an estimate.
%If we take a simple random sample of 1000 American adults
%and ask them for their opinion about solar energy,
%will we tend to get a result close to the
%\pewsolarpollpercent{} value,
%or might we see observations far from the truth?


%
%Suppose that we know that \pewsolarpollpercent{}
%of American adults 
%
%American adults' attitudes towards different forms of energy.
%They found that \pewsolarpollpercent{} of respondents
%favored expanding
%solar energy.
%In this case, Pew Research worked to ensure
%that the sample was representative.
%However, a~natural question remains:
%\begin{quote}
%\em
%If the poll was based on only a thousand people,
%how reliable is it?
%\end{quote}
%If we took another poll, we wouldn't get the exact same answer.
%Maybe we'd get 90\%, or perhaps even 80\%.
%Ultimately, it's unlikely that the actual proportion of
%Americans who support expanding solar energy is
%\emph{exactly}~\pewsolarpollpercent{}, but the data suggest
%the actual support is close to \pewsolarpollpercent{}.
%This type of uncertainty --
%the variability in the estimate from one sample to the next --
%is called the \term{sampling error},
%and it is a major focus throughout the rest of this book.

%\footnote{Another major form
%  of error is \term{bias}, which basically is a systematic
%  tendency to over or under-estimate the true population value.
%  For instance, if we took a political poll and undersampled
%  one of the political parties, the sample would not be
%  representative and would skew in a particular direction.}
%Ultimately, it's unlikely that the actual proportion of Americans
%who support expanding solar energy is \emph{exactly}
%\pewsolarpollpercent{}, but the data suggest the actual
%support is close to \pewsolarpollpercent{}.


%The Pew Research poll is a point estimate
%of the actual proportion
%of American adults who support expanding solar energy.
%This estimate of \pewsolarpollpercent{} is unlikely
%to be perfect,
%and it's quite possible for the population proportion
%to be a little lower or a little higher than the
%sample proportion.
%The difference between a point estimate and
%the parameter is called the estimate's \term{error}.

\termsub{Sampling error}{sampling error},
sometimes called \emph{\hiddenterm{sampling uncertainty}},
describes how much an estimate will tend to vary from
one sample to the next.
For instance, the estimate from one sample might be 1\% too low
while in another it may be 3\% too high.
Much of statistics, including much of this book,
is focused on understanding and quantifying sampling error,
and we will find it useful to consider a sample's size
to help us quantify this error;
the \term{sample size} is often represented by the letter $n$.
%Intuitively, a larger sample would tend to produce a more
%accurate estimate than what we would
%obtain from a smaller sample.
%This is exactly the ref
%estimate from a smaller sample,
%and this is generally true.

\termsub{Bias}{bias} describes a systematic tendency
to over- or under-estimate the true population value.
For~example, if we were taking a student poll asking
about support for a new college stadium, we'd probably
get a biased estimate of the stadium's level of student
support by wording the question as,
\emph{Do you support your school by supporting funding
  for the new stadium?}
We try to minimize bias through thoughtful data
collection procedures, which were discussed in
Chapter~\ref{ch_intro_to_data}
and are the topic of many other books.

%While bias is an incredibly important topic,
%it's forms are so varied that 
%so vast and context-specific that we 

%\begin{onebox}{Sampling error vs bias}
%  \termsub{Sampling error}{sampling error} is uncertainty
%  in a point estimate that happens naturally from one sample
%  to the next.
%  The methods we discuss are useful for understanding,
%  quantifying, and working with sampling errors.
%  \stdvspace{}
%
%  In contrast, another common form of error is \term{bias},
%  which is a systematic tendency to over or under-estimate
%  the true population value.
%  For instance, if we took a political poll but our sample
%  didn't include a roughly representative distribution of
%  the political parties, the sample would likely skew
%  in a particular direction and be biased.
%\end{onebox}




\subsection{Understanding the variability of a point estimate}
\label{simulationForUnderstandingVariabilitySection}

\newcommand{\pewsolarpollsize}{1000}
\newcommand{\pewsolarparprop}{0.88}
\newcommand{\pewsolarparpropcomplement}{0.12}
\newcommand{\pewsolarparpercent}{88\%}
\newcommand{\pewsolarparpercentcomplement}{12\%}
\newcommand{\pewsolarpollprop}{0.887}
\newcommand{\pewsolarpollpropcomplement}{0.113}
\newcommand{\pewsolarpollpercent}{88.7\%}
\newcommand{\pewsolarpollpercentcomplement}{11.3\%}
\newcommand{\pewsolarpollcount}{887}
\newcommand{\pewsolarpollexpcount}{880}
\newcommand{\pewsolarpollcountcomplement}{113}
\newcommand{\pewsolarpollexpcountcomplement}{120}
\newcommand{\pewsolarpollse}{0.010}

Suppose the proportion of American adults who support
the expansion of solar energy is $p = \pewsolarparprop{}$,
which is our parameter of interest.\footnote{We haven't
  actually conducted a census to measure this value perfectly.
  However, a very large sample has suggested the actual
  level of support is about \pewsolarparpercent{}.}
If we were to take a poll of \pewsolarpollsize{} American adults
on this topic, the estimate would not be perfect,
but how close might we expect the sample proportion
in the poll would be to \pewsolarparpercent{}?
We want to understand, \emph{how does the
sample proportion $\hat{p}$ behave when the true population
proportion is
\pewsolarparprop{}}.\footnote{\pewsolarparpercent{}
  written as a proportion would be
  \pewsolarparprop{}.
  It is common to switch between proportion and percent.
  However, formulas presented in this book always refer
  to the proportion, not the percent.}
Let's find out!
We can simulate responses we would get from a simple
random sample of 1000 American adults,
which is only possible because we know the actual
support for expanding solar energy is \pewsolarparprop{}.
%
%
%We could
%run the survey again to see how consistent the results
%are, but who has the time and money for that? Instead,
%we can investigate the properties of $\hat{p}$ using simulations.
%
%To simulate the sample, we'll suppose that the population
%proportion is exactly \pewsolarpollpercent{}.
%Now, we know
%the population proportion isn't exactly \pewsolarpollpercent\%,
%but we do expect it to be close, so this simulation will offer
%us some insights about the property of $\hat{p}$.
%If we took a random sample
%from this population, how accurate would the point estimate be?
Here's how we might go about constructing such a simulation:
%simulate it:
\begin{enumerate}
\item There were about 250 million American adults in 2018.
    On 250 million pieces of paper, write ``support''
    on \pewsolarparpercent{} of them and ``not'' on
    the other \pewsolarparpercentcomplement{}.
\item Mix up the pieces of paper and pull out \pewsolarpollsize{}
    pieces to represent our sample of \pewsolarpollsize{}
    American adults.
\item Compute the fraction of the sample that say ``support''.
\end{enumerate}
Any volunteers to conduct this simulation? Probably not. Running
this simulation with 250 million pieces of paper would be
time-consuming and very costly, but we can simulate it
using computer code; we've written a short program in
Figure~\ref{solarPollSimulationCodeR}
in case you are curious what the computer code looks like.
In this simulation, the sample gave a point estimate of
$\hat{p}_1 = 0.894$. We~know the population proportion
for the simulation was $p = \pewsolarparprop{}$, so we know
the estimate had an error of
$0.894 - \pewsolarparprop{} = \text{+0.014}$.

%\setlength\textwidth{\officialtextwidth-10mm}
\begin{figure}[h]
\texttt{\# 1.\ Create a set of 250 million entries,
where \pewsolarparpercent{} of them are "support" \\
\#\ \ \ \ and \pewsolarparpercentcomplement{} are "not". \\
pop\us{}size <- 250000000 \\
possible\_entries <- c(rep("support", \pewsolarparprop{} * pop\us{}size), rep("not", \pewsolarparpropcomplement{} * pop\us{}size))
\\[3mm]
\# 2.\ Sample \pewsolarpollsize{} entries without replacement. \\
sampled\_entries <- sample(possible\_entries, size = \pewsolarpollsize{}) \\[3mm]
\# 3.\ Compute p-hat:~count the number that are "support",
then divide by \\
\#\ \ \ \ the sample size. \\
sum(sampled\_entries == "support") / \pewsolarpollsize{}}
\caption{For those curious, this is code for
    a single $\hat{p}$ simulation using the
    statistical software called \R{}\index{R}.
    Each line that starts with \texttt{\#} is a
    \term{code comment},
    which is used to describe in regular language what the
    code is doing.
    We've provided software labs in \R{} at
    \oiRedirect{os}{openintro.org/book/os}
    for anyone interested in learning more.}
\label{solarPollSimulationCodeR}
\end{figure}
% \setlength\textwidth{\officialtextwidth}

One simulation isn't enough to get a great sense of the
distribution of estimates we might expect in the simulation,
so we should run more simulations.
In a second simulation,
we get $\hat{p}_2 = 0.885$, which has an error of~+0.005.
In another, $\hat{p}_3 = 0.878$ for an error of -0.002.
And in another,
an estimate of $\hat{p}_4 = 0.859$ with an error of -0.021.
With the help of a computer, we've run the simulation 10,000 times
and created a histogram of the results from all 10,000 simulations
in Figure~\ref{sampling_10k_prop_88p}. This
distribution of sample proportions is called a
\term{sampling distribution}.
We can characterize this sampling distribution as follows:
\begin{description}
\setlength{\itemsep}{0mm}
\item[Center.]
    The center of the distribution is
    $\bar{x}_{\hat{p}} = \pewsolarparprop{}0$,
    which is the same as the parameter.
    Notice that the simulation mimicked a simple random sample
    of the population, which is a straightforward sampling
    strategy that helps avoid sampling bias.
%    That~is, we see that the sample proportion is an
%    \termsub{unbiased estimate}{unbiased}
%    of the population proportion.
\item[Spread.]
    The standard deviation of the distribution
    is $s_{\hat{p}} = \pewsolarpollse{}$.
    When we're talking about
    a sampling distribution or the variability of
    a point estimate, we typically use the term
    \termsub{standard error}{standard error (SE)}
    rather than \emph{standard deviation},
    and the notation $SE_{\hat{p}}$ is used for the standard
    error associated with the sample proportion.
\item[Shape.]
    The distribution is symmetric and bell-shaped,
    and it \emph{resembles a normal distribution}.
\end{description}
These findings are encouraging!
When the population
proportion is $p = \pewsolarparprop{}$ and the sample size is
$n = \pewsolarpollsize{}$,
the sample proportion $\hat{p}$ tends to give
a pretty good estimate
of the population proportion.
We also have the interesting observation
that the histogram resembles a normal distribution.

\begin{figure}[h]
   \centering
   \Figure[A histogram is shown for 10,000 sample proportions where each sample is taken from a population where the population proportion is \pewsolarparprop{} and the sample size is $n = \pewsolarpollsize{}$. The distribution is bell-shaped (appears nearly normal), is centered at 0.88 and has a standard deviation of about 0.01.]{0.8}{sampling_10k_prop_88p}
   %\Figure{0.8}{sampling_10k_prop_887p}
   \caption{A histogram of 10,000 sample proportions,
       where each sample is taken from a population
       where the population proportion is
       \pewsolarparprop{} and the sample size
       is $n = \pewsolarpollsize{}$.}
   \label{sampling_10k_prop_88p}
   %\label{sampling_10k_prop_887p}
\end{figure}

\begin{onebox}{Sampling distributions are
    never observed, but we keep them in mind}
  In real-world applications, we never actually observe the
  sampling distribution, yet it is useful to always think of
  a point estimate as coming from such a hypothetical
  distribution.
  \mbox{Understanding} the sampling distribution will help us
  characterize and make sense of the point estimates that we
  do observe.
\end{onebox}

\begin{examplewrap}
\begin{nexample}{If we used a much smaller sample size of $n = 50$,
would you guess that the standard error for $\hat{p}$ would be larger
or smaller than when we used $n = \pewsolarpollsize{}$?}
\label{smallerSampleWhatHappensToPropErrorExercise}
Intuitively, it seems like more data is better
than less data, and generally that is correct! The typical error
when $p = \pewsolarparprop{}$ and $n = 50$ would be larger
than the error we would expect when $n = \pewsolarpollsize{}$.
\end{nexample}
\end{examplewrap}

%\noindent
Example~\ref{smallerSampleWhatHappensToPropErrorExercise}
highlights an important property we will see again and again:
a bigger sample tends to provide a more precise point estimate
than a smaller sample.

\index{point estimate|)}


\subsection{Central Limit Theorem}

The distribution in
Figure~\ref{sampling_10k_prop_88p} looks an awful lot like
a normal distribution. That is no anomaly; it~is the result
of a general principle called the
\index{Central Limit Theorem!proportion|textbf}
\term{Central Limit Theorem}.

\begin{onebox}{Central Limit Theorem and the success-failure condition}
  When observations are independent and the sample size is
  sufficiently large, the sample proportion $\hat{p}$ will tend
  to follow a normal distribution with the following mean and
  standard error:%\footnotemark{}
  \begin{align*}
    \mu_{\hat{p}} &= p
    &SE_{\hat{p}} &= \sqrt{\frac{p (1 - p)}{n}}
  \end{align*}
  In order for the Central Limit Theorem to hold,
  the sample size is typically considered sufficiently large
  when $np \geq 10$ and $n(1-p) \geq 10$, which is called the
  \term{success-failure condition}.
\end{onebox}
%\footnotetext{Some statisticians will say what we
%  have written for $SE_{\hat{p}}$ should be called
%  the \emph{standard deviation of $\hat{p}$}
%  and the standard error is a term for
%  an estimated version (that we'll first encounter
%  in Section~\ref{apply_clt_real_world_setting}).
%  We adhere to simpler terminology in this book
%  that is also accepted,
%  where the listed formula also can be called the
%  \emph{standard error}.}

The Central Limit Theorem is incredibly important, and it provides
a foundation for much of statistics.
As we begin applying
the Central Limit Theorem, be mindful of the two
technical conditions:
the observations must be independent, and the sample size must
be sufficiently large such that $np \geq 10$ and $n(1-p) \geq 10$.

\begin{examplewrap}
\begin{nexample}{Earlier we estimated the mean and standard
error of $\hat{p}$ using simulated data when
$p = \pewsolarparprop{}$ and $n = \pewsolarpollsize{}$.
Confirm that the Central Limit Theorem applies
and the sampling  distribution is approximately
normal.}\label{sample_p88_n1000_confirm_normal}
\begin{description}
\item[Independence.] There are $n = \pewsolarpollsize{}$
    observations for each
    sample proportion $\hat{p}$, and each of those observations
    are independent draws. \emph{The most common way for
    observations to be considered independent is if they are from
    a simple random sample.}
    \index{independent}
    \index{independence}
    \index{Central Limit Theorem!independence}
\item[Success-failure condition.] We can confirm the sample size
    is sufficiently large by checking the success-failure condition
    and confirming the two calculated values are greater than~10:
    \begin{align*}
    np &= \pewsolarpollsize{} \times \pewsolarparprop{}
        = \pewsolarpollexpcount{}
        \geq 10
    &n(1-p) &= \pewsolarpollsize{} \times (1 - \pewsolarparprop{})
        = \pewsolarpollexpcountcomplement{}
        \geq 10
    \end{align*}
\end{description}
The independence and success-failure conditions are both
satisfied, so the Central Limit Theorem applies, and it's
reasonable to model $\hat{p}$ using a normal distribution.
\end{nexample}
\end{examplewrap}

\begin{onebox}{How to verify sample observations are independent}
  Subjects in an experiment are considered independent
  if they undergo random assignment to the treatment
  groups.\stdvspace{}

  If the observations are from a simple random sample,
  then they are independent.\stdvspace{}

  If a sample is from a seemingly random process,
  e.g. an occasional error on an assembly line,
  checking independence is more difficult. In~this case,
  use your best judgement.
\end{onebox}

An additional condition that is sometimes added for samples
from a population is that they are no larger than 10\% of
the population.
When the sample exceeds 10\% of the population size,
the methods we discuss tend to overestimate the sampling error
slightly versus what we would get using more advanced
methods.\footnote{For example, we could use what's called the
  \term{finite population correction factor}:
  if the sample is of size $n$ and the population size is $N$,
  then we can multiply the typical standard error formula by
  $\sqrt{\frac{N-n}{N-1}}$
  to obtain a smaller, more precise estimate of the
  actual standard error.
  When $n < 0.1 \times N$, this correction factor is
  relatively small.}
This is very rarely an issue, and when it is an issue,
our methods tend to be conservative, so we consider this
additional check as optional.

\begin{examplewrap}
\begin{nexample}{Compute the theoretical mean and standard error
of $\hat{p}$ when
$p = \pewsolarparprop{}$ and $n = \pewsolarpollsize{}$,
according to the
Central Limit Theorem.}\label{sample_p88_n1000_mean_se}
The mean of the $\hat{p}$'s is simply the population proportion:
$\mu_{\hat{p}} = \pewsolarparprop{}$.

The calculation of the standard error of $\hat{p}$ uses
the following formula:
\begin{align*}
SE_{\hat{p}}
    = \sqrt{\frac{p (1 - p)}{n}}
    = \sqrt{\frac{\pewsolarparprop{} (1 - \pewsolarparprop{})}
        {\pewsolarpollsize{}}}
    = \pewsolarpollse{}
\end{align*}
\end{nexample}
\end{examplewrap}

\begin{examplewrap}
\begin{nexample}{Estimate how frequently the sample proportion
$\hat{p}$ should be within 0.02 (2\%) of the population value,
$p = \pewsolarparprop{}$. Based on
Examples~\ref{sample_p88_n1000_confirm_normal}
and~\ref{sample_p88_n1000_mean_se},
we know that the distribution is approximately
$N(\mu_{\hat{p}} = \pewsolarparprop{}, SE_{\hat{p}} = \pewsolarpollse{})$.}
\label{sampling_10k_prop_887p-prop_from_867_to_907}
After so much practice in Section~\ref{normalDist},
this normal distribution example will hopefully feel familiar!
We would like to understand the fraction of $\hat{p}$'s
between 0.86 and 0.90:
\begin{center}
\Figure[A normal distribution centered at 0.88 with a standard deviation of 0.01 is shown, where the region between 0.86 and 0.90 has been shaded.]{0.35}{p-hat_from_86_and_90}
\end{center}
With $\mu_{\hat{p}} = \pewsolarparprop{}$ and
$SE_{\hat{p}} = \pewsolarpollse{}$,
we can compute the Z-score for both the left and right cutoffs:
\begin{align*}
Z_{0.86}
  &= \frac{0.86 - \pewsolarparprop{}}{\pewsolarpollse{}}
  = -2
&Z_{0.90}
  &= \frac{0.90 - \pewsolarparprop{}}{\pewsolarpollse{}}
  = 2
\end{align*}
We can use either statistical software, a graphing calculator,
or a table to find the areas to the tails, and in any case we
will find that they are each 0.0228. The total tail areas are
$2 \times 0.0228 = 0.0456$, which leaves the shaded area of
0.9544. That is, about 95.44\% of the sampling distribution
in Figure~\ref{sampling_10k_prop_88p} is within $\pm0.02$
of the population proportion, $p = \pewsolarparprop{}$.
\end{nexample}
\end{examplewrap}

\D{\newpage}

\begin{exercisewrap}
\begin{nexercise}
In Example~\ref{smallerSampleWhatHappensToPropErrorExercise}
we discussed how a smaller sample would tend
to produce a less reliable estimate. Explain how this intuition
is reflected in the formula for
$SE_{\hat{p}} = \sqrt{\frac{p (1 - p)}{n}}$.\footnotemark
\end{nexercise}
\end{exercisewrap}
\footnotetext{Since the
  sample size $n$ is in the denominator
  (on the bottom) of the fraction,
  a bigger sample size means the entire
  expression when calculated will tend to be smaller.
  That is, a larger sample size would correspond to
  a smaller standard error.}


\subsection{Applying the Central Limit Theorem to
    a real-world setting}
\label{apply_clt_real_world_setting}

We do not actually know the population proportion
unless we conduct an expensive poll of all individuals
in the population.
Our earlier value of $p = 0.88$ was based on
a Pew Research conducted a poll of \pewsolarpollsize{}
American adults that found
$\hat{p} = \pewsolarpollprop{}$ of them favored
expanding solar energy.
The researchers might have wondered:
does the sample proportion from the poll approximately
follow a normal distribution?
We can check the conditions from the Central Limit Theorem:
\begin{description}
\item[Independence.] The poll is a simple random sample of
    American adults, which means that the observations are
    independent.
\item[Success-failure condition.] To check this condition,
    we need the population proportion, $p$, to check if both
    $np$ and $n(1-p)$ are greater than 10.
    However, we do not actually know $p$, which
    is exactly why the pollsters would take a sample!
    In cases like these, we often use $\hat{p}$
    as our next best way to check the success-failure condition:
    \begin{align*}
    n\hat{p}
        &= \pewsolarpollsize{} \times \pewsolarpollprop{}
        = \pewsolarpollcount{}
    &n (1 - \hat{p})
        &= \pewsolarpollsize{} \times (1 - \pewsolarpollprop{})
        = \pewsolarpollcountcomplement{}
    \end{align*}
    The sample proportion $\hat{p}$ acts as
    a reasonable substitute for $p$ during this check,
    and each value in this case is well above the minimum of 10.
\end{description}

This \term{substitution approximation} of using $\hat{p}$ in
place of $p$ is also useful when computing the standard error
of the sample proportion:
\begin{align*}
SE_{\hat{p}}
    = \sqrt{\frac{p (1 - p)}{n}}
    \approx \sqrt{\frac{\hat{p} (1 - \hat{p})}{n}}
    = \sqrt{\frac{\pewsolarpollprop{}
        (1 - \pewsolarpollprop{})}{\pewsolarpollsize{}}}
    = \pewsolarpollse{}
\end{align*}
This substitution technique is sometimes
referred to as the ``\hiddenterm{plug-in principle}''.
In this case, $SE_{\hat{p}}$ didn't change enough to
be detected using only 3 decimal places
versus when we completed the calculation with
\pewsolarparprop{} earlier.
The computed standard error tends to be reasonably stable
even when observing slightly different proportions in one
sample or another.


\D{\newpage}

\subsection{More details regarding the Central Limit Theorem}

\noindent%
We've applied the Central Limit Theorem in numerous examples
so far this chapter:
\begin{quote}{\em
When observations are independent and the sample size is
sufficiently large, the distribution of $\hat{p}$ resembles
a normal distribution with
\begin{align*}
  \mu_{\hat{p}} &= p
  &SE_{\hat{p}} &= \sqrt{\frac{p (1 - p)}{n}}
\end{align*}
The sample size is considered sufficiently large
when $n p \geq 10$ and $n (1 - p) \geq 10$.
}\end{quote}
In this section, we'll explore the success-failure
condition and seek to better understand the
Central Limit Theorem.

An interesting question to answer is, \emph{what happens when
$np < 10$ or $n(1-p) < 10$?} As we did in
Section~\ref{simulationForUnderstandingVariabilitySection},
we can simulate drawing samples of different sizes where,
say, the true proportion is $p = 0.25$.
Here's a sample of size~10:
\begin{center}
% paste(sample(c("yes", "no"), 10, TRUE, c(.25, .75)), collapse = ", ")
no, no, yes, yes, no, no, no, no, no, no
\end{center}
In this sample, we observe a sample proportion of yeses
of $\hat{p} = \frac{2}{10} = 0.2$. We can simulate many such
proportions to understand the sampling distribution of
$\hat{p}$ when $n = 10$ and $p = 0.25$, which we've plotted
in Figure~\ref{sampling_10_prop_25p}
alongside a normal distribution with the
same mean and variability.
These distributions have a number of important differences.

\begin{figure}[h]
   \centering
   \Figure[There are two plots. The first plot is a histogram of 10,000 simulations of p-hat when the sample size is n equals 10 and the population proportion is p equals 0.25. The possible values are 0.0, 0.1, 0.2, and so on up to 1.0, though the graph only shows values up to 0.8. The distribution is centered at about 0.25, and is slightly right-skewed. The frequencies are about 500 for 0.0, 1900 for 0.1, 2800 for 0.2, 2400 for 0.3, 1500 for 0.4, 500 for 0.5, 100 for 0.6, and the bin heights for the remaining values have bin heights that are not visually distinguishable from zero. The second plot shows a normal distribution centered at 0.25 with a standard deviation of 0.137. The plot has a vertical line located at 0.0, which makes it more visually evident that a portion of the area under the normal distribution -- about 5\% of this area -- represents values below 0.0.]{0.97}{sampling_10_prop_25p}
   \caption{Left: simulations of $\hat{p}$ when the sample size
       is $n = 10$ and the population proportion is $p = 0.25$.
       Right: a normal distribution with the same mean (0.25)
       and standard deviation (0.137).}
   \label{sampling_10_prop_25p}
\end{figure}

\begin{figure}
  \centering
  \Figures[Sampling distributions are shown for several scenarios for parameters p and n. The graphs are arranged in a grid of 5 rows representing proportions 0.1, 0.2, 0.5, 0.8, and 0.9 and 2 columns of sample sizes n equals 10 and 25. In each graph, the distribution is centered at the proportion. Given that these are proportions based on relatively small sample sizes, the bins do look relatively discrete (jumpy from one to the next), though less so for the distributions based on n equals 25. In cases where the true underlying proportion is near the lower bound of 0 or the upper bound of 1, the distribution tends to skew away from that boundary. This is most noticeable for both the distributions representing proportions closer to either boundary and for the smaller sample size. One distribution stands out among the 10 shown: the sample with p equals 0.5 and n equals 25, which shows a bell-shaped distribution resembling the normal distribution, though the data are still somewhat discrete.]{}{clt_prop_grid}{clt_prop_grid_1}
  \caption{Sampling distributions for several scenarios
      of $p$ and $n$. \\
      Rows: $p = 0.10$, $p = 0.20$, $p = 0.50$,
      $p = 0.80$, and $p = 0.90$. \\
      Columns: $n = 10$ and $n = 25$.}
  \label{clt_prop_grid_1}
\end{figure}

\begin{figure}
  \centering
  \Figures[Sampling distributions are shown for several scenarios for parameters p and n. The graphs are arranged in a grid of 5 rows representing proportions 0.1, 0.2, 0.5, 0.8, and 0.9 and 3 columns of sample sizes n equals 50, 100, and 250. Relative to the previous figure, which considered similar proportion scenarios but with n equals 10 and 25, the data in these graphs looks less discrete -- that is, they appear to almost be continuous. This is most evident for the largest sample sizes. Nearly all of the graphs shown also closely resemble the normal distribution, in some cases with the larger sample sizes that it resembles it so closely that there are not substantial visual differences. One aspect less evident -- but still present -- in the last figure but that continues into and becomes much more obvious in this figure, is that the distributions of the sample proportions tend to have a much smaller standard deviation with the larger sample sizes. That is, the sample proportion distributions for larger sample sizes tend to be smaller than they were for smaller sample sizes. Also, the variability within a graph also appears to be largest for the proportion p equals 0.5 than it is for the other proportions when considering a single proportion -- and this property is apparent upon inspection of a distribution based on any of the considered sample sizes.]{}{clt_prop_grid}{clt_prop_grid_2}
  \caption{Sampling distributions for several scenarios
      of $p$ and $n$. \\
      Rows: $p = 0.10$, $p = 0.20$, $p = 0.50$,
      $p = 0.80$, and $p = 0.90$. \\
      Columns: $n = 50$, $n = 100$, and $n = 250$.}
  \label{clt_prop_grid_2}
\end{figure}

\begin{center}
\begin{tabular}{lccc}
\hline
    &  Unimodal?  &  Smooth?  &  Symmetric? \\
\hline
Normal: $N(0.25, 0.14)$  &
    \highlightO{Yes}  &
    \highlightO{Yes}  &
    \highlightO{Yes} \\
$n = 10$, $p = 0.25$  &
    \highlightO{Yes}  &
    \highlightT{No}  &
    \highlightT{No} \\
\hline
\end{tabular}
\end{center}
Notice that the success-failure condition
was not satisfied when $n = 10$ and $p = 0.25$:
\begin{align*}
n p = 10 \times 0.25 = 2.5 &&
    n (1 - p) = 10 \times 0.75 = 7.5
\end{align*}
This single sampling distribution does not show that
the success-failure condition is the perfect guideline,
but we have found that the guideline did correctly
identify that a normal distribution might not be appropriate.

We can complete several additional simulations,
shown in
Figures~\ref{clt_prop_grid_1}
and~\ref{clt_prop_grid_2},
and we can see some trends:
\begin{enumerate}
\item When either $np$ or $n(1 - p)$ is small, the
    distribution is more \term{discrete},
    i.e. \emph{not continuous}.
\item When $np$ or $n(1-p)$ is smaller than~10,
    the skew in the distribution is more noteworthy.
\item The larger both $np$ \emph{and} $n(1 - p)$,
    the more normal the distribution.
    This may be a little harder to see for the larger
    sample size in these plots as the variability
    also becomes much smaller.
\item When $np$ and $n(1 - p)$ are both very large,
    the distribution's discreteness is hardly evident,
    and the distribution looks much more
    like a normal distribution.
\end{enumerate}

\D{\newpage}

So far we've only focused on the skew and discreteness
of the distributions.
We haven't considered how the mean and standard error
of the distributions change.
Take a moment to look back at the graphs,
and pay attention to three things:
\begin{enumerate}
\item The centers of the distribution are always at
    the population proportion, $p$, that was used to
    generate the simulation. Because the sampling
    distribution of $\hat{p}$ is always centered at
    the population parameter $p$, it means the sample
    proportion $\hat{p}$ is \term{unbiased} when
    the data are independent and drawn from such
    a population.
\item For a particular population proportion $p$,
    the variability in the sampling distribution
    decreases as the sample size~$n$ becomes larger.
    This will likely align with your intuition:
    an estimate based on a larger sample size will
    tend to be more accurate.
\item For a particular sample size, the variability
    will be largest when $p = 0.5$. The differences
    may be a little subtle, so take a close look.
    This reflects the role of the proportion
    $p$ in the standard error formula:
    $SE = \sqrt{\frac{p (1 - p)}{n}}$.
    The standard error is largest when $p = 0.5$.
\end{enumerate}

At no point will the distribution of $\hat{p}$ look
\emph{perfectly} normal, since $\hat{p}$ will always
take discrete values ($x / n$).
It is always a matter of degree, and we will use
the standard success-failure condition with minimums
of 10 for $np$ and $n (1 - p)$ as our guideline
within this~book.


\subsection{Extending the framework for other statistics}

The strategy of using a sample statistic to estimate
a parameter is quite common, and it's a strategy that
we can apply to other statistics besides a proportion.
For instance, if we want to estimate the average salary
for graduates from a particular college, we could
survey a random sample of recent graduates;
in that example, we'd be using a sample mean $\bar{x}$
to estimate the population mean~$\mu$ for all graduates.
As another example, if we want to estimate the
difference in product prices for two websites,
we might take a random sample of products available
on both sites, check the prices on each,
and then compute the average difference;
this strategy certainly would give us some idea
of the actual difference through a point estimate.

While this chapter emphasizes a single proportion
context, we'll encounter many different contexts
throughout this book where these methods will be
applied.
The principles and general ideas are the same,
even if the details change a little.
We've also sprinkled some other contexts into
the exercises to help you start thinking about
how the ideas generalize.


{\exercisesheader{}

% 1

\eoce{\qt{Identify the parameter, Part I\label{identify_parameter_1}} For each of the following situations, state 
whether the parameter of interest is a mean or a proportion. It may be helpful 
to examine whether individual responses are numerical or categorical.
\begin{parts}
\item In a survey, one hundred college students are asked how many hours per 
week they spend on the Internet.
\item In a survey, one hundred college students are asked: ``What percentage of 
the time you spend on the Internet is part of your course work?"
\item In a survey, one hundred college students are asked whether or not they 
cited information from Wikipedia in their papers.
\item In a survey, one hundred college students are asked what percentage of 
their total weekly spending is on alcoholic beverages.
\item In a sample of one hundred recent college graduates, it is found that 85 
percent expect to get a job within one year of their graduation date.
\end{parts}
}{}

% 2

\eoce{\qt{Identify the parameter, Part II\label{identify_parameter_2}} For each of the 
following situations, state whether the parameter of interest is a mean or a 
proportion. 
\begin{parts}
\item A poll shows that 64\% of Americans personally worry a great deal about 
federal spending and the budget deficit.
\item A survey reports that local TV news has shown a 17\% increase in revenue 
within a two year period while newspaper revenues decreased by 6.4\% during this 
time period.
\item In a survey, high school and college students are asked whether or not 
they use geolocation services on their smart phones.
\item In a survey, smart phone users are asked whether or not they use a web-based taxi service.
\item In a survey, smart phone users are asked how many times they used a web-based taxi service over the last year.
\end{parts}
}{}

% 3

\eoce{\qt{Quality control\label{comp_chips_quality_ctrl_prop}}
As part of a quality control process for computer chips,
an engineer at a factory randomly samples 212 chips
during a week of production to test the current rate of
chips with severe defects.
She finds that 27 of the chips are defective.
\begin{parts}
\item
    What population is under consideration in the data set?
\item
    What parameter is being estimated?
\item\label{comp_chips_quality_ctrl_prop_pt_est}%
    What is the point estimate for the parameter?
\item\label{comp_chips_quality_ctrl_prop_se_name}%
    What is the name of the statistic we use to measure
    the uncertainty of the point estimate?
\item\label{comp_chips_quality_ctrl_prop_se_calc_w_pt_est}%
    Compute the value from
    part~(\ref{comp_chips_quality_ctrl_prop_se_name})
    for this context.
\item
    The historical rate of defects is 10\%.
    Should the engineer be surprised by the observed
    rate of defects during the current week?
\item
    Suppose the true population value was found to be 10\%.
    If we use this proportion to recompute the value in
    part~(\ref{comp_chips_quality_ctrl_prop_se_calc_w_pt_est})
    using $p = 0.1$ instead of $\hat{p}$,
    does the resulting value change much?
\end{parts}
}{}

% 4

\eoce{\qt{Unexpected expense\label{us_emergency_expense_prop}}
In a random sample 765 adults in the United States, 322 say
they could not cover a \$400 unexpected expense without borrowing
money or going into debt.
% Ref: https://www.federalreserve.gov/publications/files/2017-report-economic-well-being-us-households-201805.pdf
\begin{parts}
\item
    What population is under consideration in the data set?
\item
    What parameter is being estimated?
\item\label{us_emergency_expense_prop_pt_est}%
    What is the point estimate for the parameter?
\item\label{us_emergency_expense_prop_se_name}%
    What is the name of the statistic we use to measure
    the uncertainty of the point estimate?
\item\label{us_emergency_expense_prop_se_calc_w_pt_est}%
    Compute the value from
    part~(\ref{us_emergency_expense_prop_se_name})
    for this context.
\item
    A cable news pundit thinks the value is actually 50\%.
    Should she be surprised by the data?
\item
    Suppose the true population value was found to be 40\%.
    If we use this proportion to recompute the value in
    part~(\ref{us_emergency_expense_prop_se_calc_w_pt_est})
    using $p = 0.4$ instead of $\hat{p}$,
    does the resulting value change much?
\end{parts}
}{}

\D{\newpage}

% 5

\eoce{\qt{Repeated water samples\label{repeated_water_samples_prop}}
A nonprofit wants to understand the fraction of households that
have elevated levels of lead in their drinking water.
They expect at least 5\% of homes will have elevated levels of
lead, but not more than about 30\%.
They randomly sample 800 homes and work with the owners to retrieve
water samples, and they compute the fraction of these homes
with elevated lead levels.
They repeat this 1,000 times and build a distribution
of sample proportions.
\begin{parts}
\item
    What is this distribution called?
\item
    Would you expect the shape of this distribution to be
    symmetric, right skewed, or left skewed?
    Explain your reasoning.
\item
    If the proportions are distributed around 8\%,
    what is the variability of the distribution?
\item
    What is the formal name of the value you computed in~(c)?
\item
    Suppose the researchers' budget is reduced, and they are only
    able to collect 250 observations per sample, but they can still
    collect 1,000 samples.
    They build a new distribution of sample proportions.
    How will the variability of this new distribution compare
    to the variability of the distribution when each sample
    contained 800 observations?
\end{parts}
}{}

% 6

\eoce{\qt{Repeated student samples\label{repeated_student_samples_prop}}
Of all freshman at a large college, 16\% made the dean's list
in the current year.
As part of a class project, students randomly sample 40 students
and check if those students made the list.
They repeat this 1,000 times and build a distribution
of sample proportions.
\begin{parts}
\item
    What is this distribution called?
\item
    Would you expect the shape of this distribution to be
    symmetric, right skewed, or left skewed?
    Explain your reasoning.
\item
    Calculate the variability of this distribution.
\item
    What is the formal name of the value you computed in~(c)?
\item
    Suppose the students decide to sample again,
    this time collecting 90 students per sample,
    and they again collect 1,000 samples.
    They build a new distribution of sample proportions.
    How will the variability of this new distribution compare
    to the variability of the distribution when each sample
    contained 40 observations?
\end{parts}
}{}
}





%__________________
\section{Confidence intervals for a proportion}
\label{confidenceIntervals}

\index{confidence interval|(}

The sample proportion $\hat{p}$ provides a single plausible value
for the population proportion $p$. However, the sample proportion
isn't perfect and will have some \emph{standard error}
associated with it.
When stating an estimate for the population  proportion,
it is better practice to provide a plausible
\emph{range of values} instead of supplying just the point
estimate.


\subsection{Capturing the population parameter}

Using only a point estimate is like fishing in a murky
lake with a spear. We can throw a spear where we
saw a fish, but we will probably miss. On the other hand,
if we toss a net in that area, we have a good chance of
catching the fish.
A \term{confidence interval} is like fishing with a net,
and it represents a range of plausible values where we
are likely to find the population parameter.

If we report a point estimate $\hat{p}$, we probably
will not hit the exact population proportion. On the
other hand, if we report a range of plausible values,
representing a confidence interval,
we have a good shot at capturing the parameter.

\begin{exercisewrap}
\begin{nexercise}
If we want to be very certain we capture the population
proportion in an interval, should we use a wider interval
or a smaller interval?\footnotemark
\end{nexercise}
\end{exercisewrap}
\footnotetext{If we want to be more
    certain we will capture the fish, we might use a
    wider net. Likewise, we use a wider confidence interval
    if we want to be more certain that we capture the
    parameter.}

\subsection{Constructing a 95\% confidence interval}

Our sample proportion $\hat{p}$ is the most plausible
value of the population proportion, so it makes sense
to build a confidence interval around this point estimate.
The standard error\index{standard error (SE)|textbf}
provides a guide for how
large we should make the confidence interval.

The standard error represents the standard deviation
of the point estimate, and when the Central
Limit Theorem conditions are satisfied,
the point estimate closely follows a normal distribution.
In a normal distribution, 95\% of
the data is within 1.96~standard deviations of the mean.
Using this principle, we can construct a confidence
interval that extends 1.96~standard errors from the sample
proportion to be \termsub{95\% confident}
    {confident!95\% confident}\index{confident|textbf}
that the interval captures the population proportion:
\begin{align*}
\text{point estimate}\ &\pm\ 1.96 \times SE \\
\hat{p}\ &\pm\ 1.96 \times \sqrt{\frac{p (1 - p)}{n}}
%\label{95PercentConfidenceIntervalFormula}
\end{align*}
But what does ``95\% confident'' mean? Suppose we took
many samples and built a 95\% confidence interval from
each. Then about 95\% of those intervals would
contain the parameter,~$p$.
Figure~\ref{95PercentConfidenceInterval} shows the
process of creating 25 intervals from 25 samples
from the simulation in
Section~\ref{simulationForUnderstandingVariabilitySection},
where 24 of the resulting confidence intervals contain
the simulation's population proportion of
$p = \pewsolarparprop{}$, and one interval does not.

\D{\newpage}

\begin{figure}
  \centering
  \Figure[Twenty-five point estimates and confidence intervals from the simulations in Section~\ref{simulationForUnderstandingVariabilitySection} are shown. These intervals are shown relative to the population proportion p equals \pewsolarparprop{}. The point estimates vary around the true population proportion of 0.88, but most of their confidence intervals overlap the value p equals 0.88. One of the 25 intervals does not have a confidence interval that overlaps the population proportion, and this interval has been bolded. We might say that this confidence interval did not "capture" the parameter p equals 0.88.]{0.75}{95PercentConfidenceInterval}
  \caption{Twenty-five point estimates and confidence
      intervals from the simulations in
      Section~\ref{simulationForUnderstandingVariabilitySection}.
      These intervals are shown relative to the population
      proportion $p = \pewsolarparprop{}$.
      Only~1 of these~25
      intervals did not capture the population
      proportion, and this interval has been bolded.}
  \label{95PercentConfidenceInterval}
\end{figure}

\begin{examplewrap}
\begin{nexample}{In Figure~\ref{95PercentConfidenceInterval},
one interval does not contain $p = \pewsolarparprop{}$.
Does this imply that the population proportion used
in the simulation could not have been
$p = \pewsolarparprop{}$?}
Just as some observations naturally
occur more than 1.96~standard deviations
from the mean, some point estimates will be more than
1.96~standard errors from the parameter of interest.
A confidence interval only provides a plausible range
of values.
While we might say other values are implausible
based on the data, this does not mean they are impossible.
\end{nexample}
\end{examplewrap}

\begin{onebox}{95\% confidence interval for a parameter}
  \index{confidence interval!95\%}
  When the distribution of a point estimate qualifies for
  the Central Limit Theorem and
  therefore closely follows a normal distribution,
  we can construct a 95\% confidence interval as
  \begin{align*}
  \text{point estimate} &\pm 1.96 \times SE
  \end{align*}
  % This confidence interval only accounts for sampling error,
  % not bias.
\end{onebox}

\begin{examplewrap}
\begin{nexample}{In Section~\ref{pointEstimates} we learned about
    a Pew Research poll where
    \pewsolarpollpercent{} of a random sample of
    \pewsolarpollsize{} American adults
    supported expanding the role of solar power.
    Compute and
    interpret a 95\% confidence interval for the population
    proportion.} \label{95p_ci_for_pew_solar_support}
  We earlier confirmed that $\hat{p}$ follows a normal
  distribution and has a standard error of
  $SE_{\hat{p}} = \pewsolarpollse{}$.
  To compute the 95\% confidence interval, plug the
  point estimate $\hat{p} = \pewsolarpollprop{}$ and
  standard error into the 95\% confidence interval formula:
  \begin{align*}
  \hat{p} \pm 1.96 \times SE_{\hat{p}}
  \quad\to\quad
  \pewsolarpollprop{} \pm 1.96 \times \pewsolarpollse{}
  \quad\to\quad
  (0.8674, 0.9066)
  \end{align*}
  We are 95\% confident that the actual proportion of
  American adults who support expanding solar power is
  between 86.7\% and 90.7\%.
  (It's common to round to the nearest percentage point
  or nearest tenth of a percentage point when reporting
  a confidence interval.)
\end{nexample}
\end{examplewrap}


\D{\newpage}

\subsection{Changing the confidence level}
\label{changingTheConfidenceLevelSection}

\index{confidence interval!confidence level|(}

Suppose we want to consider confidence intervals where the confidence
level is higher than 95\%, such as a confidence
level of~99\%. Think back to the analogy about trying to catch a fish:
if~we want to be more sure that we will catch the fish, we should use
a wider net. To create a 99\% confidence level, we must also widen our
95\% interval. On the other hand, if we want an interval with lower
confidence, such as 90\%, we could use a slightly narrower
interval than our original 95\% interval.

The 95\% confidence interval structure provides guidance in
how to make intervals with different confidence levels.
The general 95\% confidence interval for a point estimate
that follows a normal distribution is
\begin{eqnarray*}
\text{point estimate}\ \pm\ 1.96 \times SE
\end{eqnarray*}
There are three components to this interval: the point estimate,
``1.96'', and the standard error. The choice of $1.96\times SE$ was
based on capturing 95\% of the data since the estimate is within
1.96 standard errors of the parameter about 95\% of the time.
The choice of 1.96 corresponds to a 95\% confidence level. 

\begin{exercisewrap}
\begin{nexercise} \label{leadInForMakingA99PercentCIExercise}
If $X$ is a normally distributed random variable, what is the
probability of the value $X$ being
within 2.58~standard deviations of the mean?\footnotemark
\end{nexercise}
\end{exercisewrap}
\footnotetext{This is equivalent to asking how often the
    Z-score will be larger than -2.58 but less than 2.58.
    For a picture, see Figure~\ref{choosingZForCI}.
    To determine this probability, we can use statistical software,
    a calculator, or a table to look up -2.58 and 2.58 for
    a normal distribution: 0.0049 and 0.9951.
    Thus, there is a $0.9951-0.0049 \approx 0.99$ probability
    that an unobserved normal random variable
    $X$ will be within 2.58~standard deviations of $\mu$.}

Guided Practice~\ref{leadInForMakingA99PercentCIExercise} highlights
that 99\% of the time a normal random variable will be within
2.58~standard deviations of the mean.
To create a 99\% confidence interval, change 1.96 in the 95\%
confidence interval formula to be $2.58$.
That is, the formula
for a 99\% confidence interval is
\begin{align*}
\text{point estimate}\ \pm\ 2.58 \times SE
%\label{99PercCIForProp}
\end{align*}

\begin{figure}[h]
  \centering
  \Figure[A standard normal distribution is shown, where "standard" is the term used to indicate that the normal distribution is centered at 0 and has a standard deviation of 1. Portions of the normal distribution have been shaded. First, the central 95\% portion of the distribution has been shaded in a dark blue, and this region has an annotation stating "95\%, extends from -1.96 to 1.96". Recall that the value of 1.96 closely matches our 68-95-99.7 rule for the normal distribution, which had stated that about 95\% of the area under the normal distribution lied within 2 standard deviations of the mean. Second, a slightly broader region of the normal distribution is shaded, in this case from about -2.5 to positive 2.5, and this has an annotation stating, "99\%, extends -2.58 to 2.58". The values described here -- 1.96 and 2.58 -- are the z-star values that we would use for 95\% and 99\% confidence intervals, respectively.]{}{choosingZForCI}
  \caption{The area between -$z^{\star}$ and $z^{\star}$ increases as
      $z^{\star}$ becomes larger. If the confidence level is 99\%,
      we choose $z^{\star}$ such that 99\% of a normal
      normal distribution is between -$z^{\star}$ and $z^{\star}$,
      which corresponds to 0.5\%
      in the lower tail and 0.5\% in the upper tail:
      $z^{\star}=2.58$.}
\label{choosingZForCI}
\index{confidence interval!confidence level|)}
\end{figure}

\D{\newpage}

This approach -- using the Z-scores in the
normal model to compute confidence levels --
is appropriate when a point estimate such as $\hat{p}$
is associated with a normal distribution.
%For the context of sample proportions, the
%normal distribution is reasonable when the sample
%observations are independent and the success-failure condition
%holds ($np$ and $n(1-p)$ are both at least 10).
For some other point estimates, a normal model is not a good fit;
in these cases, we'll use alternative distributions that better
represent the sampling distribution.

\begin{onebox}{Confidence interval using any confidence level}
  If a point estimate closely follows a normal model
  with standard error $SE$, then a confidence interval
  for the population parameter is
  \begin{align*}
  \text{point estimate}\ \pm\ z^{\star} \times SE
  \end{align*}
  where $z^{\star}$ corresponds to the confidence
  level selected.
\end{onebox}

Figure~\ref{choosingZForCI} provides a picture of how to identify
$z^{\star}$ based on a confidence level. We~select $z^{\star}$
so that the area between -$z^{\star}$ and $z^{\star}$ in the
standard normal distribution\index{standard normal distribution}\index{normal distribution!standard}\index{distribution!normal!standard},
$N(0, 1)$, corresponds to the confidence level.

\begin{onebox}{Margin of error}
  \label{marginOfErrorTermBox}%
  In a confidence interval, $z^{\star}\times SE$ is called the
  \term{margin of error}.
\end{onebox}

\begin{examplewrap}
\begin{nexample}{Use the data in
    Example~\ref{95p_ci_for_pew_solar_support} to
    create a 90\% confidence interval for the proportion of American
    adults that support expanding the use of solar power.
    We have already verified conditions for normality.}
  We first find $z^{\star}$ such that 90\% of the distribution falls
  between -$z^{\star}$ and $z^{\star}$ in the
  \index{standard normal distribution}%
  \index{normal distribution!standard}%
  \index{distribution!normal!standard}%
  standard normal distribution, $N(\mu = 0, \sigma = 1)$.
  We can do this using a graphing calculator,
  statistical software, or a probability table by looking for an
  upper tail of 5\% (the other 5\% is in the lower tail):
  $z^{\star}=1.65$.
  The 90\% confidence interval can then be computed as
  \begin{align*}
  \hat{p}\ \pm\ 1.6449 \times SE_{\hat{p}}
      \quad\to\quad 0.887\ \pm\ 1.65 \times 0.0100
      \quad\to\quad (0.8705, 0.9034)
  \end{align*}
  That is, we are 90\% confident that 87.1\% to 90.3\% of American
  adults supported the expansion of solar power in 2018.
\end{nexample}
\end{examplewrap}

\newcommand{\onepropconfintsummary}[0]{
\begin{onebox}{Confidence interval for a single proportion}
  Once you've determined a one-proportion confidence interval
  would be helpful for an application,
  there are four steps to constructing the interval:
  \begin{description}
  \item[Prepare.]
      Identify $\hat{p}$ and $n$, and determine what
      confidence level you wish to use.
  \item[Check.]
      Verify the conditions to ensure $\hat{p}$
      is nearly normal.
      For one-proportion confidence intervals,
      use $\hat{p}$ in place of $p$ to check
      the success-failure condition.
  \item[Calculate.]
      If the conditions hold, compute $SE$ using $\hat{p}$,
      find $z^{\star}$, and construct the interval.
  \item[Conclude.]
      Interpret the confidence interval in the context
      of the problem.
  \end{description}
\end{onebox}
}
\onepropconfintsummary{}


\D{\newpage}

\subsection{More case studies}

\index{data!Ebola poll|(}

\newcommand{\wsjebolapollsize}{1042}
\newcommand{\wsjebolapollsizecomma}{1,042}
\newcommand{\wsjebolapollprop}{0.82}
\newcommand{\wsjebolapollpropcomplement}{0.18}
\newcommand{\wsjebolapollpercent}{82}
\newcommand{\wsjebolapollpercentcomplement}{18}
\newcommand{\wsjebolapollcount}{854}
\newcommand{\wsjebolapollcountcomplement}{188}
\newcommand{\wsjebolapollse}{0.012}


In New York City on October 23rd, 2014, a doctor who had recently been
treating Ebola patients in Guinea went to the hospital with a slight fever
and was subsequently diagnosed with Ebola. Soon thereafter,
an NBC~4 New York/The Wall Street Journal/Marist Poll found that
\wsjebolapollpercent{}\% of New Yorkers favored a ``mandatory 21-day
quarantine for anyone who has come in contact with an Ebola
patient''. This poll included responses
of \wsjebolapollsizecomma{} New York adults between
Oct 26th and~28th, 2014.
%\footnote{This survey, like the others
%  you'll see in this book, ...}
%We may want a confidence interval for the proportion of New York
%adults who favored a mandatory quarantine of anyone who had been in
%contact with an Ebola patient.

\begin{examplewrap}
\begin{nexample}{What is the point estimate in this case,
    and is it reasonable to
    use a normal distribution to model that point estimate?}
  The point estimate, based on a sample of size $n = \wsjebolapollsize{}$,
  is $\hat{p} = \wsjebolapollprop{}$.
  To check whether $\hat{p}$ can be reasonably
  modeled using a normal distribution, we check independence
  (the poll is based on a simple random sample) and the
  success-failure condition
  ($\wsjebolapollsize{} \times \hat{p} \approx \wsjebolapollcount{}$
  and $\wsjebolapollsize{} \times (1 - \hat{p})
      \approx \wsjebolapollcountcomplement{}$,
  both easily greater than~10).
  With the conditions met, we are assured
  that the sampling distribution of $\hat{p}$ can be
  reasonably modeled using a normal distribution.
\end{nexample}
\end{examplewrap}

\begin{examplewrap}
\begin{nexample}{Estimate the standard error of
    $\hat{p} = \wsjebolapollprop{}$ from the Ebola survey.}
  \label{seOfPropOfNYEbolaSurvey}%
  We'll use the substitution approximation of
  $p \approx \hat{p} = \wsjebolapollprop{}$ to compute
  the standard error:
  \begin{align*}
  SE_{\hat{p}}
    = \sqrt{\frac{p(1-p)}{n}}
    \approx \sqrt{\frac{\wsjebolapollprop{}
        (1 - \wsjebolapollprop{})}{\wsjebolapollsize{}}}
    = \wsjebolapollse{}
  \end{align*}
\end{nexample}
\end{examplewrap}

\begin{examplewrap}
\begin{nexample}{Construct a 95\% confidence interval for $p$,
    the proportion of New York adults who supported a quarantine
    for anyone who has come into contact with an Ebola patient.}
  \label{ex_ci_ny_ebola_quarantine}%
  Using the standard error $SE = 0.012$ from
  Example~\ref{seOfPropOfNYEbolaSurvey},
  the point estimate \wsjebolapollprop{}, and $z^{\star} = 1.96$
  for a 95\% confidence level, the confidence interval is
  \begin{eqnarray*}
  \text{point estimate} \ \pm\ z^{\star} \times SE
    \quad\to\quad \wsjebolapollprop{} \ \pm\ 1.96\times \wsjebolapollse{}
    \quad\to\quad (0.796, 0.844)
  \end{eqnarray*}
  We are 95\% confident that the proportion of New York adults
  in October 2014 who supported a quarantine for anyone who had come
  into contact with an Ebola patient was between 0.796 and 0.844.
\index{data!Ebola poll|)}
\end{nexample}
\end{examplewrap}

\begin{exercisewrap}
\begin{nexercise}
Answer the following two questions about the confidence interval
from Example~\ref{ex_ci_ny_ebola_quarantine}:\footnotemark{}
\begin{enumerate}[(a)]
\item
    What does 95\% confident mean in this context?
\item
    Do you think the confidence interval is still valid
    for the opinions of New Yorkers today?
\end{enumerate}
\end{nexercise}
\end{exercisewrap}
\footnotetext{(a)~If we took many such samples and computed
  a 95\% confidence interval for each, then about 95\% of those
  intervals would contain the actual proportion of New York
  adults who supported a quarantine for anyone who has come into
  contact with an Ebola patient. \\
  (b)~Not necessarily. The poll was taken at a
  time where there was a huge public safety concern.
  Now that people have had some time to step back,
  they may have changed their opinions.
  We would need to run a new poll if we wanted to get an
  estimate of the current proportion of New York adults who
  would support such a quarantine period.}

\D{\newpage}

\index{data!wind turbine survey|(}

\newcommand{\pewwindpollsize}{\pewsolarpollsize}
\newcommand{\pewwindpollprop}{0.848}
\newcommand{\pewwindpollpropcomplement}{0.152}
\newcommand{\pewwindpollpercent}{84.8}
\newcommand{\pewwindpollpercentcomplement}{15.2}
\newcommand{\pewwindpollcount}{848}
\newcommand{\pewwindpollcountcomplement}{152}
\newcommand{\pewwindpollse}{0.0114}

\begin{exercisewrap}
\begin{nexercise}
\label{pew_wind_turbine_support_normal_dist_gp}%
In the Pew Research poll about solar energy, they
also inquired about other forms of energy,
and \pewwindpollpercent{}\% of the \pewwindpollsize{}
respondents supported expanding the use of wind
turbines.\footnotemark{}
\begin{enumerate}[(a)]
\item
    Is it reasonable to model the proportion
    of US adults who support expanding wind turbines
    using a normal distribution?
\item
    Create a 99\% confidence interval for the level of American
    support for expanding the use of wind turbines for power
    generation.
\end{enumerate}
\end{nexercise}
\end{exercisewrap}
\footnotetext{(a)~The survey was a random sample
  and counts are both $\geq 10$
  ($\pewwindpollsize{} \times \pewwindpollprop{}
    = \pewwindpollcount{}$
  and $\pewwindpollsize{} \times \pewwindpollpropcomplement{}
    = \pewwindpollcountcomplement$),
  so independence and the success-failure condition
  are satisfied, and
  $\hat{p} = \pewwindpollprop{}$ can be
  modeled using a normal distribution. \\
  (b)~Guided
  Practice~\ref{pew_wind_turbine_support_normal_dist_gp}
  confirmed that $\hat{p}$ closely follows
  a normal distribution, so we can use the C.I.~formula:
  \begin{align*}
  \text{point estimate} \pm z^{\star} \times SE
  \end{align*}
  In this case, the point estimate is
  $\hat{p} = \pewwindpollprop{}$.
  For a 99\% confidence interval, $z^{\star} = 2.58$.
  Computing the standard error:
  $SE_{\hat{p}}
    = \sqrt{\frac{\pewwindpollprop{}(1 - \pewwindpollprop{})}
        {\pewwindpollsize{}}}
    = \pewwindpollse{}$.
  Finally, we compute the interval as
  $\pewwindpollprop{} \pm 2.58 \times \pewwindpollse{}
    \to (0.8186,   0.8774)$.
  It is also important to \emph{always} provide an interpretation
  for the interval: we are 99\% confident the proportion of
  American adults that support expanding the use of wind
  turbines in 2018 is between 81.9\% and 87.7\%.}

We can also construct confidence intervals for other
parameters, such as a population mean.
In these cases, a confidence interval would be computed
in a similar way to that of a single proportion:
a point estimate plus/minus some margin of error.
We'll dive into these details in later chapters.


\subsection{Interpreting confidence intervals}
\label{interpretingCIs}

\index{confidence interval!interpretation|(}

In each of the examples, we described the confidence
intervals by putting them into the context of the data and also
using somewhat formal language:
\begin{description}
  \item[Solar.] We are 90\% confident that 87.1\% to 90.4\% of
      American adults support the expansion of solar power in 2018.
  \item[Ebola.] We are 95\% confident that the proportion
      of New York adults in October 2014 who supported a quarantine
      for anyone who had come into contact with an Ebola patient was
      between 0.796 and 0.844.
  \item[Wind Turbine.] We are 99\% confident the proportion of
      Americans adults that support expanding the use of wind
      turbines is between 81.9\% and 87.7\% in 2018.
\end{description}
First, notice that the statements are always about the population
parameter, which considers \emph{all} American adults for the
energy polls or \emph{all} New York adults for the quarantine poll.

We also avoided another common mistake:
\emph{incorrect} language might try to describe the confidence interval
as capturing the population parameter with a certain probability.
Making a probability interpretation is a common error:
while it might be useful to think of it as a probability,
the confidence level only quantifies how plausible
it is that the parameter is in the given interval.

Another important consideration of confidence intervals is that they
are \emph{only about the population parameter}.
A confidence interval says nothing about individual
observations or point estimates.
Confidence intervals only provide a plausible range for
population parameters.

\index{bias|(}
Lastly, keep in mind the methods we discussed only apply
to sampling error, not to bias.
If a data set is collected in a way that will tend to
systematically under-estimate
(or over-estimate) the population parameter, the techniques
we have discussed will not address that problem.
Instead, we rely on careful data collection procedures to
help protect against bias in the examples we have considered,
which is a common practice employed by data scientists
to combat bias.
\index{bias|)}

\begin{exercisewrap}
\begin{nexercise}
Consider the 90\% confidence interval for the solar
energy survey: 87.1\% to 90.4\%.
If~we ran the survey again, can we say that we're
90\% confident that the new survey's proportion
will be between 87.1\% and 90.4\%?\footnotemark{}
\end{nexercise}
\end{exercisewrap}
\footnotetext{
  No, a confidence interval only provides a range of plausible
  values for a parameter,
  not future point estimates.}




\index{data!wind turbine survey|)}
\index{data!solar survey|)}
\index{confidence interval!interpretation|)}

\CalculatorVideos{confidence intervals for a single proportion}

\index{confidence interval|)}


{\exercisesheader{}

% 7

\eoce{\qt{Chronic illness, Part I\label{chronic_illness_intro}} 
In 2013, the Pew Research Foundation reported that ``45\% of U.S. adults report 
that they live with one or more chronic conditions''.
\footfullcite{data:pewdiagnosis:2013} However, this value was based on a sample, 
so it may not be a perfect estimate for the population parameter of interest on 
its own. The study reported a standard error of about 1.2\%, and a normal model 
may reasonably be used in this setting. Create a 95\% confidence interval for 
the proportion of U.S. adults who live with one or more chronic conditions. Also 
interpret the confidence interval in the context of the study.
}{}

% 8

\eoce{\qt{Twitter users and news, Part I\label{twitter_users_intro}} 
A poll conducted in 2013 found that 52\% of U.S. adult Twitter users 
get at least some news on Twitter.\footfullcite{data:pewtwitternews:2013}. 
The standard error for this estimate was 2.4\%, and a normal distribution 
may be used to model the sample proportion. Construct a 99\% confidence 
interval for the fraction of U.S. adult Twitter users who get some 
news on Twitter, and interpret the confidence interval in context.
}{}

% 9

\eoce{\qt{Chronic illness, Part II\label{chronic_illness_tf}} In 2013, the Pew Research Foundation reported that 
``45\% of U.S. adults report that they live with one or more chronic 
conditions'', and the standard error for this estimate is 1.2\%. Identify each 
of the following statements as true or false. Provide an explanation to justify 
each of your answers.
\begin{parts}
\item We can say with certainty that the confidence interval from 
Exercise~\ref{chronic_illness_intro} contains the true percentage of U.S. adults who 
suffer from a chronic illness.
\item If we repeated this study 1,000 times and constructed a 95\% confidence 
interval for each study, then approximately 950 of those confidence intervals 
would contain the true fraction of U.S. adults who suffer from chronic illnesses.
\item The poll provides statistically significant evidence (at the 
$\alpha = 0.05$ level) that the percentage of U.S. adults who suffer from 
chronic illnesses is below 50\%.
\item Since the standard error is 1.2\%, only 1.2\% of people in the study 
communicated uncertainty about their answer.
\end{parts}
}{}

% 10

\eoce{\qt{Twitter users and news, Part II\label{twitter_users_tf}} A poll conducted in 2013 found that 52\% of 
U.S. adult Twitter users get at least some news on Twitter, and the standard 
error for this estimate was 2.4\%. Identify each of the following statements as 
true or false. Provide an explanation to justify each of your answers.
\begin{parts}
\item The data provide statistically significant evidence that more than half of 
U.S. adult Twitter users get some news through Twitter. Use a significance level 
of $\alpha = 0.01$.
(This part uses concepts from Section~\ref{hypothesisTesting} and will be
corrected in a future edition.)
\item Since the standard error is 2.4\%, we can conclude that 97.6\% of all U.S. 
adult Twitter users were included in the study.
\item If we want to reduce the standard error of the estimate, we should collect 
less data.
\item If we construct a 90\% confidence interval for the percentage of U.S. 
adults Twitter users who get some news through Twitter, this confidence interval 
will be wider than a corresponding 99\% confidence interval.
\end{parts}
}{}

\D{\newpage}

% 11

\eoce{\qt{Waiting at an ER, Part I\label{er_wait_intro_prop_ok}} A hospital administrator 
hoping to improve wait times decides to estimate the average emergency 
room waiting time at her hospital. She collects a simple random sample 
of 64 patients and determines the time (in minutes) between when they 
checked in to the ER until they were first seen by a doctor. A 95\% 
confidence interval based on this sample is (128 minutes, 147 minutes), 
which is based on the normal model for the mean. Determine whether the 
following statements are true or false, and explain your reasoning.
\begin{parts}
\item We are 95\% confident that the average waiting time of these 64 emergency 
room patients is between 128 and 147 minutes.
\item We are 95\% confident that the average waiting time of all patients at 
this hospital's emergency room is between 128 and 147 minutes.
\item 95\% of random samples have a sample mean between 128 and 147 minutes.
\item A 99\% confidence interval would be narrower than the 95\% confidence 
interval since we need to be more sure of our estimate.
\item The margin of error is 9.5 and the sample mean is 137.5.
\item In order to decrease the margin of error of a 95\% confidence interval to 
half of what it is now, we would need to double the sample size.
(Hint: the margin of error for a mean scales in the same way with sample size
as the margin of error for a proportion.)
\end{parts}
}{}

% 12

\eoce{\qt{Mental health\label{mental_health}}
The General Social Survey asked the question:
``For how many days during the past 30 days was your 
mental health, which includes stress, depression,
and problems with emotions, not good?"
Based on responses from 1,151 US residents,
the survey reported a 95\% confidence interval of
3.40 to 4.24 days in 2010.
\begin{parts}
\item
    Interpret this interval in context of the data.
\item
    What does ``95\% confident" mean? Explain in the
    context of the application.
\item
    Suppose the researchers think a 99\% confidence level
    would be more appropriate for this interval.
    Will this new interval be smaller or wider than the
    95\% confidence interval?
\item
    If a new survey were to be done with 500 Americans,
    do you think the standard error of the estimate be
    larger, smaller, or about the same.
\end{parts}
}{}

% 13

\eoce{\qt{Website registration\label{website_registration_design_prop}}
A website is trying to increase registration for first-time visitors,
exposing 1\% of these visitors to a new site design.
Of 752 randomly sampled visitors over a month who saw the
new design, 64 registered.
\begin{parts}
\item
    Check any conditions required for constructing a confidence
    interval.
\item
    Compute the standard error.
\item
    Construct and interpret a 90\% confidence interval for the
    fraction of first-time visitors of the site who would register
    under the new design
    (assuming stable behaviors by new visitors over time).
\end{parts}
}{}

% 14

\eoce{\qt{Coupons driving visits\label{store_coupon_prop}}
A store randomly samples 603 shoppers over the course of a year
and finds that 142 of them made their visit because of a coupon
they'd received in the mail.
Construct a 95\% confidence interval for the fraction of all shoppers
during the year whose visit was because of a coupon they'd received
in the mail.
}{}
}






%__________________
\section{Hypothesis testing for a proportion}
\label{hypothesisTesting}

\index{hypothesis testing|(}

The following question comes from a book written by
Hans Rosling, Anna Rosling R{\"o}nnlund, and Ola Rosling
called \emph{\oiRedirect{amazon_factfulness}{Factfulness}}:
\begin{quote}
  {\em How many of the world's 1~year old children today
  have been vaccinated against some disease:
  \begin{enumerate}[a.]
  \setlength{\itemsep}{0mm}
  \item 20\%
  \item 50\%
  \item 80\%
  \end{enumerate}}
\end{quote}
Write down what your answer (or guess),
and when you're ready, find the answer in the
footnote.\footnote{The correct answer is (c):
  80\% of the world's 1~year olds have been vaccinated
  against some disease.}

In this section,
we'll be exploring how people with a 4-year college
degree perform on this and other world health questions
as we learn about hypothesis tests, which are
a framework used to rigorously evaluate competing
ideas and claims.

\newcommand{\roslingAsize}{50}
\newcommand{\roslingAprop}{0.24}
\newcommand{\roslingApropcomplement}{0.76}
\newcommand{\roslingApercent}{24}
\newcommand{\roslingApercentcomplement}{76}
\newcommand{\roslingAcount}{12}
\newcommand{\roslingAcountcomplement}{38}
\newcommand{\roslingAse}{0.060}
% n <- 50; x <- 12; (p <- x/n); (se <- sqrt(p * (1 - p) / n)); p + c(-1, 1) * 1.96 * se

%There's an adage in United States financial markets that
%it is better to get out of investments during the six ``summer''
%months: \emph{sell in May and go away!}\footnote{Summer in the
%northern hemisphere, anyways. \rotatebox[origin=c]{180}{(Hello
%Australia!)}} While this clever saying does rhyme, that doesn't
%mean it is sound financial advice. Let's investigate.

%so is this is a pretty strong statement, since the stock
%market has a very strong historical trend of moving upwards.
%
%To test this theory, we've retrieved the 
%
%If this adage holds meaning, we would expect that about half of the time the market would be in decline each year. Of course, we also would care to learn if it happens to be up more often than not, so we will also check that!

%Finance is a field where a lot of money can be made or lost. We're going to explore a few topics in relation to the US stock market and 

%The United States stock market moves down and up in unpredictable ways, and it can be useful to look for small inconsistencies in the market behavior that can be leveraged for minor gains. We will test three theories about the stock market in this section:

%\item We might wonder whether the stock market is more likely to go up or down in any given day. Of course, the average return each day has been historically positive, and so this exploration will allow us to better understand if that is also reflected in the fraction of days that are up.
%\item Each week there is a 65.5 hours window from the time the market closes on Friday to when it opens on the weekdays. That's a lot of time for good news and bad news that can affect the returns on Mondays. We'll see whether we 

%The market has the same chance of going up or down on any given day of the week. For example, we would be interested to learn if the stock market goes up a little more often on, say, Fridays, that could be useful for 


\subsection{Hypothesis testing framework}

We’re interested in understanding how much people know
about world health and development.
If we take a multiple choice
world health question, then we might like to understand~if
\begin{description}
\item[$\mathbf{H_0}$:]
    People never learn these particular topics and their
    responses are simply equivalent to random guesses.
\item[$\mathbf{H_A}$:]
    People have knowledge that helps them do better
    than random guessing, or perhaps, they have false knowledge
    that leads them to actually do worse than random guessing.
\end{description}
These competing ideas are called \term{hypotheses}.
We call $H_0$ the null hypothesis and $H_A$ the alternative
hypothesis.
When there is a subscript 0 like in $H_0$,
data scientists pronounce it as ``nought''
(e.g.~$H_0$ is pronounced ``H-nought'').

\begin{onebox}{Null and alternative hypotheses}
  The \term{null hypothesis ($H_0$)} often represents
  a skeptical perspective or a claim to be tested.
  The \term{alternative hypothesis ($H_A$)} represents an
  alternative claim under consideration and is often
  represented by a range of possible parameter values.
  \stdvspace{}
  
  Our job as data scientists is to play the role of a skeptic:
  before we buy into the alternative hypothesis, we need to
  see strong supporting evidence.
\end{onebox}

The null hypothesis often represents a skeptical position
or a perspective of ``no difference''.
In our first example, we'll consider whether
the typical person does any different than random guessing
on Roslings' question about infant vaccinations.

The alternative hypothesis generally represents a new
or stronger perspective. In the case of the question
about infant vaccinations,
it would certainly be interesting to learn whether
people do better than random guessing, since that would
mean that the typical person knows something about
world health statistics.
It would also be very interesting if we learned
that people do \emph{worse} than random guessing,
which would suggest people believe
incorrect information about world health.

The hypothesis testing framework is a very general tool, and we often use it without a second thought. If a person makes a somewhat unbelievable claim, we are initially skeptical. However, if~there is sufficient evidence that supports the claim, we set aside our skepticism and reject the null hypothesis in favor of the alternative. The hallmarks of hypothesis testing are also found in the US~court system. 

\D{\newpage}

\begin{exercisewrap}
\begin{nexercise} \label{hypTestCourtExample}
A US court considers two possible claims about a defendant: she is either innocent or guilty. If we set these claims up in a hypothesis framework, which would be the null hypothesis and which the alternative?\footnotemark
\end{nexercise}
\end{exercisewrap}
\footnotetext{The jury considers whether the evidence is so
    convincing (strong) that there is no reasonable doubt
    regarding the person's guilt;
    in such a case, the jury rejects innocence
    (the null hypothesis) and concludes the defendant
    is guilty (alternative hypothesis).}

Jurors examine the evidence to see whether it convincingly
shows a defendant is guilty.
Even if the jurors leave unconvinced of guilt beyond
a reasonable doubt, this does not mean they believe the
defendant is innocent.
This is also the case with hypothesis testing:
\emph{even if we fail to reject the null hypothesis,
we typically do not accept the null hypothesis as true}.
Failing to find strong evidence for the alternative
hypothesis is not equivalent to accepting
the null hypothesis.

When considering Roslings' question about infant vaccination,
the null hypothesis represents the notion that the people
we will be considering -- college-educated adults --
are as accurate as random guessing.
That is, the proportion
$p$ of respondents who pick the correct
answer, that 80\% of 1~year olds have been vaccinated
against some disease, is about 33.3\%
(or 1-in-3 if wanting to be perfectly precise).
The alternative hypothesis is that this proportion is something
other than 33.3\%. While it's helpful to write these hypotheses
in words, it can be useful to write them using mathematical
notation:
\begin{description}
\item[$H_0$:] $p = 0.333$
\item[$H_A$:] $p \neq 0.333$
\end{description}
In this hypothesis setup, we want to make a conclusion about
the population parameter $p$. The value we are comparing the
parameter to is called the \term{null value}, which in this
case is 0.333. It's common to label the null value with the
same symbol as the parameter but with a subscript~`0'.
That is, in this case, the null value is $p_0 = 0.333$
(pronounced ``p-nought equals 0.333'').

\begin{examplewrap}
\begin{nexample}{It may seem impossible that the
    proportion of people who get the correct answer
    is \emph{exactly} 33.3\%. If we don't believe the
    null hypothesis, should we simply reject it?}
  No. While we may not buy into the notion that
  the proportion is exactly 33.3\%, the hypothesis testing
  framework requires that there be strong evidence before
  we reject the null hypothesis and conclude something
  more interesting.

  After all, even if we don't believe the proportion is
  \emph{exactly} 33.3\%, that doesn't really tell us anything
  useful! We would still be stuck with the original question:
  do people do better or worse than random guessing on
  Roslings' question?
  Without data that strongly
  points in one direction or the other, it is both
  uninteresting and pointless to reject $H_0$.
\end{nexample}
\end{examplewrap}

\begin{exercisewrap}
\begin{nexercise}
  Another example of a real-world hypothesis testing situation
  is evaluating whether a new drug is better or worse
  than an existing drug at treating a particular disease.
  What should we use for the null and alternative hypotheses in
  this case?\footnotemark{}
\end{nexercise}
\end{exercisewrap}
\footnotetext{The null hypothesis ($H_0$) in this case is
    the declaration of \emph{no difference}: the drugs are equally
    effective. The alternative hypothesis ($H_A$) is that the
    new drug performs differently than the original,
    i.e. it could perform better or worse.}


\D{\newpage}

\subsection{Testing hypotheses using confidence intervals}
\label{utilizingOurCI}

We will use the \data{rosling\us{}responses}
data set to evaluate
the hypothesis test evaluating whether college-educated adults
who get the question about infant vaccination correct is different
from 33.3\%.
This data set summarizes the answers of \roslingAsize{}
college-educated adults.
Of these \roslingAsize{} adults, \roslingApercent{}\%~of
respondents got the question correct that 80\% of 1~year olds
have been vaccinated against some disease.

Up until now, our discussion has been philosophical.
However, now that we have data, we might ask ourselves:
does the data provide strong evidence that the proportion
of all college-educated adults who would answer this
question correctly is different than 33.3\%?

We learned in Section~\ref{pointEstimates} that there is
fluctuation from one sample to another, and it is unlikely
that our sample proportion, $\hat{p}$,
will exactly equal $p$, but we want to make
a conclusion about~$p$.
We~have a nagging concern:
is this deviation of \roslingApercent{}\%
from 33.3\% simply due to chance,
or~does the data provide strong evidence that the
population proportion is different from 33.3\%?

In Section~\ref{confidenceIntervals}, we learned how to
quantify the uncertainty in our estimate using confidence
intervals. 
The same method for measuring variability can be useful
for the hypothesis test.

\begin{examplewrap}
\begin{nexample}{Check whether it is reasonable to construct
    a confidence interval for $p$ using the sample data, and
    if so, construct a 95\% confidence interval.}
  The conditions are met for $\hat{p}$ to be approximately
  normal: the data come from a simple random sample (satisfies
  independence), and $n\hat{p} = \roslingAcount$ and
  $n(1 - \hat{p}) = \roslingAcountcomplement$ are both
  at least 10 (success-failure condition).

  To construct the confidence interval, we will need to identify
  the point estimate ($\hat{p} = \roslingAprop$),
  the critical value for
  the 95\% confidence level ($z^{\star} = 1.96$), and the standard
  error of $\hat{p}$
  ($SE_{\hat{p}} = \sqrt{\hat{p}(1 - \hat{p}) / n} = \roslingAse$).
  With those pieces, the confidence interval for $p$ can be
  constructed:
  \begin{align*}
    &\hat{p} \pm z^{\star} \times SE_{\hat{p}} \\
    &\roslingAprop \pm 1.96 \times \roslingAse \\
    &(0.122, 0.358)
  \end{align*}
  We are 95\% confident that the proportion of all
  college-educated adults to correctly answer this
  particular question about infant vaccination is between
  12.2\% and 35.8\%.
\end{nexample}
\end{examplewrap}
%At a first glance, it looks like it might be. After all,
%36\% isn't that close to 50\%, so maybe this data constitutes
%\emph{strong evidence}. We need to 

Because the null value in the hypothesis test is $p_0 = 0.333$,
which falls within the range of plausible values from the
confidence interval, we cannot say the null value is
implausible.\footnote{Arguably this method is slightly imprecise.
  As we'll see in a few pages, the standard error is often
  computed slightly differently in the context of a hypothesis
  test for a proportion.}
That is, the data do not provide sufficient evidence to reject
the notion that the performance of college-educated
adults was different than random guessing,
and we do not reject the null hypothesis,~$H_0$.

\begin{examplewrap}
\begin{nexample}{Explain why we cannot conclude that
    college-educated adults simply guessed on the
    infant vaccination question.}
  While we failed to reject $H_0$, that does not
  necessarily mean the null hypothesis is true.
  Perhaps there was an actual difference,
  but we were not able to detect it with the
  relatively small sample of~\roslingAsize{}.

%  Second, we are only evaluating the proportion,
%  and if the population proportion is 0.333,
%  there are still multiple ways to arrive at that proportion.
%  For example,
%  perhaps some adults guessed but others did not.
%  And of those who didn't guess,
%  their past knowledge simply wasn't very useful on this
%  question and so most of them still got it wrong.
\end{nexample}
\end{examplewrap}

\begin{onebox}{Double negatives can sometimes be used in statistics}
  In many statistical explanations, we use double negatives.
  For instance, we might say that the null hypothesis is
  \emph{not implausible} or we \emph{failed to reject}
  the null hypothesis.
  Double negatives are used to communicate that while we
  are not rejecting a position, we are also not saying it
  is correct.
\end{onebox}

\begin{exercisewrap}
\begin{nexercise}\label{roslingB_hypothesis_setup}%
Let's move onto a second question posed by the Roslings:
\begin{quote}{\em
  There are 2 billion children in the world today
  aged 0-15 years old, how many children will there
  be in year 2100 according to the United Nations?
  \begin{enumerate}[a.]
  \setlength{\itemsep}{0mm}
  \item 4 billion.
  \item 3 billion.
  \item 2 billion.
  \end{enumerate}
}\end{quote}
Set up appropriate hypotheses to evaluate whether
college-educated adults are better than random guessing
on this question.
Also, see if you can guess the correct answer before checking
the answer in the footnote!\footnotemark
\end{nexercise}
\end{exercisewrap}
\footnotetext{%
The appropriate hypotheses are:

$H_0$: the proportion who get the answer correct is the same
as random guessing: 1-in-3, or $p = 0.333$.

$H_A$: the proportion who get the answer correct is different
than random guessing, $p \neq 0.333$.

The correct answer to the question is 2~billion.
While the world population is projected to increase,
the average age is also expected to rise.
That is, the majority of the population growth will
happen in older age groups, meaning people are projected
to live longer in the future across much of the world.}

% n <- 228; x <- 39; p <- x / n; n; p; 1 - p; x; n - x; sqrt(p*(1-p)/n)
\newcommand{\roslingBsize}{228}
\newcommand{\roslingBprop}{0.149}
\newcommand{\roslingBpropcomplement}{0.851}
\newcommand{\roslingBpercent}{14.9\%}
\newcommand{\roslingBpercentcomplement}{85.1\%}
\newcommand{\roslingBcount}{34}
\newcommand{\roslingBcountcomplement}{194}
\newcommand{\roslingBse}{0.024}
% n <- 228; x <- 34; (p <- x/n); (se <- sqrt(p * (1 - p) / n)); p + c(-1, 1) * 1.96 * se

\begin{exercisewrap}
\begin{nexercise}\label{roslingB_normality}%
This time we took a larger sample of
\roslingBsize{} college-educated adults,
\roslingBcount{} (\roslingBpercent{}) selected the correct
answer to the question in Guided
Practice~\ref{roslingB_hypothesis_setup}: 2~billion.
Can we model the sample proportion using a normal distribution
and construct a confidence interval?\footnotemark{}
\end{nexercise}
\end{exercisewrap}
\footnotetext{We check both conditions, which are satisfied,
so it is reasonable to use
a normal distribution for $\hat{p}$: \\
\textbf{Independence.} Since the data are from a simple
    random sample, the observations are independent. \\
\textbf{Success-failure.} We'll use $\hat{p}$ in place of $p$
    to check: $n\hat{p} = \roslingBcount$
    and $n(1 - \hat{p}) = \roslingBcountcomplement$.
    Both are greater than 10, so the success-failure condition
    is satisfied.}

\begin{examplewrap}
\begin{nexample}{Compute a 95\% confidence interval for the
    fraction of college-educated adults who answered the
    children-in-2100 question correctly, and evaluate the
    hypotheses in Guided
    Practice~\ref{roslingB_hypothesis_setup}.}
  To compute the standard error, we'll again use
  $\hat{p}$
  in place of $p$ for the calculation:
  \begin{align*}
  SE_{\hat{p}}
      = \sqrt{\frac{\hat{p}(1 - \hat{p})}{n}}
      = \sqrt{\frac{\roslingBprop{}(1 - \roslingBprop{})}
          {\roslingBsize{}}}
      = \roslingBse{}
  \end{align*}
  In Guided Practice~\ref{roslingB_normality},
  we found that $\hat{p}$ can be modeled using
  a normal distribution,
  which ensures a 95\% confidence interval may be accurately
  constructed as
  \begin{align*}
  \hat{p}~\pm~z^{\star} \times SE
  \quad\to\quad
  \roslingBprop{}~\pm~1.96 \times \roslingBse{}
  \quad\to\quad
  (0.103, 0.195)
  \end{align*}
  Because the null value, $p_0 = 0.333$, is not in the
  confidence interval, a population proportion of 0.333
  is implausible and we reject the null hypothesis.
  That is, the data provide statistically significant
  evidence that the actual proportion of college adults
  who get the children-in-2100 question correct is
  different from random guessing. Because the entire
  95\% confidence interval
  is below 0.333, we can conclude college-educated adults
  do \emph{worse} than random guessing on this question.

  One subtle consideration is that we used a
  95\% confidence interval.
  What if we had used a 99\% confidence level?
  Or even a 99.9\% confidence level?
  It's possible to come to a different conclusion
  if using a different confidence level.
  Therefore, when we make a conclusion based
  on confidence interval, we should also be sure
  it is clear what confidence level we used.
\end{nexample}
\end{examplewrap}

The worse-than-random performance on this
last question is not a fluke:
there are many such world health questions where people
do worse than random guessing.
In general, the answers suggest that people tend to be
more pessimistic about progress than reality suggests.
This topic is discussed in much greater detail in
the Roslings' book,
\emph{\oiRedirect{amazon_factfulness}{Factfulness}}.


\D{\newpage}

\subsection{Decision errors}

\index{hypothesis testing!decision errors|(}

Hypothesis tests are not flawless: we can make an incorrect
decision in a statistical hypothesis test based on the data.
For example, in the court system innocent people are
sometimes wrongly convicted and the guilty sometimes walk free.
%Unfortunately, we never truly know if $H_0$ or $H_A$ holds true.
One key distinction with statistical hypothesis tests is that
we have the tools necessary to probabilistically quantify how
often we make errors in our conclusions.

Recall that there are two competing hypotheses:
the null and the alternative.
In a hypothesis test, we make a statement about which one might
be true, but we might choose incorrectly. There are four possible
scenarios, which are summarized in Figure~\ref{fourHTScenarios}.

\begin{figure}[ht]
\centering
\begin{tabular}{l l c c}
& & \multicolumn{2}{c}{\textbf{Test conclusion}} \\
\cline{3-4}
\vspace{-3.7mm} \\
& & do not reject $H_0$ &  reject $H_0$ in favor of $H_A$ \\
\cline{2-4}
\vspace{-3.7mm} \\
& $H_0$ true &
    okay &  \highlight{Type~1 Error} \\
\raisebox{1.5ex}{\textbf{Truth}} & $H_A$ true &
    \highlight{Type~2 Error} & okay \\
\cline{2-4}
\end{tabular}
\caption{Four different scenarios for hypothesis tests.}
\label{fourHTScenarios}
\end{figure}

A \term{Type~1 Error} is rejecting the null hypothesis when
$H_0$ is actually true.
A \term{Type~2 Error} is failing to
reject the null hypothesis when the alternative is actually
true.

\begin{exercisewrap}
\begin{nexercise} \label{whatAreTheErrorTypesInUSCourts}
In a US court, the defendant is either innocent ($H_0$) or
guilty ($H_A$).
What does a Type~1 Error represent in this context?
What does a Type~2 Error represent?
Figure~\ref{fourHTScenarios} may be useful.\footnotemark{}
\end{nexercise}
\end{exercisewrap}
\footnotetext{If
  the court makes a Type~1 Error, this means the defendant
  is innocent ($H_0$ true) but wrongly convicted.
  Note that a Type~1 Error is only possible if we've rejected
  the null hypothesis.

  A Type~2 Error means the court failed to reject $H_0$
  (i.e. failed to convict the person) when she was
  in fact guilty ($H_A$ true).
  Note that a Type~2 Error is only possible if we have
  failed to reject the null hypothesis.}

\begin{examplewrap}
\begin{nexample}{How could we reduce the Type~1 Error rate
    in US courts?
    What influence would this have on the Type~2 Error rate?}
    \label{howToReduceType1ErrorsInUSCourts}%
  To lower the Type~1 Error rate, we might
  raise our standard for conviction from
  ``beyond a reasonable doubt'' to
  ``beyond a conceivable doubt'' so fewer people would
  be wrongly convicted. However, this would also make
  it more difficult to convict the people who are
  actually guilty, so we would make more Type~2 Errors.
\end{nexample}
\end{examplewrap}

\begin{exercisewrap}
\begin{nexercise} \label{howToReduceType2ErrorsInUSCourts}
How could we reduce the Type~2 Error rate in US courts?
What influence would this have on the Type~1 Error
rate?\footnotemark
\end{nexercise}
\end{exercisewrap}
\footnotetext{To lower the Type~2 Error rate, we want
  to convict more guilty people. We could lower the
  standards for conviction from ``beyond a reasonable
  doubt'' to ``beyond a little doubt''. Lowering the bar
  for guilt will also result in more wrongful convictions,
  raising the Type~1 Error rate.}

\index{hypothesis testing!decision errors|)}

Exercises~\ref{whatAreTheErrorTypesInUSCourts}-\ref{howToReduceType2ErrorsInUSCourts} provide
an important lesson: if we reduce how often we make
 one type of error, we generally make more of the
 other type.

Hypothesis testing is built around rejecting or failing
to reject the null hypothesis.
That is, we do not reject $H_0$ unless we have strong evidence.
But what precisely does \emph{strong evidence} mean?
As a general rule of thumb, for those cases where the null
hypothesis is actually true, we do not want to incorrectly
reject $H_0$ more than 5\% of the time.
This corresponds to a \term{significance level}%
\index{hypothesis testing!significance level} of 0.05.
That is, if the null hypothesis is true,
the significance level indicates how often
the data lead us to incorrectly reject $H_0$.
We often write the significance level using $\alpha$
(the Greek letter \emph{alpha}\index{Greek!alpha@alpha ($\alpha$)}):
$\alpha = 0.05$.
We discuss the appropriateness of different significance
levels in Section~\ref{significanceLevel}.

\D{\newpage}

If we use a 95\% confidence interval to evaluate a
hypothesis test and the null hypothesis happens to be true,
we will make an error whenever the point estimate is
at least 1.96 standard errors away from the population
parameter.
This happens about 5\% of the time (2.5\% in each tail).
Similarly, using a 99\% confidence interval to evaluate
a hypothesis is equivalent to a significance level of
$\alpha = 0.01$.

A confidence interval is very helpful in determining
whether or not to reject the null hypothesis.
However, the confidence interval approach isn't always
sustainable.
In several sections, we will encounter situations where
a confidence interval cannot be constructed.
For example, if we wanted to evaluate the hypothesis
that several proportions are equal, it isn't clear how
to construct and compare many confidence intervals
altogether.

Next we will introduce a statistic called the \emph{p-value}
to help us expand our statistical toolkit, which will
enable us to both better understand the strength of
evidence and work in more complex data scenarios in
later sections.



\subsection{Formal testing using p-values}

\label{pValue}

\index{hypothesis testing!p-value|(}

The p-value is a way of quantifying the strength of the
evidence against the null hypothesis and in favor of the
alternative hypothesis.
Statistical hypothesis testing typically uses the
p-value method rather than making a decision based
on  confidence intervals.

\begin{onebox}{p-value}
  The \term{p-value}\index{hypothesis testing!p-value|textbf}
  is the probability of observing data at least as favorable
  to the alternative hypothesis as our current data set,
  if the null hypothesis were true. We typically use a summary
  statistic of the data, in this section the sample proportion,
  to help compute the p-value and evaluate the hypotheses.
\end{onebox}

%To apply the normal distribution framework in the context of a hypothesis test for a proportion, the independence and success-failure conditions must be satisfied. In a hypothesis test, the success-failure condition is checked using the null proportion: we verify $np_0$ and $n(1-p_0)$ are at least 10, where $p_0$ is the null value.

\index{data!coal power support|(}

\newcommand{\pewcoalpollsize}{1000}
\newcommand{\pewcoalpollprop}{0.37}
\newcommand{\pewcoalpollpropcomplement}{0.63}
\newcommand{\pewcoalpollpercent}{37\%}
\newcommand{\pewcoalpollpercentcomplement}{63\%}
\newcommand{\pewcoalpollcount}{370}
\newcommand{\pewcoalpollcountcomplement}{630}
\newcommand{\pewcoalpollse}{0.0153}
\newcommand{\pewcoalpollnullvalue}{0.5}
\newcommand{\pewcoalpollnullse}{0.016}

\begin{examplewrap}
\begin{nexample}{Pew Research asked a random sample of
    \pewcoalpollsize{} American
    adults whether they supported the increased usage of coal to
    produce energy.
    Set up hypotheses to evaluate whether
    a majority of American adults support or oppose
    the increased usage of coal.}
  The uninteresting result is that there is no majority either way:
  half of Americans support and the other half oppose expanding the
  use of coal to produce energy. The alternative hypothesis would
  be that there is a majority support or oppose
  (though we do not known which one!) expanding the
  use of coal. If $p$ represents the proportion supporting, then
  we can write the hypotheses as
  \begin{description}
    \item[$H_0$:] $p = 0.5$
    \item[$H_A$:] $p \neq 0.5$
  \end{description}
  In this case, the null value is $p_0 = 0.5$.
\end{nexample}
\end{examplewrap}

%\begin{examplewrap}
%\begin{nexample}{Suppose the null value, $p_0 = 0.5$,
%    was the actual level of support for coal usage.
%    Describe how we could simulate a survey of
%    \pewcoalpollsize{} responses when $p_0 = 0.5$.}
%  \label{simOnePropExample}%
%  If we pick a random person to participate in the survey,
%  then \emph{under the null hypothesis},
%  the chances they would support coal usage is $p_0 = 0.5$.
%  If this were true, then it's the same as flipping a fair coin.
%  That is, we can simulate an individual person's response by
%  flipping a coin;
%  if it's heads, we say \resp{support},
%  and if it's tails, \resp{oppose}.
%  To simulate \pewcoalpollsize{} independent responses,
%  we can flip the coin a total of 1000 times and compute the
%  fraction of instances that were heads as the observed
%  proportion.
%  We did this and observed 487 heads for a proportion
%  of $\hat{p}_{\text{sim, 1}} = 0.487$.
%\end{nexample}
%\end{examplewrap}
%
%Example~\ref{simOnePropExample} described how we could
%simulate a survey result under the null hypothesis that
%the population proportion is equal to $p_0$.
%In this way, we check what kind of sample observations
%we might expect to see \emph{if the null hypothesis were true}.
%Of course, a single simulation is interesting, but not that
%informative.
%If we run the simulation again, we get a value of
%$\hat{p}_{\text{sim, 2}} = 0.502$.
%And again: $\hat{p}_{\text{sim, 3}} = 0.523$.
%We can do this many times on a computer,
%just like we did for a population proportion of 0.88
%in Section~\ref{pointEstimates}.
%The results of 5,000 simulated surveys are summarized
%in a histogram in Figure~\ref{sampling_5k_prop_50p}.
%
%\begin{figure}[h]
%  \centering
%  \Figure{0.8}{sampling_5k_prop_50p}
%  \caption{
%      Simulated surveys proportion
%      \emph{if} the population proportion
%      were equal to the null value, $p_0 = 0.5$.
%      All 5,000 simulated sample proportions
%      lie between 0.44 and 0.56.}
%  \label{sampling_5k_prop_50p}
%\end{figure}
%
%\begin{examplewrap}
%\begin{nexample}{The actual Pew Research survey found that
%    \pewcoalpollpercent{} of the \pewcoalpollsize{}
%    respondents supported increasing the use of coal.
%    Use Figure~\ref{sampling_5k_prop_50p}
%    to estimate how frequently we might observe a proportion
%    of \pewcoalpollprop{} if the null hypothesis that
%    the population proportion is 0.5 were actually true.
%    What might you conclude from this finding?}
%  Not one of the 5,000 simulations yielded a sample proportion
%  of \pewcoalpollpercent{} or further from 0.5.
%  That is, \emph{if} the actually population proportion is
%  actually 0.5, then we observed something so rare that we
%  wouldn't necessarily see it if we repeated the process
%  5,000 times.
%  Ultimately, the observed sample result is nearly
%  impossible (extremely improbable!) if we believe that
%  the population proportion is 0.5.
%  This evidence casts significant doubt on the notion that
%  $p = 0.5$, and we should reject the null hypothesis,~$H_0$.
%\end{nexample}
%\end{examplewrap}

When evaluating hypotheses for proportions using the
p-value method,
we will slightly modify how we check the success-failure
condition and compute the standard error for the
single proportion case.
These changes aren't dramatic, but pay close attention
to how we use the null value, $p_0$.

\begin{examplewrap}
\begin{nexample}{Pew Research's sample show that
    \pewcoalpollpercent{}
    of American adults support increased usage of coal.
    We now wonder, does \pewcoalpollpercent{} represent
    a real difference from the null hypothesis of 50\%?
    What would the sampling distribution of $\hat{p}$
    look like if the null hypothesis were true?}
  If the null hypothesis were true, the population proportion
  would be the null value, 0.5.
  We~previously learned that
  the sampling distribution of $\hat{p}$ will be normal when
  two conditions are~met:
  \begin{description}
    \item[Independence.]
        The poll was based on a simple random sample,
        so independence is satisfied.
    \item[Success-failure.]
        Based on the poll's sample size of
        $n = \pewcoalpollsize{}$,
        the success-failure condition is met, since
        \begin{align*}
        np ~ \stackrel{H_0}{=}
            ~ \pewcoalpollsize{} \times \pewcoalpollnullvalue{}
            = 500
        \qquad\qquad
        n (1 - p) ~ \stackrel{H_0}{=}
            ~ \pewcoalpollsize{} \times
                (1 - \pewcoalpollnullvalue{})
            = 500
        \end{align*}
        are both at least 10.
        Note that the success-failure condition was checked
        using the null value, $p_0 = 0.5$;
        this is the first procedural difference from
        confidence intervals.
  \end{description}
  If the null hypothesis were true, the sampling distribution
  indicates that a sample proportion based on
  $n = \pewcoalpollsize{}$ observations
  would be normally distributed. Next, we can compute the standard
  error, where we will again use the null value $p_0 = 0.5$ in the
  calculation:
  \begin{align*}
  SE_{\hat{p}}
    = \sqrt{\frac{p (1 - p)}{n}}
    \quad \stackrel{H_0}{=} \quad
        \sqrt{\frac{\pewcoalpollnullvalue{} \times
            (1 - \pewcoalpollnullvalue{})}{\pewcoalpollsize{}}}
    = \pewcoalpollnullse{}
  \end{align*}
  This marks the other procedural difference from confidence
  intervals: since the sampling distribution is determined
  under the null proportion, the null value $p_0$ was used for
  the proportion in the calculation rather than $\hat{p}$.

  Ultimately, if the null hypothesis were true, then the sample
  proportion should follow a normal distribution with mean
  \pewcoalpollnullvalue{}
  and a standard error of \pewcoalpollnullse{}.
  This distribution is shown in
  Figure~\ref{normal_dist_mean_500_se_016}.
\end{nexample}
\end{examplewrap}

\begin{figure}[h]
  \centering
  \Figure[A normal distribution centered at 0.5 with a standard deviation of 0.016 is shown. Additionally, an annotation is located at 0.37 that states, "Observed p-hat equals 0.37".]{0.7}{normal_dist_mean_500_se_016}
  \caption{
      If the null hypothesis were true,
      this normal distribution describes the
      distribution of $\hat{p}$.}
  \label{normal_dist_mean_500_se_016}
\end{figure}

\begin{onebox}{Checking success-failure and computing
      $\mathbf{SE_{\hat{p}}}$
      for a hypothesis test}
  When using the p-value method to evaluate a hypothesis test,
  we check the conditions for $\hat{p}$ and construct the
  standard error using the null value, $p_0$, instead of using
  the sample proportion. \stdvspace{}

  In a hypothesis test with a p-value, we are supposing the
  null hypothesis is true,
  which is a different mindset than when we compute
  a confidence interval.
  This is why we use $p_0$ instead of $\hat{p}$
  when we check conditions and compute the standard error
  in this context.
\end{onebox}

When we identify the sampling distribution under the null hypothesis,
it has a special name: the \term{null distribution}.
The p-value represents the probability of the observed $\hat{p}$,
or a $\hat{p}$ that is more extreme,
if the null hypothesis were true.
To find the p-value, we generally find the null distribution,
and then we find a tail area in that distribution corresponding
to our point estimate.
%In some cases, as in this particular instance,
%the null distribution is a normal distribution.

\begin{examplewrap}
\begin{nexample}{If the null hypothesis were true,
    determine the chance of finding $\hat{p}$ at least
    as far into the tails as \pewcoalpollprop{}
    under the null distribution,
    which is a normal distribution with mean
    $\mu = \pewcoalpollnullvalue{}$
    and $SE = \pewcoalpollnullse{}$.}
%  When we compute the p-value, we think about the chance
%  of our observation, if the null hypothesis were true.
%
  This is a normal probability problem where
  $x = \pewcoalpollprop{}$.
  First, we draw a simple graph to represent the situation,
  similar to what is shown in
  Figure~\ref{normal_dist_mean_500_se_016}.
  Since $\hat{p}$ is so far out in the tail, we know the
  tail area is going to be very small. To find it, we start
  by computing the Z-score using the mean of 0.5 and the
  standard error of \pewcoalpollnullse{}:
  \begin{align*}
  Z = \frac{\pewcoalpollprop{} - 0.5}{\pewcoalpollnullse{}} = -8.125 
  \end{align*}
  We can use software to find the tail area:
  $2.2 \times 10^{-16}$
  (0.00000000000000022).
  If using the normal probability table in
  Appendix~\ref{normalProbabilityTable},
  we'd find that $Z = -8.125$ is off the table,
  so we would use the smallest area listed: 0.0002.

  The potential $\hat{p}$'s in the upper tail beyond
  \pewcoalpollpropcomplement{}, which are shown
  in Figure~\ref{normal_dist_mean_500_se_016_with_upper},
  also represent observations at least as extreme as
  the observed value of \pewcoalpollprop{}.
  To account for these values that are also more
  extreme under the hypothesis setup,
  we double the lower tail to get an estimate
  of the p-value: $4.4 \times 10^{-16}$
  (or if using the table method, 0.0004).

  The p-value represents the probability of observing
  such an extreme sample proportion by chance, if the null
  hypothesis were true.
\end{nexample}
\end{examplewrap}

\begin{figure}[h]
  \centering
  \Figures[A normal distribution centered at 0.5 with a standard deviation of 0.016 is shown. Additionally, the tail areas below 0.37 and above 0.63 are emphasized -- the regions under the normal distribution are nearly zero. Two annotations also appear. First, an annotation located at 0.37 states, "Tail area for p-hat". Second, an annotation located at 0.68 states, "Equally unlikely if H-sub-zero (the null hypothesis) is true".]{0.7}{normal_dist_mean_500_se_016}
      {normal_dist_mean_500_se_016_with_upper}
  \caption{
      If $H_0$ were true, then the values above
      \pewcoalpollpropcomplement{} are just
      as unlikely as values below \pewcoalpollprop{}.}
  \label{normal_dist_mean_500_se_016_with_upper}
\end{figure}




\begin{examplewrap}
\begin{nexample}{How should we evaluate the hypotheses using the
    p-value of $4.4 \times 10^{-16}$?
    Use the standard significance level of $\alpha = 0.05$.}
  If the null hypothesis were true, there's only an incredibly
  small chance of observing such an extreme deviation of
  $\hat{p}$ from 0.5.
  This means one of the following must be true:
  \begin{enumerate}
    \item The null hypothesis is true, and we just happened
        to observe something so extreme that it only happens
        about once in every 23 quadrillion times
        (1~quadrillion = 1~million $\times$ 1~billion).
    \item The alternative hypothesis is true,
        which would be consistent
        with observing a sample proportion far from 0.5.
  \end{enumerate}
  The first scenario is laughably improbable,
  while the second scenario seems much more plausible.

  Formally, when we evaluate a hypothesis test,
  we compare the p-value to the significance level,
  which in this case is $\alpha = 0.05$.
  Since the p-value is less than $\alpha$,
  we reject the null hypothesis.
  That is, the data provide strong evidence against $H_0$.
  The data indicate the direction of the difference:
  a majority of Americans do not support
  expanding the use of coal-powered energy.
\end{nexample}
\end{examplewrap}

\index{data!coal power support|)}

\begin{onebox}{Compare the p-value to $\pmb{\alpha}$ to
      evaluate $\pmb{H_0}$}
  When the p-value is less than the significance level, $\alpha$,
  reject $H_0$. We would report a conclusion that the data provide
  strong evidence supporting the alternative hypothesis. \\[2mm]
  When the p-value is greater than $\alpha$, do not reject $H_0$,
  and report that we do not have sufficient evidence to reject the
  null hypothesis. \\[2mm]
  In either case, it is important to describe the conclusion
  in the context of the data.
\end{onebox}







\index{data!nuclear arms reduction|(}

\begin{exercisewrap}
\begin{nexercise}
Do a majority of Americans support or oppose nuclear arms
reduction? Set up hypotheses to evaluate this
question.\footnotemark
\end{nexercise}
\end{exercisewrap}
\footnotetext{We would like to understand if a majority
  supports or opposes, or ultimately, if there is no difference.
  If $p$ is the proportion of Americans who support nuclear
  arms reduction, then
  $H_0$: $p = 0.50$ and $H_A$: $p \neq 0.50$.}

\newcommand{\gallupnucleararmspollsize}{1028}
\newcommand{\gallupnucleararmspollprop}{0.56}
\newcommand{\gallupnucleararmspollpropcomplement}{0.44}
\newcommand{\gallupnucleararmspollpercent}{56}
\newcommand{\gallupnucleararmspollpercentcomplement}{44}
\newcommand{\gallupnucleararmspollnullcount}{514}
\newcommand{\gallupnucleararmspollse}{0.0155}
\newcommand{\gallupnucleararmspollnullvalue}{0.5}
\newcommand{\gallupnucleararmspollnullse}{0.0156}

\begin{examplewrap}
\begin{nexample}{A simple random sample of
    \gallupnucleararmspollsize{} US adults
    in March 2013 show that
    \gallupnucleararmspollpercent{}\% support nuclear arms
    reduction.
    Does this provide convincing evidence that a majority
    of Americans supported nuclear arms reduction at the
    5\% significance level?} \label{NuclearArmsInferenceExample}
  First, check conditions:
  \begin{description}
  \item[Independence.] The poll was of a simple random sample
      of US adults, meaning the observations are independent.
  \item[Success-failure.] In a one-proportion hypothesis test,
      this condition is checked using the null proportion,
      which is $p_0 = \gallupnucleararmspollnullvalue{}$
      in this context:
      $n p_0 = n (1 - p_0)
          = \gallupnucleararmspollsize{} \times
              \gallupnucleararmspollnullvalue{}
          = \gallupnucleararmspollnullcount{} \geq 10$.
  \end{description}
  With these conditions verified,
  we can model $\hat{p}$ using a normal model.

  Next the standard error can be computed.
  The null value $p_0$ is used again here,
  because this is a hypothesis test for a single proportion.
  \begin{align*}
  SE_{\hat{p}}
      = \sqrt{\frac{p_0 (1 - p_0)}{n}}
      = \sqrt{\frac{\gallupnucleararmspollnullvalue{}
          (1 - \gallupnucleararmspollnullvalue{})}
          {\gallupnucleararmspollsize{}}}
      = \gallupnucleararmspollnullse{}
  \end{align*}
  Based on the normal model, the test statistic can be
  computed as the Z-score of the point estimate:
  \begin{align*}
  Z = \frac{\text{point estimate} - \text{null value}}{SE}
      = \frac{\gallupnucleararmspollprop{} - 0.50}
          {\gallupnucleararmspollnullse{}}
      = 3.85
  \end{align*}
  It's generally helpful to draw null distribution and
  the tail areas of interest for computing the p-value:
  \begin{center}
  \Figures[A normal distribution centered at 0.5 is shown, which has a standard deviation of about 0.0156. Two tails several standard deviations away from the center are emphasized. The first, at and above 0.56, is annotated with the text "upper tail". The second, which appears to be at and below 0.44, is annotated with the text "lower tail".]{0.48}{nuclearArmsReduction}{nuclearArmsReductionPValue}
  \end{center}
  The upper tail area is about 0.0001,
  and we double this tail area to get the p-value: 0.0002.
  Because the p-value is smaller than 0.05, we reject $H_0$.
  The poll provides convincing evidence that a majority
  of Americans supported nuclear arms reduction efforts
  in March 2013.
\end{nexample}
\end{examplewrap}

\index{data!nuclear arms reduction|)}

\D{\newpage}

\newcommand{\oneprophtsummary}{
\begin{onebox}{Hypothesis testing for a single proportion}
  Once you've determined a one-proportion hypothesis test is the
  correct procedure, there are four steps to completing the
  test:
  \begin{description}
  \item[Prepare.]
      Identify the parameter of interest,
      list hypotheses,
      identify the significance level,
      and identify $\hat{p}$ and $n$.
  \item[Check.]
      Verify conditions
      to ensure $\hat{p}$ is nearly normal under $H_0$.
      For one-proportion hypothesis tests, use the null
      value to check the success-failure condition.
  \item[Calculate.]
      If the conditions hold, compute the standard
      error, again using $p_0$, compute the Z-score,
      and identify the p-value.
  \item[Conclude.]
      Evaluate the hypothesis test by comparing the p-value
      to $\alpha$, and provide a conclusion in the context
      of the problem.
  \end{description}
\end{onebox}
}
\oneprophtsummary{}

\CalculatorVideos{hypothesis tests for a single proportion}


\subsection{Choosing a significance level}
\label{significanceLevel}

\index{hypothesis testing!significance level|(}
\index{significance level|(}

Choosing a significance level for a test is important in
many contexts, and the traditional level is $\alpha = 0.05$.
However, it can be helpful to adjust the significance level
based on the application. We may select a level that is
smaller or larger than 0.05 depending on the consequences
of any conclusions reached from the test.

If making a Type~1 Error is dangerous or especially costly,
we should choose a small significance level (e.g. 0.01).
Under this scenario we want to be very cautious about
rejecting the null hypothesis, so we demand very strong
evidence favoring $H_A$ before we would reject $H_0$.

If a Type~2 Error is relatively more dangerous or much more
costly than a Type~1 Error, then we might choose a higher
significance level (e.g. 0.10). Here we want to be cautious
about failing to reject $H_0$ when the alternative hypothesis
is actually true.

Additionally, if the cost of collecting data is small relative
to the cost of a Type~2 Error, then it may also be a good
strategy to collect more data.
Under this strategy, the Type~2 Error can be reduced
while not affecting the Type~1 Error rate.
Of course, collecting extra data is often costly,
so~there is typically a cost-benefit analysis to be considered.
%We'll discuss this topic a bit more in
%Sections~\ref{} and~\ref{}.
%\Comment{Fix this reference.}

\newcommand{\doorhingeflawrate}{0.2}

\begin{examplewrap}
\begin{nexample}{A car manufacturer is considering switching
    to a new, higher quality piece of equipment that constructs
    vehicle door hinges.
    They figure that they will save money in the long run
    if this new machine produces hinges
    that have flaws less than
    \doorhingeflawrate{}\% of the time.
    However, if the hinges are flawed more than
    \doorhingeflawrate{}\% of
    the time, they wouldn't get a good enough
    return-on-investment from the new piece of equipment,
    and they would lose money.
    Is there good reason to modify the significance level
    in such a hypothesis test?}
  The null hypothesis would be that the rate of flawed
  hinges is \doorhingeflawrate{}\%,
  while the alternative is that it the rate
  is different than \doorhingeflawrate{}\%.
  This decision is just one of many that have a marginal
  impact on the car and company.
  A significance level of 0.05 seems reasonable since
  neither a Type~1 or Type~2 Error should be dangerous
  or (relatively) much more expensive.
\end{nexample}
\end{examplewrap}

\begin{examplewrap}
\begin{nexample}{The same car manufacturer is considering
    a slightly more expensive supplier for parts related
    to safety, not door hinges.
    If the durability of these
    safety components is shown to be better than the
    current supplier, they will switch manufacturers.
    Is there good reason to modify the significance level
    in such an evaluation?}
  The null hypothesis would be that the suppliers' parts
  are equally reliable. Because safety is involved,
  the car company should be eager to switch to the slightly
  more expensive manufacturer (reject $H_0$), even if the
  evidence of increased safety is only moderately strong.
  A slightly larger significance level,
  such as $\alpha=0.10$, might be appropriate.
\end{nexample}
\end{examplewrap}

\begin{exercisewrap}
\begin{nexercise}
A part inside of a machine is very expensive to replace.
However, the machine usually functions properly even if
this part is broken, so the part is replaced only if we
are extremely certain it is broken based on a series of
measurements.
Identify appropriate hypotheses for this test
(in plain language) and suggest an appropriate significance
level.\footnotemark
\end{nexercise}
\end{exercisewrap}
\footnotetext{Here
  the null hypothesis is that the part is not broken,
  and the alternative is that it is broken.
  If we don't have sufficient evidence to reject $H_0$,
  we would not replace the part.
  It sounds like failing to fix the part if it is broken
  ($H_0$ false, $H_A$ true) is not very problematic,
  and replacing the part is expensive.
  Thus, we should require very strong evidence against
  $H_0$ before we replace the part.
  Choose a small significance level, such as $\alpha=0.01$.}

\begin{onebox}{Why is 0.05 the default?}
  The $\alpha = 0.05$ threshold is most common. But why?
  Maybe the standard level should be smaller, or perhaps larger.
  If you're a little puzzled, you're reading with an
  extra critical eye -- good job!
  We've made a 5-minute task to help clarify \emph{why 0.05}:
  \begin{center}
  \oiRedirect{textbook-why05}{www.openintro.org/why05}
  \end{center}
\end{onebox}


\index{significance level|)}
\index{hypothesis testing!significance level|)}
\index{hypothesis testing|)}


\subsection{Statistical significance versus practical significance}

When the sample size becomes larger,
point estimates become more precise and any real differences
in the mean and null value become easier to detect and recognize.
Even a very small difference would likely be detected if we took
a large enough sample.
Sometimes researchers will take such large samples that even
the slightest difference is detected, even differences where
there is no practical value.
In such cases, we still say the difference is
\term{statistically significant},
but it is not \term{practically significant}.
For example, an online experiment might identify that placing
additional ads on a movie review website statistically
significantly increases viewership of a TV show by 0.001\%,
but this increase might not have any practical value.

%Statistically significant differences are sometimes
%so minor that they are not practically relevant.
%This is especially important to research:
%if we conduct a study, we want to focus on finding
%a meaningful result.
%We don't want to spend lots of money finding results
%that hold no practical value.

One role of a data scientist in conducting a study often
includes planning the size of the study.
The data scientist might first consult experts or scientific
literature to learn what would be the smallest meaningful
difference from the null value.
She also would obtain other information,
such as a very rough estimate of the true proportion $p$,
so that she could roughly estimate the standard error.
From here, she can suggest a sample size that is sufficiently
large that, if there is a real difference that is meaningful,
we could detect it.
While larger sample sizes may still be used,
these calculations are especially helpful when considering
costs or potential risks, such as possible health impacts
to volunteers in a medical study.


\D{\newpage}

\subsection{One-sided hypothesis tests (special topic)}

So far we've only considered what are called \term{two-sided
hypothesis tests}, where we care about detecting whether $p$
is either above or below some null value $p_0$.
There is a second type of hypothesis test called a
\term{one-sided hypothesis test}.
For a one-sided hypothesis test,
the hypotheses take one of the following forms:
\begin{enumerate}
\item There's only value in detecting if the population
    parameter is \emph{less than} some value~$p_0$.
    In~this case, the alternative hypothesis is written
    as $p < p_0$ for some null value $p_0$.
\item There's only value in detecting if the population
    parameter is \emph{more than} some value~$p_0$:
    In~this case, the alternative hypothesis is written
    as $p > p_0$.
\end{enumerate}
While we adjust the form of the alternative hypothesis,
we continue to write the null hypothesis using an equals-sign
in the one-sided hypothesis test case.

In the entire hypothesis testing procedure,
there is only one difference in evaluating a one-sided
hypothesis test vs a two-sided hypothesis test:
how to compute the p-value.
In a one-sided hypothesis test, we compute the p-value as
the tail area in the \emph{direction of the alternative
hypothesis only}, meaning it is represented by a single
tail area. Herein lies the reason why one-sided tests
are sometimes interesting: if we don't have to double
the tail area to get the p-value, then the p-value is
smaller and the level of evidence required to identify
an interesting finding in the direction of the
alternative hypothesis goes down.
However, one-sided tests aren't all sunshine and rainbows:
the heavy price paid is that any interesting findings
in the opposite direction must be disregarded.

\begin{examplewrap}
\begin{nexample}{
    In Section~\ref{basicExampleOfStentsAndStrokes},
    we encountered an example where doctors were interested
    in determining whether stents would help people who had
    a high risk of stroke.
    The researchers believed the stents would help.
    Unfortunately, the data showed the opposite:
    patients who received stents actually did worse.
    Why was using a two-sided test so important in
    this context?}
    \label{basicExampleOfStentsAndStrokesOneSided}
  Before the study, researchers had reason to believe
  that stents would help patients since existing research
  suggested stents helped in patients with heart attacks.
  It would surely have been tempting to use a one-sided
  test in this situation, and had they done this,
  they would have limited their ability to identify
  potential harm to patients.
\end{nexample}
\end{examplewrap}

Example~\ref{basicExampleOfStentsAndStrokesOneSided}
highlights that using a one-sided hypothesis creates
a risk of overlooking data supporting the opposite
conclusion.
We could have made a similar error when reviewing
the Roslings' question data this section;
if we had a pre-conceived notion that
college-educated people wouldn't do worse than random
guessing and so used a one-sided test,
we would have missed the really interesting finding
that many people have incorrect knowledge about
global public health.
%Here are a few other situations where it has been,
%or would have been, very useful to have an open mind
%and consider the contrarian view:
%\begin{itemize}
%\item The 2008 financial crisis. There were warning signs,
%    but few people recognized them.
%    In fact, some financial firms essentially bought into
%    the notion that housing prices could only rise, not fall.
%\item 
%    
%\end{itemize}

When might a one-sided test be appropriate to use?
\emph{Very rarely.}
Should you ever find yourself considering using a
one-sided test, carefully answer the following question:
\begin{quote}{\em
  What would I, or others, conclude if the data happens
  to go clearly in the opposite direction than my
  alternative hypothesis?
}\end{quote}
If you or others would find any value in making
a conclusion about the data that goes in the opposite
direction of a one-sided test, then a two-sided hypothesis
test should actually be used.
These considerations can be subtle, so exercise caution.
We will only apply two-sided tests in the rest of
this book.

\begin{examplewrap}
\begin{nexample}{
    Why can't we simply run a one-sided
    test that goes in the direction of the data?}
  We've been building a careful framework that
  controls for the Type~1 Error, which is the
  significance level $\alpha$ in a hypothesis test.
  We'll use the $\alpha = 0.05$ below to keep
  things simple.

  Imagine we could pick the one-sided test after
  we saw the data. What will go wrong?
  \begin{itemize}
  \item If $\hat{p}$ is \emph{smaller} than
      the null value,
      then a one-sided test where $p < p_0$ would
      mean that any observation in the
      \emph{lower} 5\% tail of the null distribution
      would lead to us rejecting $H_0$.
  \item If $\hat{p}$ is \emph{larger} than
      the null value,
      then a one-sided test where $p > p_0$ would
      mean that any observation in the
      \emph{upper} 5\% tail of the null distribution
      would lead to us rejecting $H_0$.
  \end{itemize}
  Then if $H_0$ were true, there's a 10\% chance of
  being in one of the two tails, so our testing error
  is actually $\alpha = 0.10$, not 0.05.
  That is,
  not being careful about when to use one-sided tests
  effectively undermines the methods we're working
  so hard to develop and utilize.
\end{nexample}
\end{examplewrap}

\index{hypothesis testing|)}


{\exercisesheader{}

% 15

\eoce{\qt{Identify hypotheses, Part I\label{
}}
Write the null and alternative hypotheses in words and then symbols
for each of the following  situations.
\begin{parts}
\item
    A tutoring company would like to understand if most
    students tend to improve their grades (or not) after
    they use their services.
    They sample 200 of the students who used their service
    in the past year and ask them if their grades have
    improved or declined from the previous year.
\item
    Employers at a firm are worried about the effect of March Madness,
    a basketball championship held each spring in the US, on employee
    productivity.
    They estimate that on a regular business day employees spend on
    average 15 minutes of company time checking personal email,
    making personal phone calls, etc.
    They also collect data on how much company time employees spend
    on such non-business activities during March Madness.
    They want to determine if these data provide convincing evidence
    that employee productivity changed during March Madness.
\end{parts}
}{}

% 16

\eoce{\qt{Identify hypotheses, Part II\label{identify_hypotheses_prop_and_mean_2}} 
Write the null and alternative hypotheses in words and using symbols 
for each of the following situations.
\begin{parts}
\item
    Since 2008, chain restaurants in California have been required
    to display calorie counts of each menu item. Prior to menus
    displaying calorie counts, the average calorie intake of diners
    at a restaurant was 1100 calories.
    After calorie counts started to be displayed on menus,
    a nutritionist collected data on the number of calories consumed
    at this restaurant from a random sample of diners.
    Do these data provide convincing evidence of a difference in the
    average calorie intake of a diners at this restaurant?
\item
    The state of Wisconsin would like to understand
    the fraction of its adult residents that consumed alcohol
    in the last year,
    specifically if the rate is different from the
    national rate of 70\%.
    To help them answer this question, they conduct
    a random sample of 852 residents and ask them
    about their alcohol consumption.
\end{parts}
}{}

% 17

\eoce{\qt{Online communication\label{online_communication_prop_ht_errors}}
A study suggests that 60\% of college student spend
10~or more hours per week communicating with others online.
You believe that this is incorrect and decide to collect your 
own sample for a hypothesis test.
You randomly sample 160 students from your dorm
and find that 70\% spent 10~or more hours a week
communicating with others online.
A~friend of yours, who offers to help you with
the hypothesis test, comes up with the following
set of hypotheses.
Indicate any errors you see.
\begin{align*}
H_0&: \hat{p} < 0.6 \\
H_A&: \hat{p} > 0.7
\end{align*}
}{}

% 18

\eoce{\qt{Married at 25\label{married_at_25_prop_ht_errors}}
A study suggests that the 25\% of 25 year olds have
gotten married.
You believe that this is incorrect and decide to collect
your own sample for a hypothesis test.
From a random sample of 25 year olds in census data
with size 776,
you find that 24\% of them are married.
A friend of yours offers to help you with setting
up the hypothesis test and comes up with the following
hypotheses.
Indicate any errors you see.
\begin{align*}
H_0&: \hat{p} = 0.24 \\
H_A&: \hat{p} \neq 0.24
\end{align*}
}{}

% 19

\eoce{\qt{Cyberbullying rates\label{cyberbullying_prop_ci_ht}}
Teens were surveyed about cyberbullying, and
54\% to 64\% reported experiencing cyberbullying
(95\% confidence interval).\footfullcite{pew_cyber_bully_2018}
Answer the following questions based on this interval.
\begin{parts}
\item 
    A newspaper claims that a majority of teens
    have experienced cyberbullying.
    Is this claim supported by the confidence interval?
    Explain your reasoning.
\item\label{cyberbullying_prop_ci_ht_researcher}
    A researcher conjectured that 70\% of teens have
    experienced cyberbullying.
    Is this claim supported by the confidence interval?
    Explain your reasoning.
\item
    Without actually calculating the interval, determine
    if the claim of the researcher from
    part~(\ref{cyberbullying_prop_ci_ht_researcher})
    would be supported based on a 90\% confidence interval?
\end{parts}
}{}

\D{\newpage}

% 20

\eoce{\qt{Waiting at an ER, Part II\label{er_wait_ci_ht_prop_ok}}
Exercise~\ref{er_wait_intro_prop_ok} 
provides a 95\% confidence interval for the mean waiting
time at an emergency room (ER) of (128 minutes, 147 minutes).
Answer the following questions based on this interval.
\begin{parts}
\item
    A local newspaper claims that the average waiting time
    at this ER exceeds 3 hours.
    Is this claim supported by the confidence interval?
    Explain your reasoning.
\item\label{er_wait_ci_ht_prop_ok_dean}
    The Dean of Medicine at this hospital claims the
    average wait time is 2.2 hours.
    Is this claim supported by the confidence interval?
    Explain your reasoning.
\item
    Without actually calculating the interval,
    determine if the claim of the Dean from
    part~(\ref{er_wait_ci_ht_prop_ok_dean})
    would be supported based on a 99\% confidence interval?
\end{parts}
}{}

% 21

\eoce{\qt{Minimum wage, Part I\label{minimum_wage_prop_1}}
Do a majority of US adults believe raising
the minimum wage will help the economy,
or is there a majority who do not believe this?
A~Rasmussen Reports survey of a random sample of 1,000 US adults found
that 42\% believe it will help the
economy.\footfullcite{webpage:rasmussen-2019-raise-minimum-wage}
Conduct an appropriate hypothesis test to help
answer the research question.
}{}

% 22

\eoce{\qt{Getting enough sleep\label{univ_students_enough_sleep}}
400 students were randomly sampled from a large university,
and 289 said they did not get enough sleep.
Conduct a hypothesis test to check whether this
represents a statistically significant difference
from 50\%, and use a significance level of 0.01.
}{}

% 23

\eoce{\qt{Working backwards, Part I\label{backwards_prop_1}}
You are given the following hypotheses:
\begin{align*}
H_0&: p = 0.3 \\
H_A&: p \ne 0.3
\end{align*}
We know the sample size is 90.
For what sample proportion would the p-value be equal to 0.05?
Assume that all conditions  necessary for inference are satisfied.
}{}

% 24

\eoce{\qt{Working backwards, Part II\label{backwards_prop_2}}
You are given the following hypotheses:
\begin{align*}
H_0&: p = 0.9 \\
H_A&: p \ne 0.9
\end{align*}
We know that the sample size is 1,429.
For what sample proportion would the p-value be equal to 0.01?
Assume that all conditions necessary for inference are satisfied.
}{}

% 25

\eoce{\qt{Testing for Fibromyalgia\label{errors_fibromyalgia}} A patient named Diana 
was diagnosed with Fibromyalgia, a long-term syndrome of body pain, and was 
prescribed anti-depressants. Being the skeptic that she is, Diana didn't 
initially believe that anti-depressants would help her symptoms. However after 
a couple months of being on the medication she decides that the 
anti-depressants are working, because she feels like her symptoms are in fact 
getting better.
\begin{parts}
\item Write the hypotheses in words for Diana's skeptical position when she 
started taking the anti-depressants.
\item What is a Type~1 Error in this context?
\item What is a Type~2 Error in this context?
\end{parts}
}{}

% 26

\eoce{\qtq{Which is higher\label{prop_which_higher_found_inf}}
In each part below, there is a value of interest and two
scenarios (I and II).
For each part, report if the value of interest is larger
under scenario I, scenario II, or whether the value is
equal under the scenarios.
\begin{parts}
\item
     The standard error of $\hat{p}$ when
     (I)~$n = 125$ or (II)~$n = 500$.
\item
    The margin of error of a confidence interval
    when the confidence level is
    (I)~90\% or (II)~80\%.
\item
    The p-value for a Z-statistic of 2.5 calculated
    based on a (I)~sample with $n = 500$ or based on
    a (II)~sample with $n = 1000$.
\item
    The probability of making a Type~2 Error when the
    alternative hypothesis is true and the significance
    level is (I)~0.05 or (II)~0.10.
\end{parts}
}{}
}
