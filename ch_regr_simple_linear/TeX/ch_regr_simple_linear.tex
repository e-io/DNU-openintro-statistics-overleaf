\begin{chapterpage}{Introduction to linear regression}
  \chaptertitle{Introduction to linear \titlebreak{} regression}
  \label{linRegrForTwoVar}
  \label{ch_regr_simple_linear}
  \chaptersection{fitting_line_to_data_section}
  \chaptersection{fittingALineByLSR}
  \chaptersection{typesOfOutliersInLinearRegression}
  \chaptersection{inferenceForLinearRegression}
\end{chapterpage}
\renewcommand{\chapterfolder}{ch_regr_simple_linear}


\index{regression|textbf}
\index{regression|(}
\index{linear regression|seealso{regression}}

\chapterintro{Linear regression is a very powerful
  statistical technique.
  Many people have some familiarity with regression just from
  reading the news, where straight lines are overlaid
  on scatterplots.
  Linear models can be used for prediction or to
  evaluate whether there is a linear relationship
  between two numerical variables.}



%__________
\section{Fitting a line, residuals, and correlation}
% \section{Using a line to model data}
\label{fitting_line_to_data_section}

It's helpful to think deeply about the line fitting process.
In this section, we define the form of a linear model,
explore criteria for what makes a good fit,
and introduce a new statistic called
\emph{correlation}\index{correlation}.


\subsection{Fitting a line to data}

Figure~\ref{perfLinearModel} shows two variables whose
relationship can be modeled perfectly with a straight line.
The equation for the line is
\begin{eqnarray*}
y = 5 + 64.96 x
\end{eqnarray*}
Consider what a perfect linear relationship means:
we know the exact value of $y$ just by knowing
the value of $x$.
This is unrealistic in almost any natural process.
For example, if we took family income ($x$),
this value would provide some useful information about
how much financial support a college may offer a prospective
student~($y$).
However, the prediction would be far from perfect,
since other factors play a role in financial support
beyond a family's finances.

\begin{figure}[h]
   \centering
   \Figure[A scatterplot with a straight line fit to the data are shown for the date December 28th, 2018. The horizontal axis is "Number of Target Corporation Stocks to Purchase" and the vertical axis is "Total Cost of the Shares Purchase". Twelve data points are shown that all fall exactly on a straight line with an equation of y equals 5 plus 64.96 times x. Because the cost is computed using a linear formula, this explains why the linear fit is perfect.]{0.6}{perfLinearModel}
   \caption{Requests from twelve separate buyers were
       simultaneously placed with a trading company to purchase
       Target Corporation stock
       (ticker \texttt{TGT}, December 28th, 2018),
       and the total cost of the shares were reported.
       Because the cost is computed using a linear formula,
       the linear fit is perfect.}
   \label{perfLinearModel}
\end{figure}

Linear regression is the statistical method for fitting
a line to data where the relationship between two variables,
$x$ and $y$, can be modeled by a straight line with some error:
\begin{align*}
y = \beta_0 + \beta_1x + \varepsilon
\end{align*}
The values $\beta_0$ and $\beta_1$ represent the model's
parameters\index{parameter}
($\beta$ is the Greek letter
  \emph{beta}\index{Greek!beta@beta ($\beta$)}),
and the error is represented by $\varepsilon$
(the Greek letter \emph{epsilon}\index{Greek!epsilon@epsilon ($\varepsilon$)}).
The parameters are estimated using data,
and we write their point estimates as $b_0$ and $b_1$.
When we use $x$ to predict $y$,
we usually call $x$ the explanatory\index{explanatory variable}
or \term{predictor} variable,
and we call $y$ the response;
we also often drop the $\epsilon$ term when writing down the
model since our main focus is often on the prediction of
the average outcome.

It is rare for all of the data to fall perfectly on a straight line.
Instead, it's more common for data to appear as
a \emph{cloud of points}\index{cloud of points},
such as those examples shown in  Figure~\ref{imperfLinearModel}.
In each case, the data fall around a straight line,
even if none of the observations fall exactly on the line.
The first plot shows a relatively strong downward
linear trend,
where the remaining variability in the data around the
line is minor relative to the strength of the relationship
between $x$ and $y$.
The second plot shows an upward trend that,
while evident, is not as strong as the first.
The last plot shows a very weak downward trend in the data,
so slight we can hardly notice it.
In each of these examples,
we will have some uncertainty regarding our estimates
of the model parameters, $\beta_0$ and $\beta_1$.
For instance, we might wonder, should we move the line
up or down a little, or should we tilt it more or less?
As we move forward in this chapter,
we will learn about criteria for line-fitting,
and we will also learn about the uncertainty associated
with estimates of model parameters.

\begin{figure}
   \centering
   \Figure[Three scatterplots are shown. The first has data ranging from -50 to positive 50 on both the horizontal and vertical axes. The data start in the upper left corner of the plot and then move steadily down to the right corner. The second plot has the horizontal axis running from 500 to about 2,000 and the vertical axis from about 0 to 25,000. At the left side of the plot, the data are in the lower half of the plot, and the points generally are steadily higher as we move right, where most points near the right end of the plot are in the upper region of the plot. A upwards trending line has been fit to these points. The last plot runs from about -10 to positive 50 on the horizontal axis and about -200 to positive 400 on the vertical axis. The points are scattered broadly across the range, with only the slightest downward trend evident in the data. A trend line has been fit to this data, though it is nearly flat.]{}{imperfLinearModel}
   \caption{Three data sets where a linear model may be useful
       even though the data do not all fall exactly on the line.}
   \label{imperfLinearModel}
\end{figure}

There are also cases where fitting a straight line to the data,
even if there is a clear relationship between the variables,
is not helpful.
One such case is shown in
Figure~\ref{notGoodAtAllForALinearModel}
where there is a very clear relationship between the variables
even though the trend is not linear.
We discuss \index{nonlinear}nonlinear trends in this chapter
and the next, but details of fitting nonlinear models
are saved for a later course.

\begin{figure}
   \centering
   \Figure[A linear model is not useful in a nonlinear set of data shown in this plot. The data are from an introductory physics experiment, where a ball is shot at many angles of inclination between 0 degrees and 90 degrees (represented by the horizontal axis), and the measured horizontal distance traveled by the ball before it hits the ground is shown in meters. The first point, at an angle of inclination of 0 hits the ground at 0 meters traveled. As the angle is increased, the ball travels further before it hits the ground until reaching a peak at 45 degrees angle of inclination, at which point it decreases again until we reach an angle of 90 degrees, at which point the ball again does not travel any horizontal distance before it hits the ground. For the data shown, the best fitting straight line is shown and is flat. This is a good example of why a straight line fit to data where there is curvature is often not useful.]{0.8}{notGoodAtAllForALinearModel}
   \caption{A linear model is not useful in this nonlinear case.
       These data are from an introductory physics experiment.}
   \label{notGoodAtAllForALinearModel}
\end{figure}




\subsection{Using linear regression to predict possum head lengths}

\index{data!possum|(}

Brushtail possums are a marsupial that lives in Australia,
and a photo of one is shown in
Figure~\ref{brushtail_possum}.
Researchers captured 104 of these animals and took body
measurements before releasing the animals back into the wild.
We consider two of these measurements:
the total length of each possum, from head to tail,
and the length of each possum's head.

\setlength{\captionwidth}{0.83\mycaptionwidth}
\begin{figure}[h]
  \centering
  \Figure[A common brushtail possum of Australia is shown. It has a brown fur coat with some gray sprinkled in along with a face and ears that somewhat resemble a house cat. The possum also has a big bushy tail.]{0.5}{brushtail_possum}
  \caption{The common brushtail possum of Australia.\vspace{-1mm} \\
      -----------------------------\vspace{-2mm}\\
      {\footnotesize Photo by Greg Schechter
      (\oiRedirect{textbook-flickr_com_schechter_brushtail_possum_5653697137}
          {https://flic.kr/p/9BAFbR}).
      \oiRedirect{textbook-CC_BY_2}
          {CC~BY~2.0~license}.}}
  \label{brushtail_possum}
\end{figure}
\setlength{\captionwidth}{\mycaptionwidth}

%Scatterplots were introduced in Chapter~\ref{introductionToData}
%as a graphical technique to present two numerical variables
%simultaneously.
%Such plots permit the relationship between the variables
%to be examined with ease.
Figure~\ref{scattHeadLTotalL} shows a scatterplot for the head
length and total length of the possums.
Each point represents a single possum from the data.
The head and total length variables are associated:
possums with an above average total length also tend to have
above average head lengths.
While the relationship is not perfectly linear, it could
be helpful to partially explain the connection between these
variables with a straight line.

\D{\newpage}

\begin{figure}[h]
  \centering
  \Figure[A scatterplot showing head length against total length for 104 brushtail possums, where the horizontal axis for total length runs from 75 centimeters to about 97 centimeters (2.5 to 3.3 feet) and the vertical axis for head length runs from about 82 millimeters up to about 104 millimeters (3 to 4 inches). For possums with a total length between 75 to 80 centimeters, there are three points shown, each with head lengths of about 85 millimeters. For possums with total length from 80 to 85 centimeters, most head lengths range from about 85 millimeters to 95 millimeters. For possums with total lengths from 85 to 90 centimeters, head lengths mostly lie between 90 millimeters and 97 millimeters. For possums with total lengths larger than 90 centimeters, the head lengths are mostly between 93 millimeters and 100 millimeters. The trend is evidently upward and approximately linear. A point representing a possum with head length 94.1mm and total length 89cm is highlighted (although not relevant for any other purpose than giving an example or reminder for how a point is read in a scatterplot).]{0.75}{scattHeadLTotalL}
  \caption{A scatterplot showing head length against total length
      for 104 brushtail possums.
      A point representing a possum with head length 94.1mm
      and total length 89cm is highlighted.}
  \label{scattHeadLTotalL}
\end{figure}

%Straight lines should only be used when the data appear to have
%a linear relationship, such as the case shown in the left panel
%of Figure~\ref{scattHeadLTotalLTube}.
%The right panel of Figure~\ref{scattHeadLTotalLTube} shows
%a case where a curved line would be more useful in understanding
%the relationship between the two variables.

%\begin{figure}[h]
%  \centering
%  \Figure{0.95}{scattHeadLTotalLTube}
%  \caption{The figure on the left shows head length versus
%      total length, and reveals that many of the points could
%      be captured by a straight band.
%      On the right, we see that a curved band is more appropriate
%      in this scatterplot.}
%  \label{scattHeadLTotalLTube}
%\end{figure}

We want to describe the relationship between the head length and total length variables in the possum data set using a line. In this example, we will use the total length as the predictor variable, $x$, to predict a possum's head length, $y$. We could fit the linear relationship by eye, as in Figure~\ref{scattHeadLTotalLLine}. The equation for this line is
\begin{align*}
\hat{y} = 41 + 0.59x
\end{align*}
A ``hat'' on $y$ is used to signify that this is an estimate.
We can use this line to discuss properties of possums.
For instance, the equation predicts a possum with a total length
of 80 cm will have a head length of
\begin{align*}
\hat{y} &= 41 + 0.59\times 80 \\
	&= 88.2 % mm
\end{align*}
The estimate may be viewed as an average:
the equation predicts that possums with a total length of
80~cm will have an average head length of 88.2~mm.
Absent further information about an 80~cm possum,
the prediction for head length that uses the average
is a reasonable estimate.

\begin{figure}
  \centering
  \Figures[The same scatterplot showing head length against total length for 104 brushtail possums is shown. A linear trend line has been added with an equation of y-hat equals 41 plus 0.59 times x, which shows the clear upward trajectory of the data. Additionally, three points are highlighted. The first is labeled with an "X" and is at approximately (77, 85) and lies about 1 unit below the trend line. A second point labeled with a "plus sign" is at about (85, 98) and appears to be about 7 units above the trend line. The last point highlighted is a "triangle" and is located at about (95, 93) and is about 3 units below the trend line.]{0.7}{scattHeadLTotalLLine}
      {scattHeadLTotalLLineResiduals}
  \caption{A reasonable linear model was fit to represent
      the relationship between head length and total length.}
  \label{scattHeadLTotalLLine}
\end{figure}

\begin{examplewrap}
\begin{nexample}{What other variables might help us predict the
    head length of a possum besides its length?}
  Perhaps the relationship would be a little different for
  male possums than female possums,
  or perhaps it would differ for possums from one region
  of Australia versus another region.
  In Chapter~\ref{ch_regr_mult_and_log},
  we'll learn about how we can include more than one predictor.
  Before we get there, we first need to better understand
  how to best build a simple linear model with one predictor.
\end{nexample}
\end{examplewrap}


\subsection{Residuals}

\index{residual|(}

\noindent%
\termsub{Residuals}{residual} are the leftover variation in the data after accounting for the model fit:
\begin{align*}
\text{Data} = \text{Fit} + \text{Residual}
\end{align*}
Each observation will have a residual, and three of the
residuals for the linear model we fit for the \data{possum}
data is shown in
Figure~\ref{scattHeadLTotalLLine}.
If an observation is above the regression line, then its residual,
the vertical distance from the observation to the line, is positive.
Observations below the line have negative residuals.
One goal in picking the right linear model is for these residuals
to be as small as possible.

%\begin{figure}[h]
%  \centering
%  \Figures{0.7}{scattHeadLTotalLLine}
%      {scattHeadLTotalLLineResiduals}
%  \caption{The linear model from
%      Figure~\ref{scattHeadLTotalLLine}
%      where 3 residuals are highlighted.}
%  \label{scattHeadLTotalLLineResiduals}
%\end{figure}

Let's look closer at the three residuals featured in
Figure~\ref{scattHeadLTotalLLine}.
The observation marked by an ``$\times$'' has a small,
negative residual of about -1;
the observation marked by ``$+$'' has a large residual of about +7;
and the observation marked by ``$\triangle$'' has a moderate
residual of about -4.
The size of a residual is usually discussed in terms of its
absolute value.
For example, the residual for ``$\triangle$'' is larger than
that of ``$\times$'' because $|-4|$ is larger than $|-1|$.

\begin{onebox}{Residual: difference between observed and expected}
The residual of the $i^{th}$ observation $(x_i, y_i)$ is the difference of the observed response ($y_i$) and the response we would predict based on the model fit ($\hat{y}_i$):
\begin{eqnarray*}
e_i = y_i - \hat{y}_i
\end{eqnarray*}
We typically identify $\hat{y}_i$ by plugging $x_i$ into the model.
\end{onebox}

\begin{examplewrap}
\begin{nexample}{The linear fit shown in Figure~\ref{scattHeadLTotalLLine} is given as $\hat{y} = 41 + 0.59x$. Based on this line, formally compute the residual of the observation $(77.0, 85.3)$. This observation is denoted by ``$\times$'' in Figure~\ref{scattHeadLTotalLLine}.
Check it against the earlier visual estimate,~-1.}
We first compute the predicted value of point ``$\times$'' based on the model:
\begin{eqnarray*}
\hat{y}_{\times} = 41+0.59x_{\times} = 41+0.59\times 77.0 = 86.4
\end{eqnarray*}
Next we compute the difference of the actual head length and the predicted head length:
\begin{eqnarray*}
e_{\times} = y_{\times} - \hat{y}_{\times} = 85.3 -  86.4 = -1.1
\end{eqnarray*}
The model's error is $e_{\times} = -1.1$mm,
which is very close to the visual estimate of -1mm.
The negative residual indicates that the linear model
overpredicted head length for this particular possum.
\end{nexample}
\end{examplewrap}

\begin{exercisewrap}
\begin{nexercise}
If a model underestimates an observation, will the residual be positive or negative? What about if it overestimates the observation?\footnotemark
\end{nexercise}
\end{exercisewrap}
\footnotetext{If a model underestimates an observation, then the model estimate is below the actual. The residual, which is the actual observation value minus the model estimate, must then be positive. The opposite is true when the model overestimates the observation: the residual is negative.}

\begin{exercisewrap}
\begin{nexercise}
Compute the residuals for the ``$+$'' observation $(85.0, 98.6)$
and the ``$\triangle$'' observation $(95.5, 94.0)$ in the figure
using the linear relationship $\hat{y} = 41 + 0.59x$.\footnotemark{}
\end{nexercise}
\end{exercisewrap}
\footnotetext{($+$) First compute the predicted value based on
  the model:
  \begin{align*}
  \hat{y}_{+} = 41+0.59x_{+} = 41+0.59\times 85.0 = 91.15
  \end{align*}
  Then the residual is given by
  \begin{align*}
  e_{+} = y_{+} - \hat{y}_{+} = 98.6-91.15=7.45
  \end{align*}
  This was close to the earlier estimate of 7.

\noindent%
($\triangle$) $\hat{y}_{\triangle} = 41+0.59x_{\triangle} = 97.3$.
$e_{\triangle} = y_{\triangle} - \hat{y}_{\triangle} = -3.3$,
close to the estimate of -4.}

Residuals are helpful in evaluating how well a linear model
fits a data set.
We often display them in a \term{residual plot} such as the
one shown in Figure~\ref{scattHeadLTotalLResidualPlot}
for the regression line in Figure~\ref{scattHeadLTotalLLine}.
The residuals are plotted at their original horizontal locations
but with the vertical coordinate as the residual.
For instance, the point $(85.0,98.6)_{+}$ had a residual
of 7.45, so in the residual plot it is placed at $(85.0, 7.45)$.
Creating a residual plot is sort of like tipping the
scatterplot over so the regression line is horizontal. 
\index{data!possum|)}

\begin{figure}[h]
  \centering
  \Figure[A residual plot for the trend line fit to the brushtail possum data is shown. Here, the horizontal axis is the same -- representing "total length", it spans 75 to 97 -- while the vertical axis represents "Residuals" and spans from about -7 to positive 8. There is on evident trend in the residuals. Three points are specifically highlighted to reflect the three points discussed in the last figure. The first is labeled with an "X" with a total length of 77 and a residual of about -1. The second is labeled with a "plus sign" and has a total length of 85 and a residual of about 7. The last point highlighted is a "triangle" with a total length of about 95 and a residual of about -3. Note that the location of the residuals above and below the trend line reflects exactly with whether the residual is positive or negative, respectively.]{0.7}{scattHeadLTotalLResidualPlot}
  \caption{Residual plot for the model in
      Figure~\ref{scattHeadLTotalLLine}.}
  \label{scattHeadLTotalLResidualPlot}
\end{figure}

\D{\newpage}

\begin{examplewrap}
\begin{nexample}{One purpose of residual plots is to identify
    characteristics or patterns still apparent in data after
    fitting a model.
    Figure~\ref{sampleLinesAndResPlots} shows three scatterplots
    with linear models in the first row and residual plots in the
    second row.
    Can you identify any patterns remaining in the residuals?}

  In the first data set (first column), the residuals show
  no obvious patterns.
  The residuals appear to be scattered randomly around the
  dashed line that represents 0.

  The second data set shows a pattern in the residuals.
  There is some curvature in the scatterplot, which is more
  obvious in the residual plot.
  We should not use a straight line to model these data.
  Instead, a more advanced technique should be used.

  The last plot shows very little upwards trend, and the
  residuals also show no obvious patterns.
  It is reasonable to try to fit a linear model to the data.
  However, it is unclear whether there is statistically
  significant evidence that the slope parameter is different
  from zero.
  The point estimate of the slope parameter, labeled $b_1$,
  is not zero, but we might wonder if this could just be due
  to chance.
  We will address this sort of scenario in
  Section~\ref{inferenceForLinearRegression}.
\end{nexample}
\end{examplewrap}

\begin{figure}
   \centering
   \Figure[Sample data with their best fitting lines (top row of three plots) and their corresponding residual plots (bottom row of three plots). The upper left plot shows a scatterplot where the data trend downwards steadily with a straight line fit to the data, which appears to fit well everywhere. The bottom left plot is the residual plot of this first scatterplot, and it likewise shows no pattern in the residuals when looking left to right. The upper middle plot shows data with a downward trend, but the data's trend is more steep on the right side of the plot, so the overall shape of the data is that it trends downward and curves downward. A straight, downward-trending line has also been fit to this data, but it doesn't fit as well. The data are below this downward trending line initially, but it is above the line in the middle, and finally on the right it is once again below the linear trend line. The residual plot for this scatterplot is shown in the lower middle plot, and the curvature in the residuals is more evident than what was visible in the scatterplot: the residuals have negative values on the left and trend upwards until peaking with positive residuals in the middle, and then trending back down and having negative residual values again on the right. The last scatterplot in the upper right shows data with very little trend, but a slightly-upward trending straight line has been fit to the data. The corresponding residual plot, shown as the bottom right plot, also shows data with no evident trend or pattern, where observations appear relatively randomly scattered above and below 0 (in the vertical).]{0.9}{sampleLinesAndResPlots}
   \caption{Sample data with their best fitting lines (top row) and their corresponding residual plots (bottom row).}
   \label{sampleLinesAndResPlots}
\end{figure}

\index{residual|)}


\subsection{Describing linear relationships with correlation}

\index{correlation|(}

\noindent%
We've seen plots with strong linear relationships and
others with very weak linear relationships.
It would be useful if we could quantify the strength of these
linear relationships with a statistic.

\begin{onebox}{Correlation: strength of a linear relationship}
  \termsub{Correlation}{correlation}, which always takes values
  between -1 and 1, describes the strength of the linear
  relationship between two variables.
  We denote the correlation by $R$.
\end{onebox}

We can compute the correlation using a formula, just as we did
with the sample mean and standard deviation.
This formula is rather complex,\footnote{Formally,
  we can compute the correlation for observations $(x_1, y_1)$,
  $(x_2, y_2)$, ..., $(x_n, y_n)$ using the formula
  \begin{align*}
  R = \frac{1}{n-1}
      \sum_{i=1}^{n} \frac{x_i-\bar{x}}{s_x}\frac{y_i-\bar{y}}{s_y}
  \end{align*}
  where $\bar{x}$, $\bar{y}$, $s_x$, and $s_y$ are the sample
  means and standard deviations for each variable.}
and like with other statistics, we generally perform the
calculations on a computer or calculator.
Figure~\ref{posNegCorPlots} shows eight plots and their
corresponding correlations.
Only when the relationship is perfectly linear is the
correlation either -1 or~1.
If~the relationship is strong and positive, the correlation
will be near~+1.
If~it is strong and negative, it will be near~-1.
If~there is no apparent linear relationship between the
variables, then the correlation will be near zero.

\begin{figure}
   \centering
   \Figure[Eight scatterplots are shown, each with their correlation noted. Each scatterplot appears to represent about 50 points. The first has a correlation of R equals 0.33, and there is a slight upward trend evident in the data -- if a trend line were drawn for this data, much of the data would fall relatively far from the line. The second plot has a correlation of R equals 0.69, and a clearer upward trend is evident, but it is still pretty volatile with many points deviating far from where the trend line would be. The third plot has a correlation of 0.98, and the data show a very clear upward trend, where if a trend line were drawn, the data would be (relatively) quite close to this line. The fourth plot shows a correlation of R equals 1.00, and here the points appear exactly on a line with an upward trajectory. The fifth plot shows data with a correlation of R equals 0.08, where no trend is visually evident in the data. The sixth plot has a correlation of R equals -0.64, and a downward trend is evident in the data, but the individual observations would in many cases be pretty distant from any trend line fit to the data (on a relative basis). The seventh plot has a correlation of R equals -0.92 and shows data with a clear downward trend, where the data would deviate just a modest amount from a trend line fit to the data. The last plot shows a correlation of R equals -1, where the observations would fit exactly on a line trending downwards.]{0.9}{posNegCorPlots}
   \caption{Sample scatterplots and their correlations.
       The first row shows variables with a positive
       relationship, represented by the trend up and to
       the right.
       The second row shows one plot with an approximately neutral trend
       and three plots with a negative trend.}
   \label{posNegCorPlots}
\end{figure}

The correlation is intended to quantify the strength of a linear trend. Nonlinear trends, even when strong, sometimes produce correlations that do not reflect the strength of the relationship; see three such examples in Figure~\ref{corForNonLinearPlots}.

\begin{figure}[h]
   \centering
   \Figures[Three scatterplots are shown. In each case, there is a strong relationship between the variables. However, because the relationship is nonlinear, the correlation is relatively weak. The first plot shows data that trends upwards on the left before peaking and then trending downward on the right -- the correlation of the data in this plot is R equals -0.23. The second plot shows data with a sharp downward trend on the left before reaching a trough and rising then sharply upward before reaching a peak and then trending sharply downwards again -- the correlation of the data in this plot is R equals 0.31. The third plots shows data that without a trend on the far left, followed by a steep drop, a trough, and then a steep rise to a peak, and then another drop and then finally a slight increase at the end -- the correlation of the data in this plot is R equals 0.50.]{0.85}{posNegCorPlots}{corForNonLinearPlots}
   \caption{Sample scatterplots and their correlations.
       In each case, there is a strong relationship between
       the variables.
       However, because the relationship is nonlinear,
       the correlation is relatively weak.}
   \label{corForNonLinearPlots}
\end{figure}

\begin{exercisewrap}
\begin{nexercise}
No straight line is a good fit for the data sets
represented in Figure~\ref{corForNonLinearPlots}.
Try drawing nonlinear curves on each plot.
Once you create a curve for each, describe what is important
in your~fit.\footnotemark{}
\index{correlation|)}
\end{nexercise}
\end{exercisewrap}
\footnotetext{We'll leave it to you to draw the lines. In general, the lines you draw should be close to most points and reflect overall trends in the data.}

%\begin{examplewrap}
%\begin{nexample}{What other variables might help us predict the
%    head length of a possum besides its length?}
%  Perhaps the relationship would be a little different for
%  male possums than female possums,
%  as shown in Figure~\ref{scattHeadLTotalLSex},
%  Or perhaps it would differ for possums from one region
%  of Australia versus another region.
%  In Chapter~\ref{ch_regr_mult_and_log},
%  we'll learn about how we can include more than one predictor.
%  Before we get there, we first need to better understand
%  how to best build a simple linear model with one predictor.
%\end{nexample}
%\end{examplewrap}
%
%\begin{figure}
%  \centering
%  \Figure{0.6}{scattHeadLTotalLSex}
%  \caption{Possums where the possum's sex is represented
%      by the plotting icon.}
%  \label{scattHeadLTotalLSex}
%\end{figure}


{\exercisesheader{}

% 1

\eoce{\qt{Visualize the residuals\label{visualize_residuals}} 
The scatterplots shown below each have a 
superimposed regression line. If we were to construct a residual plot 
(residuals versus $x$) for each, describe what those plots would look 
like.
\begin{center}
\FigureFullPath[A scatterplot is shown, where the data have a steady upward trend throughout. The observations above and below the line appear random and have stable variability moving from left to right.]{0.42}{ch_regr_simple_linear/figures/eoce/visualize_residuals/visualize_residuals_linear}
\FigureFullPath[A scatterplot is shown, where the data have a steady upward trend throughout. The observations above and below the line appear random. If looking at the leftmost region of data, the observations are more broadly scattered around the line, while when moving further right the variability of the points around the line gets notably smaller by a factor of at least 5 (if using standard deviation).]{0.42}{ch_regr_simple_linear/figures/eoce/visualize_residuals/visualize_residuals_fan_back}
\end{center}
}{}

% 2

\eoce{\qt{Trends in the residuals\label{trends_in_residuals}} 
Shown below are two plots of residuals 
remaining after fitting a linear model to two different sets of data. 
Describe important features and determine if a linear model would be 
appropriate for these data. Explain your reasoning.
\begin{center}
\FigureFullPath[A scatterplot of the residuals is shown. When looking at any horizontal region of the plot, the observations are consistently scattered around the "y equals 0" line. On the left, the points tend to be very close to this horizontal 0 line. The further moving to the right, the more variability that is evident in the observations around "y equals 0".]{0.42}{ch_regr_simple_linear/figures/eoce/trends_in_residuals/trends_in_residuals_fan} 
\FigureFullPath[A scatterplot of the residuals is shown. The points on the very left tend to be below the "y equals 0" line for the first 5\% of the horizontal region, where the trend is sharply upwards to the "y equals 0" line. The points then tend to be stably clustered around "y equals 0", if not slightly above, with a slight downward trend evident in the observations on the right half of the plot. The vertical variability of observations is about stable throughout.]{0.42}{ch_regr_simple_linear/figures/eoce/trends_in_residuals/trends_in_residuals_log}
\end{center}
}{}

% 3

\eoce{\qt{Identify relationships, Part I\label{identify_relationships_1}} 
For each of the six plots, 
identify the strength of the relationship (e.g. weak, moderate, or 
strong) in the data and whether fitting a linear model would be 
reasonable.
\begin{center}
\FigureFullPath[A scatterplot is shown. The observations start in the upper left corner of the plot, trend sharply downwards before tapering off and stabilizing at about the middle of the plot, before steadily and then faster rising again to the upper right corner of the plot. The trend is approximately symmetric from left-to-right.]{0.32}{ch_regr_simple_linear/figures/eoce/identify_relationships_1/identify_relationships_u}
\FigureFullPath[A scatterplot is shown. The start on the lower left corner, only spanning about 20\% of the vertical region of the plot, and have a steady upwards trend to the upper right corner of the plot. The vertical variability of the points around the trend is relatively stable across the plot.]{0.32}{ch_regr_simple_linear/figures/eoce/identify_relationships_1/identify_relationships_lin_pos_strong}
\FigureFullPath[A scatterplot is shown. On the left side of the plot, the points are appear randomly scattered across the full range of the plot, and this property holds across the entire plot. No trend is evident.]{0.32}{ch_regr_simple_linear/figures/eoce/identify_relationships_1/identify_relationships_lin_pos_weak}
%
\FigureFullPath[A scatterplot is shown. On the left side of the plot, the observations are in concentrated in the bottom half of the plot but rise steadily. The trend peaks near the center of the plot, where nearly all the points in the (horizontal) center region of the scatterplot are concentrated in the upper half of the scatterplot. On the right side of the plot, the points show a trend downwards, with points concentrated in the lower quarter of the scatterplot for the rightmost handful of points.]{0.32}{ch_regr_simple_linear/figures/eoce/identify_relationships_1/identify_relationships_n}
\FigureFullPath[A scatterplot is shown. The start on the upper left corner, only spanning about 20\% of the vertical region of the plot, and have a steady downwards trend to the bottom right corner of the plot. The vertical variability of the points around the trend is relatively stable across the plot.]{0.32}{ch_regr_simple_linear/figures/eoce/identify_relationships_1/identify_relationships_lin_neg_strong}
\FigureFullPath[A scatterplot is shown. On the left side of the plot, the points are appear randomly scattered across the full range of the plot, and this property holds across the entire plot. No trend is evident or at least obvious.]{0.32}{ch_regr_simple_linear/figures/eoce/identify_relationships_1/identify_relationships_none}
\end{center}
}{}

\D{\newpage}

% 4

\eoce{\qt{Identify relationships, Part II\label{identify_relationships_2}} 
For each of the six plots, 
identify the strength of the relationship (e.g. weak, moderate, or 
strong) in the data and whether fitting a linear model would be 
reasonable.
\begin{center}
\FigureFullPath[A scatterplot is shown. On the left side of the plot, the observations are in concentrated in the upper corner of the plot, with a sharp trend downwards, before stabilizing, then rising slightly at halfway through the plot, reaching a peak, and then declining again, with a sharp decline on the right-most portion of the plot to the bottom-right corner of the plot.]{0.32}{ch_regr_simple_linear/figures/eoce/identify_relationships_2/identify_relationships_s}
\FigureFullPath[A scatterplot is shown. On the left side of the plot, the observations are concentrated around a region about 30\% of the way up from the bottom-left corner of the plot, there is a slight downward trend that reaches the bottom area of the plot for about the center half of the plot, then the points rise gradually and then sharply in the last 25-30\% of the plot.]{0.32}{ch_regr_simple_linear/figures/eoce/identify_relationships_2/identify_relationships_hockey_stick}
\FigureFullPath[A scatterplot is shown. Pointers in the leftmost region of the plot are concentrated in the lower-left corner, ranging from the bottom up to about 25\% of the way up the plot. The points follow a steady upward trend to the top-right corner of the plot and show consistent vertical variability around the trend throughout.]{0.32}{ch_regr_simple_linear/figures/eoce/identify_relationships_2/identify_relationships_pos_lin_strong}
%
\FigureFullPath[A scatterplot is shown. The points appear randomly scattered across the left, middle, and right portion of the plot. There might be a very slight upward trend.]{0.32}{ch_regr_simple_linear/figures/eoce/identify_relationships_2/identify_relationships_pos_weak}
\FigureFullPath[A scatterplot is shown. The points appear randomly scattered across the left, middle, and right portion of the plot. There is a very slight downward trend.]{0.32}{ch_regr_simple_linear/figures/eoce/identify_relationships_2/identify_relationships_pos_weaker}
\FigureFullPath[A scatterplot is shown. The points on the leftmost side are concentrated in the upper half of the plot, and the data trend steadily downwards and with consistent variability to the bottom right portion of the plot.]{0.32}{ch_regr_simple_linear/figures/eoce/identify_relationships_2/identify_relationships_neg_lin_weak}
\end{center}
}{}

% 5

\eoce{\qt{Exams and grades\label{exams_grades_correlation}} 
The two scatterplots below show the 
relationship between final and mid-semester exam grades recorded 
during several years for a Statistics course at a university.
\begin{parts}
\item Based on these graphs, which of the two exams has the strongest 
correlation with the final exam grade? Explain.
\item Can you think of a reason why the correlation between the exam 
you chose in part (a) and the final exam is higher?
\end{parts}
\begin{center}
\FigureFullPath[A scatter plot with 100 points is shown with an upward trending line fit to the data. Exam 1 scores are on the horizontal axis and range from 40 to 100. Final Exam scores are on the vertical axis and also range from 40 to 100. Only about ten Exam 1 scores are below 60, and these have Final Exam scores between about 55 and 85. Exam 1 scores between 60 and 80 represent about 50\% of the points shown and have Final Exam scores mostly between 50 and 85. For the points where Midterm 1 scores are larger than 80, Final Exam scores mostly lie between 65 and 90, where a slightly upward trend is evident.]{0.485}{ch_regr_simple_linear/figures/eoce/exams_grades_correlation/exam_grades_1}
\hspace{0.02\textwidth}%
\FigureFullPath[A scatter plot with 100 points is shown with an upward trending line fit to the data. Exam 2 scores are on the horizontal axis and range from 40 to 100. Final Exam scores are on the vertical axis and also range from 40 to 100. Midterm 2 scores are roughly uniformly distributed across the full range. For Exam 2 scores below 60, these mostly have Final Exam scores between about 45 and 70. Exam 2 scores between 60 and 80 have Final Exam scores mostly between 55 and 80. For the points where Midterm 2 scores are larger than 80, Final Exam scores mostly lie between 70 and 90.]{0.485}{ch_regr_simple_linear/figures/eoce/exams_grades_correlation/exam_grades_2}
\end{center}
}{}

\D{\newpage}

% 6

\eoce{\qt{Husbands and wives, Part I\label{husbands_wives_correlation}}
The Great Britain Office of Population Census and Surveys once 
collected data on a random sample of 170 married couples in 
Britain, recording the age (in years) and heights (converted 
here to inches) of the husbands and wives.\footfullcite{Hand:1994} 
The scatterplot on the left shows the wife's age plotted against her 
husband's age, and the plot on the right shows wife's height 
plotted against husband's height.
\begin{center}
\FigureFullPath[A scatterplot is shown. The horizontal axis represents "Husband's Age (in years)" with values ranging from about 20 to 65. The vertical axis represents "Wife's Age (in years)" with values ranging from about 18 to 65. When husband age is between 20 and 30, wife age mostly ranges from 18 to about 30. When husband age is between 30 and 40, wife age mostly ranges from 23 to about 40. When husband age is between 40 and 50, wife age mostly ranges from 35 to about 50. When husband age is between 50 and 60, wife age mostly ranges from 45 to about 60. When husband age is larger than 60, wife age mostly ranges from 55 to about 65.]{0.35}{ch_regr_simple_linear/figures/eoce/husbands_wives_correlation/husbands_wives_age} 
\hspace{5mm}
\FigureFullPath[A scatterplot is shown. The horizontal axis represents "Husband's Height (in inches)" with values ranging from about 60 to 75. The vertical axis represents "Wife's Height (in inches)" with values ranging from about 55 to 70. When husband height is between 60 and 65, wife height mostly ranges from about 58 to 65 inches, though there are only about 10 points in this range, which is about 5\% of the data. When husband height is between 65 and 70, wife height mostly ranges from 57 to 68 inches. When husband height is larger than 70 inches, wife height mostly ranges from 61 to about 74 inches.]{0.35}{ch_regr_simple_linear/figures/eoce/husbands_wives_correlation/husbands_wives_height}
\end{center}
\begin{parts}
\item Describe the relationship between husbands' and wives' ages.
\item Describe the relationship between husbands' and wives' heights.
\item Which plot shows a stronger correlation? Explain your reasoning.
\item Data on heights were originally collected in centimeters, and 
then converted to inches. Does this conversion affect the correlation 
between husbands' and wives' heights?
\end{parts}
}{}

% 7

\eoce{\qt{Match the correlation, Part I\label{match_corr_1}} 
Match each correlation to the corresponding scatterplot.

\noindent%
\begin{minipage}[c]{0.17\textwidth}
\begin{parts}
\item $R = -0.7$
\item $R = 0.45$ 
\item $R = 0.06$
\item $R = 0.92$
\end{parts}\vspace{3mm}
\end{minipage}%
\begin{minipage}[c]{0.83\textwidth}
\FigureFullPath[A scatterplot is shown. The observations start in the upper left corner of the plot, trend sharply downwards before tapering off and stabilizing at about the middle of the plot, before steadily and then faster rising again to the upper right corner of the plot. The trend is approximately symmetric from left-to-right.]{0.245}{ch_regr_simple_linear/figures/eoce/match_corr_1/match_corr_1_u}
\FigureFullPath[A scatterplot is shown. The start on the lower left corner, only spanning about 20\% of the vertical region of the plot, and have a steady upwards trend to the upper right corner of the plot. The vertical variability of the points around the trend is relatively stable across the plot.]{0.245}{ch_regr_simple_linear/figures/eoce/match_corr_1/match_corr_2_strong_pos}
\FigureFullPath[A scatterplot is shown. The points appear randomly scattered across the left, middle, and right portion of the plot. There is a very slight upward trend.]{0.245}{ch_regr_simple_linear/figures/eoce/match_corr_1/match_corr_3_weak_pos}
\FigureFullPath[A scatterplot is shown. The points appear randomly scattered across the left, middle, and right portion of the plot. There is a very slight downward trend.]{0.245}{ch_regr_simple_linear/figures/eoce/match_corr_1/match_corr_4_weak_neg}
\end{minipage}
}{}

% 8

\eoce{\qt{Match the correlation, Part II\label{match_corr_2}} 
Match each correlation to the corresponding scatterplot.

\noindent%
\begin{minipage}[c]{0.17\textwidth}
\begin{parts}
\item $R = 0.49$
\item $R = -0.48$ 
\item $R = -0.03$ 
\item $R = -0.85$
\end{parts}\vspace{3mm}
\end{minipage}%
\begin{minipage}[c]{0.83\textwidth}
\FigureFullPath[A scatterplot is shown. For the left half of the plot, the points are scattered around the upper half of the plot. On the right portion of the plot, the data show a clear downward trend, and for the points on the far right, they are concentrated in the lower 25\% of the plot.]{0.245}{ch_regr_simple_linear/figures/eoce/match_corr_2/match_corr_1_strong_neg_curved}
\FigureFullPath[A scatterplot is shown. The points appear randomly scattered across the left, middle, and right portion of the plot. There is a very slight upward trend.]{0.245}{ch_regr_simple_linear/figures/eoce/match_corr_2/match_corr_2_weak_pos}
\FigureFullPath[A scatterplot is shown. The observations start in the lower left corner of the plot, trend sharply upwards before tapering off and stabilizing at about the middle of the plot, before steadily and then faster dropping to the lower right corner of the plot. The trend is approximately symmetric from left-to-right.]{0.245}{ch_regr_simple_linear/figures/eoce/match_corr_2/match_corr_3_n}
\FigureFullPath[A scatterplot is shown. The points appear randomly scattered across the left, middle, and right portion of the plot. There is a very slight downward trend.]{0.245}{ch_regr_simple_linear/figures/eoce/match_corr_2/match_corr_4_weak_neg}
\end{minipage}
}{}

% 9

\eoce{\qt{Speed and height\label{speed_height_gender}} 1,302 UCLA students 
were asked to fill out a survey where they were asked about their height, 
fastest speed they have ever driven, and gender. The scatterplot on the 
left displays the relationship between height and fastest speed, and 
the scatterplot on the right displays the breakdown by gender in 
this relationship.
\begin{center}
\FigureFullPath[A scatterplot is shown. The horizontal axis represents "Height (in inches)" with values ranging from about 50 to 80. The vertical axis represents "Fastest Speed (in mph)" and has values ranging from 0 to 150. First, it is worth noting that there several points along the bottom of the plot with a fastest speed of 0 mph. The remainder of the description will concentrate on the other points. A small portion of the points are shown with heights below 60 inches, and these have fastest speeds mostly ranging from about 70 to 110 mph. For points shown with heights between 60 and 70, fastest speeds mostly ranged from about 30 to 120 mph. For points shown with heights of 70 or more, fastest speeds mostly ranged from about 50 to 140 mph. There were no points corresponding to heights greater than 75 that had fastest speeds slower than about 75 mph, which left a sort of gap in the lower right portion of the scatterplot.]{0.4}{ch_regr_simple_linear/figures/eoce/speed_height_gender/speed_height}
\hspace{0.02\textwidth}%
\FigureFullPath[A scatterplot is shown, where points are colored to differentiate between males and females. The horizontal axis represents "Height (in inches)" with values ranging from about 50 to 80. The vertical axis represents "Fastest Speed (in mph)" and has values ranging from 0 to 150. Female heights are largely 70 inches or smaller, while Male heights are largely 65 inches and taller. When focusing exclusively on Females, no upward trend is evident, with about 95\% of observations having Fastest Speed between about 30 mph and 120 mph. When focusing exclusively on Males, no upward trend is evident there either, with about 95\% of observations having Fastest Speed between about 50 mph and 140 mph. In contrast, if we ignore the male/female differentiation, there is a slight upward trend in the points.]{0.4}{ch_regr_simple_linear/figures/eoce/speed_height_gender/speed_height_gender.pdf}
\end{center}
\begin{parts}
\item Describe the relationship between height and fastest speed.
\item Why do you think these variables are positively associated?
\item What role does gender play in the relationship between height 
and fastest driving speed?
\end{parts}
}{}

% 10

\eoce{\qt{Guess the correlation\label{guess_correlation}} Eduardo and Rosie 
are both collecting data on number of rainy days in a year and the total 
rainfall for the year. Eduardo records rainfall in inches and Rosie in 
centimeters. How will their correlation coefficients compare?
}{}

% 11

\eoce{\qt{The Coast Starlight, Part I\label{coast_starlight_corr_units}} 
The Coast Starlight Amtrak train runs from Seattle to Los Angeles. 
The scatterplot below displays the distance between each stop 
(in miles) and the amount of time it takes to travel from one stop 
to another (in minutes).\vspace{2mm}

\noindent\begin{minipage}[c]{0.4\textwidth}
{\raggedright\begin{parts}
\item Describe the relationship between distance and travel time.
\item How would the relationship change if travel time was instead measured 
in hours, and distance was instead measured in kilometers?
\item Correlation between travel time (in miles) and distance (in minutes) 
is $r = 0.636$. What is the correlation between travel time (in kilometers) 
and distance (in hours)?
\end{parts}\vspace{7mm}}
\end{minipage}
\begin{minipage}[c]{0.1\textwidth}
$\:$\\
\end{minipage}
\begin{minipage}[c]{0.485\textwidth}
\FigureFullPath[A scatterplot is shown with about 15 points. The horizontal axis represents "Distance (miles)" with values ranging from just over 0 to about 350. The vertical axis represents "Travel Time (in minutes)" and has values ranging from about 20 to 380. The point with the smallest distance -- about 10 miles -- shows a travel time of about 40 minutes. Next, there is a cluster of 6 points with distances between 40 and 60 miles and travel times ranging from about 20 to 60 minutes. The remainder of the points are scattered pretty broadly but may show a slightly upward trend. A few points that highlight the widely varying nature of the data are located at the following approximate locations: (190 miles, 60 minutes), (240 miles, 250 minutes), (250 miles, 380 minutes), and (350 miles, 200 minutes).]{}{ch_regr_simple_linear/figures/eoce/coast_starlight_corr_units/coast_starlight}
\end{minipage}
}{}

% 12

\eoce{\qt{Crawling babies, Part I\label{crawling_babies_corr_units}}  
A study conducted at the University of Denver investigated whether babies 
take longer to learn to crawl in cold months, when they are often bundled 
in clothes that restrict their movement, than in warmer months.
\footfullcite{Benson:1993} Infants born during the study year were split 
into twelve groups, one for each birth month. We consider the average 
crawling age of babies in each group against the average temperature when 
the babies are six months old (that's when babies often begin trying to 
crawl). Temperature is measured in degrees Fahrenheit (\degree F) and age 
is measured in weeks.\vspace{2mm}

\noindent\begin{minipage}[c]{0.4\textwidth}
{\raggedright\begin{parts}
\item Describe the relationship between temperature and crawling age.
\item How would the relationship change if temperature was measured in 
degrees Celsius (\degree C) and age was measured in months?
\item The correlation between temperature in \degree F and age in weeks 
was $r=-0.70$. If we converted the temperature to \degree C and age to 
months, what would the correlation be?
\end{parts}\vspace{3mm}}
\end{minipage}
\begin{minipage}[c]{0.1\textwidth}
$\:$\\
\end{minipage}
\begin{minipage}[c]{0.485\textwidth}
\FigureFullPath[A scatterplot is shown with a dozen points. The horizontal axis is "Temperature (F)" with values ranging from 30 to 75. The vertical axis is "Average Crawling Age (weeks)" with values ranging from 28.5 to 34. For those points with temperatures from 30 to 40, average crawling ages range from 31.5 to 34. For the single point with temperatures between 40 to 50, average crawling age was about 33.5. For the two points with temperature between 50 and 60, average crawling age was 28.5 and 32.5. For the last 4 points with temperature above 60, average crawling ages were 32, 30, 30, and 30.5.]{}{ch_regr_simple_linear/figures/eoce/crawling_babies_corr_units/crawling_babies}
\end{minipage}
}{}

\D{\newpage}

% 13

\eoce{\qt{Body measurements, Part I\label{body_measurements_shoulder_height_corr_units}} 
Researchers studying anthropometry collected body girth measurements and 
skeletal diameter measurements, as well as age, weight, height and gender 
for 507 physically active individuals.\footfullcite{Heinz:2003} The 
scatterplot below shows the relationship between height and shoulder 
girth (over deltoid muscles), both measured in centimeters.\vspace{3mm}

\noindent%
\begin{minipage}[c]{0.4\textwidth}
{\raggedright\begin{parts}
\item Describe the relationship between shoulder girth and height.
\item How would the relationship change if shoulder girth was measured 
in inches while the units of height remained in centimeters?
\end{parts}\vspace{20mm}}
\end{minipage}
\begin{minipage}[c]{0.1\textwidth}                  
$\:$\\
\end{minipage}
\begin{minipage}[c]{0.485\textwidth}
\FigureFullPath[A scatter plot with several hundred points is shown. The horizontal axis represents "Shoulder Girth (cm)" with values ranging from about 85 to 135. The vertical axis represents "Height (cm)" with values ranging from about 145 to 200. For points where Shoulder Girth is smaller than 100, 95\% of points have heights between 152 and 170. For points where Shoulder Girth is between 100 and 110, 95\% of points have heights between 155 and 180. For points where Shoulder Girth is between 110 and 120, 95\% of points have heights between 162 and 190. For points where Shoulder Girth larger than 120, 95\% of points have heights between 170 and 190.]{}{ch_regr_simple_linear/figures/eoce/body_measurements_shoulder_height_corr_units/body_measurements_height_shoulder_girth}
\end{minipage}
}{}

% 14

\eoce{\qt{Body measurements, Part II\label{body_measurements_hip_weight_corr_units}} 
The scatterplot below shows the relationship between weight 
measured in kilograms and hip girth measured in centimeters 
from the data described in 
Exercise~\ref{body_measurements_shoulder_height_corr_units}.%
\vspace{3mm}

\noindent%
\begin{minipage}[c]{0.4\textwidth}
{\raggedright\begin{parts}
\item Describe the relationship between hip girth and weight.
\item How would the relationship change if weight was measured in pounds 
while the units for hip girth remained in centimeters?
\end{parts}\vspace{20mm}}
\end{minipage}
\begin{minipage}[c]{0.1\textwidth}
$\:$\\
\end{minipage}
\begin{minipage}[c]{0.485\textwidth}
\FigureFullPath[A scatter plot with several hundred points is shown. The horizontal axis represents "Hip Girth (cm)" with values ranging from about 80 to 115, with about 4 observations with larger hip girth up to about 130 cm. The vertical axis represents "Weight (kg)" with values ranging from about 40 to 105, with a few observations with larger weights up to 120. For points where Hip Girth is smaller than 90, 95\% of points have weight between roughly 45 and 60. For points where Hip Girth is between 90 and 100, 95\% of points have heights between roughly 50 and 80. For points where Hip Girth is between 100 and 110, 95\% of points have heights between roughly 65 and 90. For points where Hip Girth is between 110 and 115, points have heights between roughly 70 and 105. There are four additional points located at about (115, 120), (115, 90), (118, 90), and (128, 105).]{}{ch_regr_simple_linear/figures/eoce/body_measurements_hip_weight_corr_units/body_measurements_weight_hip_girth.pdf}
\end{minipage}
}{}

% 15

\eoce{\qt{Correlation, Part I\label{corr_husband_wife_age}} What would be the 
correlation between the ages of husbands and wives if men always married 
woman who were
\begin{parts}
\item 3 years younger than themselves? 
\item 2 years older than themselves? 
\item half as old as themselves?
\end{parts}
}{}

% 16

\eoce{\qt{Correlation, Part II\label{corr_men_women_salary}} What would be the 
correlation between the annual salaries of males and females at a company 
if for a certain type of position men always made
\begin{parts}
\item \$5,000 more than women?
\item 25\% more than women?
\item 15\% less than women?
\end{parts}
}{}
}







%__________________
\section{Least squares regression}
\label{fittingALineByLSR}

\index{least squares regression|(}

Fitting linear models by eye is open to criticism since
it is based on an individual's preference.
In this section, we use \emph{least squares regression}
as a more rigorous approach.


\subsection{Gift aid for freshman at Elmhurst College}

This section considers family income and gift aid data from
a random sample of fifty students in the freshman class of
Elmhurst College in Illinois.
Gift aid is financial aid that does not need to be paid back,
as opposed to a loan.
A scatterplot of the data is shown in
Figure~\ref{elmhurstScatterW2Lines}
along with two linear fits.
The lines follow a negative trend in the data;
students who have higher family incomes tended to have lower
gift aid from the university.

\begin{figure}[h]
  \centering
  \Figures[A scatterplot is shown for a random sample of 50 freshman students from Elmhurst College. The horizontal axis is for "family income" and has values ranging from \$0 to about \$300,000. The vertical axis is for "gift aid" and has values ranging from \$0 to about \$35,000. Two lines are fit to the data, which show a downward trend, representing a slight downward trend in the data. One of those lines is a solid line representing what is called the "least squares line". About 10 observations are shown where family income is between \$0 and \$50,000, and gift aid for these values is roughly between \$17,000 and \$28,000. About 20 observations are shown where family income is between \$50,000 and \$100,000, and gift aid for these values is roughly between \$10,000 and \$33,000. About 10 observations are shown where family income is between \$100,000 and \$150,000, and gift aid for these values is roughly between \$9,000 and \$25,000. Three observations are shown where family income is between \$150,000 and \$200,000, and gift aid for these values of \$25,000, \$12,000, and \$13,000. Six more observations are shown where family income is larger than \$200,000, and gift aid for these values range from about \$7,000 to \$22,000, \$12,000, and \$13,000. The data in this graph will be frequently discussed throughout this section and referred to as the "Elmhurst data".]{0.67}{elmhurstPlots}{elmhurstScatterW2Lines}
  \caption{Gift aid and family income for a random sample of
      50~freshman students from Elmhurst College.
      Two lines are fit to the data, the solid line being the
      \emph{least squares line}.}
  \label{elmhurstScatterW2Lines}
\end{figure}

\begin{exercisewrap}
\begin{nexercise}
Is the correlation positive or negative in Figure~\ref{elmhurstScatterW2Lines}?\footnotemark
\end{nexercise}
\end{exercisewrap}
\footnotetext{Larger family incomes are associated with lower amounts of aid, so the correlation will be negative. Using a computer, the correlation can be computed: -0.499.}


\subsection{An objective measure for finding the best line}

We begin by thinking about what we mean by ``best''.
Mathematically, we want a line that has small residuals.
The first option that may come to mind is to minimize the
sum of the residual magnitudes:
\begin{align*}
|e_1| + |e_2| + \dots + |e_n|
\end{align*}
which we could accomplish with a computer program.
The resulting dashed line shown in
Figure~\ref{elmhurstScatterW2Lines}
demonstrates this fit can be quite reasonable.
However, a more common practice is to choose the line that
minimizes the sum of the squared residuals:
\begin{align*}
e_{1}^2 + e_{2}^2 + \dots + e_{n}^2
\end{align*}


The line that minimizes this \term{least squares criterion}
is represented as the solid line in
Figure~\ref{elmhurstScatterW2Lines}.
This is commonly called the \term{least squares line}.
The following are three possible reasons to choose this option
instead of trying to minimize the sum of residual magnitudes
without any squaring:
\begin{enumerate}
\item
    It is the most commonly used method.
\item
    Computing the least squares line is widely supported
    in statistical software.
\item
    In many applications, a residual twice as large
    as another residual is more than twice as bad.
    For example, being off by 4 is usually more than twice
    as bad as being off by 2.
    Squaring the residuals accounts for this discrepancy.
\end{enumerate}
The first two reasons are largely for tradition and convenience;
the last reason explains why the least squares criterion
is typically most helpful.\footnote{There
  are applications where the sum of residual magnitudes
  may be more useful, and there are plenty of other criteria
  we might consider.
  However, this book only applies the least squares criterion.}


\subsection{Conditions for the least squares line}

\noindent%
When fitting a least squares line, we generally require
\begin{description}
\setlength{\itemsep}{0mm}
\item[Linearity.]
    The data should show a linear trend.
    If there is a nonlinear trend (e.g. left panel of
    Figure~\ref{whatCanGoWrongWithLinearModel}),
    an advanced regression method from another book
    or later course should be applied.
\item[Nearly normal residuals.]
    Generally, the residuals must be nearly normal.
    When this condition is found to be unreasonable,
    it is usually because of outliers or concerns
    about influential points,
%    The theoretical condition is that the residuals
%    must be normally distributed.
%    The importance of this condition depends on a few factors:
%    \begin{enumerate}[(1)]
%    \item
%        Is there any interest in predicting the range of
%        plausible values for individual observations?
%        If yes, then normality is important.
%    \item
%        Are there very few observations, such as fewer than~30?
%        If yes, then normality is important.
%    \end{enumerate}
%    If the answer is \emph{no} to each of these questions,
%    then
%    However, this condition can be taken with a grain of salt
%    when primarily focused on the trend of the data.
%    When the data's trend is the focus,
%    the number of observations can be modest in number,
%    such as 30 or more, at which point this condition
%    can be somewhat relaxed.
%    Generally, it is important to look for outliers,
    which we'll talk about more in
    Sections~\ref{typesOfOutliersInLinearRegression}.
    An example of a residual that would be a potentially
    concern is shown in
    Figure~\ref{whatCanGoWrongWithLinearModel},
    where one observation is clearly much further from the
    regression line than the others.
\item[Constant variability.]
    The variability of points around the least squares line
    remains roughly constant.
    An example of non-constant variability is shown in the
    third panel of Figure~\ref{whatCanGoWrongWithLinearModel},
    which represents the most common pattern observed
    when this condition fails:
    the variability of $y$ is larger when $x$ is larger.
\item[Independent observations.]
    Be cautious about applying regression to \term{time series}
    data, which are sequential observations in time such as a
    stock price each day.
    Such data may have an underlying structure that should
    be considered in a model and analysis.
    An example of a data set where successive observations
    are not independent is shown in the fourth panel of
    Figure~\ref{whatCanGoWrongWithLinearModel}.
    There are also other instances where correlations within
    the data are important, which is further discussed in
    Chapter~\ref{ch_regr_mult_and_log}.
\end{description}

\begin{figure}[h]
  \centering
  \Figure[Four scatterplots are shown, each with their own residual plot. These four examples show when methods in this chapter are insufficient to apply to the data. In the first set, a scatterplot with arch-shaped data is shown with a straight line fit to the data, which poorly fits the curved nature of the data; this is meant to highlight an example where "linearity" fails. In the second set, a set of data with a line fit is shown, where the data tightly pack around the line, except one point in particular that is far from the line and represents the case where there are "extreme outliers" in the data. The third set shows a case where a straight line fits the data, but the variability around the line changes, where observations tend to be quite close to the line on the left, but when looking further right, the observations tend to be increasingly far from the line, indicating "changing variability" in the residuals over different regions of the plot. The fourth set provides another case of what is called "time series" data, which is a context where "successive observations are correlated".]{}{whatCanGoWrongWithLinearModel}
  \caption{Four examples showing when the methods in this
      chapter are insufficient to apply to the data.
      First panel: linearity fails.
      Second panel: there are outliers, most especially
      one point that is very far away from the line.
      Third panel: the variability of the errors is related
      to the value of $x$.
      Fourth panel: a time series data set is shown,
      where successive observations are highly correlated.}
  \label{whatCanGoWrongWithLinearModel}
\end{figure}

\begin{exercisewrap}
\begin{nexercise}
Should we have concerns about applying least squares regression to the Elmhurst data in Figure~\ref{elmhurstScatterW2Lines}?\footnotemark
\end{nexercise}
\end{exercisewrap}
\footnotetext{The trend appears to be linear, the data fall around the line with no obvious outliers, the variance is roughly constant. These are also not time series observations. Least squares regression can be applied to these data.}


\D{\newpage}

\subsection{Finding the least squares line}
\label{findingTheLeastSquaresLineSection}

For the Elmhurst data, we could write the equation of the least squares regression line as
\begin{eqnarray*}
\widehat{aid} = \beta_0 + \beta_{1}\times
    \textit{family\us{}income}
\end{eqnarray*}
Here the equation is set up to predict gift aid based on a student's family income, which would be useful to students considering Elmhurst. These two values, $\beta_0$ and $\beta_1$, are the parameters\index{parameter} of the regression line.

As in
Chapters~\ref{ch_foundations_for_inf},
\ref{ch_inference_for_props},
and~\ref{ch_inference_for_means},
the parameters are estimated using observed data. In practice, this estimation is done using a computer in the same way that other estimates, like a sample mean, can be estimated using a computer or calculator. However, we can also find the parameter estimates by applying two properties of the least squares line:
\begin{itemize}
\item
    The slope of the least squares line can be estimated by
    \begin{align*}
    b_1 = \frac{s_y}{s_x} R
    \end{align*}
    where $R$ is the correlation between the two variables,
    and $s_x$ and $s_y$ are the sample standard deviations
    of the explanatory variable and response, respectively.
\item
    If $\bar{x}$ is the sample mean of the explanatory variable
    and $\bar{y}$ is the sample mean of the vertical variable,
    then the point $(\bar{x}, \bar{y})$ is on the least squares
    line.

    Figure~\ref{summaryStatsElmhurstRegr} shows the sample means
    for the family income and gift aid as \$101,780 and \$19,940,
    respectively.
    We could plot the point $(101.8, 19.94)$ on
    Figure~\vref{elmhurstScatterW2Lines}
    to verify it falls on the least squares line (the solid line).
%     and from the point-slope formula, we can identify $b_0$:
%    \begin{align*}
%    \hat{y} - \bar{y} = b_1 (x - \bar{x})
%    \qquad \to \qquad
%    \hat{y} = (\bar{y} - b_1 \bar{x}) + b_1 x
%    \end{align*}
%    This is the point-slope form of a line,
%    where $b_0 = \bar{y} - b_1 \bar{x}$.
\end{itemize}
Next, we formally find the point estimates $b_0$ and $b_1$
of the parameters $\beta_0$ and $\beta_1$.


\begin{figure}[ht]
\centering
\begin{tabular}{l rr}
\hline
\vspace{-4mm} & & \\
\vspace{0.4mm}	&	\ \ Family Income ($x$)	&
    \ \ Gift Aid ($y$) \\
\hline
  \vspace{-3.9mm} & & \\
mean & $\bar{x} = \text{\$101,780}$ &
    $\bar{y} = \text{\$19,940}$ \\
sd & $s_x = \text{\$63,200}$ &
    $s_y = \text{\$5,460}$ \vspace{0.4mm} \\
\hline
\vspace{-4mm}\ &\\
	& \multicolumn{2}{r}{$R=-0.499$} \\
\hline
\end{tabular}
\caption{Summary statistics for family income and gift aid.}
\label{summaryStatsElmhurstRegr}
\end{figure}

\D{\newpage}

\begin{exercisewrap}
\begin{nexercise} \label{findingTheSlopeOfTheLSRLineForIncomeAndAid}
Using the summary statistics in Figure~\ref{summaryStatsElmhurstRegr}, compute the slope for the regression line of gift aid against family income.\footnotemark
\end{nexercise}
\end{exercisewrap}
\footnotetext{Compute the slope using the summary statistics from Figure~\ref{summaryStatsElmhurstRegr}:
\begin{eqnarray*}
b_1
  = \frac{s_y}{s_x} R
  = \frac{\text{5,460}}{\text{63,200}}(-0.499)
  = -0.0431
\end{eqnarray*}}

You might recall the \term{point-slope} form of a line
from math class, which we can use to find the model fit,
including the estimate of $b_0$.
Given the slope of a line and a point on the line,
$(x_0, y_0)$, the equation for the line can be written as
\begin{align*}
y - y_0 = slope\times (x - x_0)
\end{align*}
%We could plug in $(\bar{x}, \bar{y})$ in for $(x_0, y_0$ and solve for $\hat{y}$ to arrive at the model.
%A common exercise to become more familiar with foundations of least squares regression is to use basic summary statistics and point-slope form to produce the least squares line. 

\begin{onebox}{Identifying the least squares line from summary statistics}
To identify the least squares line from summary statistics:\vspace{-1mm}
\begin{itemize}
\setlength{\itemsep}{0mm}
\item
    Estimate the slope parameter, $b_1 = (s_y / s_x) R$.
\item
    Noting that the point $(\bar{x}, \bar{y})$ is on the least
    squares line, use $x_0 = \bar{x}$ and $y_0 = \bar{y}$ with
    the point-slope equation: $y - \bar{y} = b_1 (x - \bar{x})$.
\item
    Simplify the equation, which would reveal that
    $b_0 = \bar{y} - b_1 \bar{x}$.
\end{itemize}
\end{onebox}

\begin{examplewrap}
\begin{nexample}{Using the point $(101780, 19940)$
    from the sample means and the slope estimate
    $b_1 = -0.0431$ from Guided
    Practice~\ref{findingTheSlopeOfTheLSRLineForIncomeAndAid},
    find the least-squares line for predicting aid based
    on family income.}
  \label{exampleToFindLSRLineOfElmhurstData}%
  Apply the point-slope equation using $(101.78, 19.94)$
  and the slope $b_1 = -0.0431$:
  \begin{align*}
  y - y_0    &= b_1 (x - x_0) \\
  y - \text{19,940}  &= -0.0431(x - \text{101,780})
  \end{align*}
  Expanding the right side and then adding 19,940 to each side,
  the equation simplifies:
  \begin{align*}
  \widehat{aid} = \text{24,327} - 0.0431 \times
      \textit{family\us{}income}
  \end{align*}
  Here we have replaced $y$ with $\widehat{aid}$ and $x$ with
  \textit{family\us{}income} to put the equation in context.
  The final equation should always include a ``hat''
  on the variable being predicted, whether it is a generic
  ``$y$'' or a named variable like ``$aid$''.
\end{nexample}
\end{examplewrap}

A computer is usually used to compute the least squares line,
and a summary table generated using software for the Elmhurst
regression line is shown in
Figure~\ref{rOutputForIncomeAidLSRLine}.
The first column of numbers provides estimates for ${b}_0$
and ${b}_1$, respectively.
These results match those from
Example~\ref{exampleToFindLSRLineOfElmhurstData}
(with some minor rounding error).

\begin{figure}[ht]
\centering
\begin{tabular}{l rrrr}
  \hline
  \vspace{-3.7mm} & & & & \\
 & Estimate & Std. Error & t value & Pr($>$$|$t$|$) \\ 
  \hline
  \vspace{-3.6mm} & & & & \\
(Intercept) & 24319.3 & 1291.5 & 18.83 & $<$0.0001 \\ 
family\us{}income & -0.0431 & 0.0108 & -3.98 & 0.0002 \\ 
  \hline
\end{tabular}
\caption{Summary of least squares fit for the Elmhurst data.
    Compare the parameter estimates in the first column to
    the results of
    Example~\ref{exampleToFindLSRLineOfElmhurstData}.}
\label{rOutputForIncomeAidLSRLine}
\end{figure}

\D{\newpage}

\begin{examplewrap}
\begin{nexample}{Examine the second, third, and fourth columns
    in Figure~\ref{rOutputForIncomeAidLSRLine}.
    Can you guess what they represent?
    (If you have not reviewed any inference chapter yet,
    skip this example.)}
  We'll describe the meaning of the columns using the
  second row, which corresponds to~$\beta_1$.
  The first column provides the point estimate for $\beta_1$,
  as we calculated in an earlier example: $b_1 = -0.0431$.
  The second column is a standard error for this point estimate:
  $SE_{b_1} = 0.0108$.
  The third column is a $t$-test statistic for the null
  hypothesis that $\beta_1 = 0$: $T = -3.98$.
  The last column is the p-value for the $t$-test statistic
  for the null hypothesis $\beta_1 = 0$ and a two-sided
  alternative hypothesis: 0.0002.
  We will get into more of these details in
  Section~\ref{inferenceForLinearRegression}.
\end{nexample}
\end{examplewrap}

\begin{examplewrap}
\begin{nexample}{Suppose a high school senior is considering
    Elmhurst College.
    Can she simply use the linear equation that we have estimated
    to calculate her financial aid from the university?}
  She may use it as an estimate, though some qualifiers on this
  approach are important.
  First, the data all come from one freshman class,
  and the way aid is determined by the university may change
  from year to year.
  Second, the equation will provide an imperfect estimate.
  While the linear equation is good at capturing the trend
  in the data, no individual student's aid will be perfectly
  predicted.
\end{nexample}
\end{examplewrap} 

\index{least squares regression|)}


\subsection{Interpreting regression model parameter estimates}

\index{least squares regression!interpreting parameters|(}

\noindent%
Interpreting parameters in a regression model is often one
of the most important steps in the analysis.

\begin{examplewrap}
\begin{nexample}{The intercept and slope estimates for
    the Elmhurst data are $b_0 = \text{24,319}$
    and $b_1 = -0.0431$.
    What do these numbers really mean?}
  Interpreting the slope parameter is helpful in almost any
  application.
  For each additional \$1,000 of family income, we would expect
  a student to receive a net difference of
  $\$\text{1,000}\times (-0.0431) = -\$43.10$ in aid on average,
  i.e. \$43.10 \emph{less}.
  Note that a higher family income corresponds to less aid
  because the coefficient of family income is negative in
  the model.
  We must be cautious in this interpretation:
  while there is a real association, we cannot interpret
  a causal connection between the variables because these
  data are observational.
  That is, increasing a student's family income may not
  cause the student's aid to drop.
  (It would be reasonable to contact the college and ask
  if the relationship is causal,
  i.e. if Elmhurst College's aid decisions are partially
  based on students' family income.)

  The estimated intercept $b_0 = \text{24,319}$
  describes the average aid if a student's family had no income.
  The meaning of the intercept is relevant to this application
  since the family income for some students at Elmhurst is~\$0.
  In other applications, the intercept may have little
  or no practical value if there are no observations where
  $x$ is near zero.
\end{nexample}
\end{examplewrap}

\begin{onebox}{Interpreting parameters estimated by least squares}
  The slope describes the estimated difference in the
  $y$ variable if the explanatory variable $x$ for a case
  happened to be one unit larger.
  The intercept describes the average outcome of $y$ if $x=0$
  \emph{and} the linear model is valid all the way to $x=0$,
  which in many applications is not the case.
\end{onebox}

\index{least squares regression!interpreting parameters|)}


\D{\newpage}

\subsection{Extrapolation is treacherous}

\index{least squares regression!extrapolation|(}

{\em\small When those blizzards hit the East Coast this winter,
it proved to my satisfaction that global warming was a fraud.
That snow was freezing cold.
But in an alarming trend, temperatures this spring have risen.
Consider this: On February $6^{th}$ it was 10 degrees.
Today it hit almost 80. At this rate, by August it will be
220 degrees.
So clearly folks the climate debate rages on.\vspace{0.5mm}}

\noindent\hspace{\textwidth}\hspace{-40mm}Stephen Colbert

\noindent\hspace{\textwidth}\hspace{-40mm}April 6th,
2010\footnote{\oiRedirect{textbook-colbert_extrapolation}
    {www.cc.com/video-clips/l4nkoq}} \\

Linear models can be used to approximate the relationship
between two variables.
However, these models have real limitations.
Linear regression is simply a modeling framework.
The truth is almost always much more complex than our simple line.
For example, we do not know how the data outside of our limited
window will behave.

\begin{examplewrap}
\begin{nexample}{Use the model
    $\widehat{aid}
      = \text{24,319} - 0.0431 \times
          \textit{family\us{}income}$
    to estimate the aid of another freshman student whose
    family had income of \$1~million.}
  We want to calculate the aid for
  $\textit{family\us{}income} = \text{1,000,000}$:
  \begin{align*}
  \text{24,319} - 0.0431\times \textit{family\us{}income}
    = \text{24,319} - 0.0431\times \text{1,000,000}
    = -\text{18,781}
  \end{align*}
  The model predicts this student will have -\$18,781 in aid (!).
  However, Elmhurst College does not offer \emph{negative aid}
  where they select some students to pay extra on top of tuition
  to attend.
\end{nexample}
\end{examplewrap}

Applying a model estimate to values outside of the realm of the
original data is called \term{extrapolation}.
Generally, a linear model is only an approximation of the real
relationship between two variables.
If we extrapolate, we are making an unreliable bet that the
approximate linear relationship will be valid in places where
it has not been analyzed.

\index{least squares regression!extrapolation|)}


\subsection{Using $R^2$ to describe the strength of a fit}

\index{least squares regression!R-squared ($R^2$)|(}

We evaluated the strength of the linear relationship between
two variables earlier using the correlation, $R$.
However, it is more common to explain the strength of a linear
fit using $R^2$, called
\termsub{R-squared}{least squares regression!R-squared ($R^2$)}.
\index{R-squared ($R^2$)|textbf}
If provided with a linear model, we might like to describe how
closely the data cluster around the linear fit.

\begin{figure}[h]
  \centering
  \Figures[A scatterplot of the Elmhurst data is shown for gift aid and family income with the least squares regression line overlaid against the data, which has a slight downward trend.]{0.7}{elmhurstPlots}{elmhurstScatterWLSROnly}
  \caption{Gift aid and family income for a random sample
      of 50 freshman students from Elmhurst College, shown
      with the least squares regression line.}
  \label{elmhurstScatterWLSROnly}
\end{figure}

\newcommand{\mil}[0]{\text{ million}}
The $R^2$ of a linear model describes the amount of variation
in the response that is explained by the least squares line.
For example, consider the Elmhurst data,
shown in Figure~\ref{elmhurstScatterWLSROnly}.
The variance of the response variable, aid received,
is about $s_{aid}^2 \approx 29.8$ million.
However, if we apply our least squares line, then this model
reduces our uncertainty in predicting aid using a student's
family income.
The variability in the residuals describes how much variation
remains after using the model: $s_{_{RES}}^2 \approx 22.4$ million.
In short, there was a reduction of
\begin{align*}
\frac{s_{aid}^2 - s_{_{RES}}^2}{s_{aid}^2}
  = \frac{\text{29,800,000} - \text{22,400,000}}
      {\text{29,800,000}}
  = \frac{\text{7,500,000}}{\text{29,800,000}}
  = 0.25
\end{align*}
or about  25\% in the data's variation by using information
about family income for predicting aid using a linear model.
This corresponds exactly to the R-squared value:
\begin{align*}
R &= -0.499 &R^2 &= 0.25
\end{align*}

\begin{exercisewrap}
\begin{nexercise}
If a linear model has a very strong negative relationship with
a correlation of -0.97, how much of the variation in the response
is explained by the explanatory variable?\footnotemark{}
\end{nexercise}
\end{exercisewrap}
\footnotetext{About $R^2 = (-0.97)^2 = 0.94$ or 94\%
  of the variation is explained by the linear model.}

\index{least squares regression!R-squared ($R^2$)|)}


\subsection{Categorical predictors with two levels}
\label{categoricalPredictorsWithTwoLevels}

Categorical variables are also useful in predicting outcomes.
Here we consider a categorical predictor with two levels
(recall that a \emph{level} is the same as a \emph{category}).
We'll consider Ebay auctions for a video game, \emph{Mario Kart}
for the Nintendo Wii, where both the total price of the auction
and the condition of the game were recorded.
Here we want to predict total price based on game condition,
which takes values \resp{used} and \resp{new}.
A plot of the auction data is shown in Figure~\ref{marioKartNewUsed}.

\begin{figure}[h]
  \centering
  \Figure[A scatterplot is shown for total auction prices for the video game "Mario Kart", broken down by condition on the horizontal axis. The prices are divided into "used" and "new" condition groups. All used games are shown with an x-value of 0 on the left, and all new games are shown with an x-value of 1 on the right of the plot. The used games on the left show a lower average price of about \$43, and new games on the right show a higher average price of about \$54. The least squares regression line is also shown for this scatterplot, which shows an upward trend and has a formula of "price equals 42.87 plus 10.90 times cond-subscript-new.]{0.6}{marioKartNewUsed}
  \caption{Total auction prices for the video game
      \emph{Mario Kart}, divided into used ($x=0$)
      and new ($x=1$) condition games.
      The least squares regression line is also shown.}
  \label{marioKartNewUsed}
\end{figure}

To incorporate the game condition variable into a regression
equation, we must convert the categories into a numerical form.
We will do so using an \term{indicator variable}
called \var{cond\us{}new}, which takes value 1 when the game
is new and 0 when the game is used.
Using this indicator variable, the linear model may be written as
\begin{align*}
\widehat{price} = \beta_0 + \beta_1 \times \text{\var{cond\us{}new}}
\end{align*}
The parameter estimates are given in
Figure~\ref{marioKartNewUsedRegrSummary},
and the model equation can be summarized as
\begin{align*}
\widehat{price} = 42.87 + 10.90 \times \text{\var{cond\us{}new}}
\end{align*}
For categorical predictors with just two levels,
the linearity assumption will always be satisfied.
However, we must evaluate whether the residuals in
each group are approximately normal and have approximately
equal variance.
As can be seen in Figure~\ref{marioKartNewUsed},
both of these conditions are reasonably satisfied
by the auction data.

\begin{figure}
\centering
\begin{tabular}{rrrrr}
  \hline
  \vspace{-3.7mm} & & & & \\
 & Estimate & Std. Error & t value & Pr($>$$|$t$|$) \\ 
  \hline
  \vspace{-3.6mm} & & & & \\
(Intercept) & 42.87 & 0.81 & 52.67 & $<$0.0001 \\ 
  cond\us{}new & 10.90 & 1.26 & 8.66 & $<$0.0001 \\ 
   \hline
\end{tabular}
\caption{Least squares regression summary for the final auction price against the condition of the game.}
\label{marioKartNewUsedRegrSummary}
\end{figure}

\begin{examplewrap}
\begin{nexample}{Interpret the two parameters estimated in the
    model for the price of \emph{Mario Kart} in eBay auctions.}
  The intercept is the estimated price when \var{cond\us{}new}
  takes value 0, i.e. when the game is in used condition.
  That is, the average selling price of a used version of
  the game is \$42.87.

  The slope indicates that, on average, new games sell for
  about \$10.90 more than used games.
\end{nexample}
\end{examplewrap}

\begin{onebox}{Interpreting model estimates for
    categorical predictors}
  The estimated intercept is the value of the response variable
  for the first category (i.e. the category corresponding to an
  indicator value of~0).
  The estimated slope is the average change in the response
  variable between the two categories.
\end{onebox}

We'll elaborate further on this topic in
Chapter~\ref{ch_regr_mult_and_log},
where we examine the influence of many
predictor variables simultaneously using
multiple regression.


{\exercisesheader{}

% 17

\eoce{\qt{Units of regression\label{regression_units}} Consider a regression 
predicting weight (kg) from height (cm) for a sample of adult males. 
What are the units of the correlation coefficient, the intercept, 
and the slope?
}{}

% 18

\eoce{\qtq{Which is higher\label{which_higher_scatter}} Determine if I or II 
is higher or if they are equal. Explain your reasoning.
\noindent For a regression line, the uncertainty associated with the 
slope estimate, $b_1$, is higher when
\begin{enumerate}
\item[I.] there is a lot of scatter around the regression line or
\item[II.] there is very little scatter around the regression line
\end{enumerate}
}{}

% 19

\eoce{\qt{Over-under, Part I\label{residual_apple_weight}} Suppose we fit a 
regression line to predict the shelf life of an apple based on its weight. 
For a particular apple, we predict the shelf life to be 4.6 days. The 
apple's residual is -0.6 days. Did we over or under estimate the 
shelf-life of the apple? Explain your reasoning.
}{}

% 20

\eoce{\qt{Over-under, Part II\label{residual_sun_cancer}} Suppose we fit a 
regression line to predict the number of incidents of skin cancer per 
1,000 people from the number of sunny days in a year. For a particular 
year, we predict the incidence of skin cancer to be 1.5 per 1,000 people, 
and the residual for this year is 0.5. Did we over or under estimate 
the incidence of skin cancer? Explain your reasoning.
}{}

% 21

\eoce{\qt{Tourism spending\label{tourism_spending_reg_conds}} The Association of 
Turkish Travel Agencies reports the number of foreign tourists 
visiting Turkey and tourist spending by year.
\footfullcite{data:turkeyTourism} Three plots are provided: 
scatterplot showing the relationship between these two variables 
along with the least squares fit, residuals plot, and histogram of 
residuals.
\begin{center}
\FigureFullPath[A scatterplot with a least squares regression line is fit based on about 50 points. The horizontal axis represents "Number of tourists" and has values ranging from about 0 to about 27 million. The vertical axis represents "Spending, in US dollars", with values ranging from about \$0 to about \$17 billion. There are many points shown with the number of tourists between 0 and 5 million, which has spending between about \$0 and \$3 billion, where even on this small scale a roughly linear trend is evident. The linear trend continues on across the plot and is quite strong -- where residuals generally do not deviate from the least square line by more than very roughly \$1 billion. The data are also more sparse for larger values in the plot. There is one region in the center of the plot where about 10 points in a row lie above the regression line. Also consider the next two plots before answering any questions for this exercise.]{0.32}{ch_regr_simple_linear/figures/eoce/tourism_spending_reg_conds/tourism_spending_count}
\FigureFullPath[A residual plot is shown. The horizontal axis represents "Number of tourists" and has values ranging from about 0 to about 27 million. Residuals are shown on the vertical axis and have values ranging from about -\$1.5 billion to about \$1.2 billion. The points on the far left between 0 and 3 million tourists shows a "v" pattern. There are about 15 points with number of tourists between 3 million and 10 million, which shows an slight upward trend from about -\$700 million to \$1.2 billion. There about 10 points with number of tourists greater than 10 million up to about 27 million, and these show a slight downward trend from about \$1 billion to -\$1.5 billion.]{0.32}{ch_regr_simple_linear/figures/eoce/tourism_spending_reg_conds/tourism_spending_count_residuals}
\FigureFullPath[A histogram is shown for the residuals, which shows a roughly bell-shaped distribution centered at 0 and a standard deviation of about \$500 million.]{0.32}{ch_regr_simple_linear/figures/eoce/tourism_spending_reg_conds/tourism_spending_count_residuals_hist}
\end{center}
\begin{parts}
\item Describe the relationship between number of tourists and spending.
\item What are the explanatory and response variables?
\item Why might we want to fit a regression line to these data?
\item Do the data meet the conditions required for fitting a least squares 
line? In addition to the scatterplot, use the residual plot and histogram 
to answer this question. 
\end{parts}
}{}

\D{\newpage}

% 22

\eoce{\qt{Nutrition at Starbucks, Part I\label{starbucks_cals_carbos}} 
The scatterplot below shows the relationship between the number of 
calories and amount of carbohydrates (in grams) Starbucks food menu 
items contain.\footfullcite{data:starbucksCals} Since Starbucks only 
lists the number of calories on the display items, we are interested 
in predicting the amount of carbs a menu item has based on its 
calorie content.
\begin{center}
\FigureFullPath[A scatterplot is shown with about 75 points and an overlaid regression line that trends upward. The horizontal axis represents "Calories" and has values ranging from about 100 to 500. The vertical axis represents "Carbs, in grams" and has values ranging from about 20 to 80. About 15 points are shown with fewer than 200 calories, and these have between about 18 and 25 grams of carbs. About 30 points are shown with 200 to 400 calories, and these mostly have between 30 and 60 grams of carbs. About 20 points are shown with more than 400 calories, and these mostly have between 35 and 80 grams of carbs.]{0.32}{ch_regr_simple_linear/figures/eoce/starbucks_cals_carbos/starbucks_cals_carbos}
\FigureFullPath[A residual plot is shown with about 75 points. The horizontal axis represents "Calories" and has values ranging from about 100 to 500. The vertical axis represents "Residuals" and has values ranging from about -30 to 30. About 15 points are shown with fewer than 200 calories, and these have residuals roughly between -7 and positive 2. About 30 points are shown with 200 to 400 calories, and these residuals largely range from about -15 to positive 15. About 20 points are shown with more than 400 calories, and the residuals for these points mostly range between -20 and positive 20.]{0.32}{ch_regr_simple_linear/figures/eoce/starbucks_cals_carbos/starbucks_cals_carbos_residuals}
\FigureFullPath[A histogram is shown for the residuals, which shows a roughly bell-shaped distribution centered at 0 and a standard deviation of about 10.]{0.32}{ch_regr_simple_linear/figures/eoce/starbucks_cals_carbos/starbucks_cals_carbos_residuals_hist}
\end{center}
\begin{parts}
\item Describe the relationship between number of calories and amount 
of carbohydrates (in grams) that Starbucks food menu items contain.
\item In this scenario, what are the explanatory and response 
variables?
\item Why might we want to fit a regression line to these data?
\item Do these data meet the conditions required for fitting a least 
squares line?
\end{parts}
}{}

% 23

\eoce{\qt{The Coast Starlight, Part II\label{coast_starlight_reg}}
Exercise~\ref{coast_starlight_corr_units} introduces data on the Coast Starlight 
Amtrak train that runs from Seattle to Los Angeles. The mean travel 
time from one stop to the next on the Coast Starlight is 129 mins, 
with a standard deviation of 113 minutes. The mean distance traveled 
from one stop to the next is 108 miles with a standard deviation of 
99 miles. The correlation between travel time and distance is 0.636.
\begin{parts}
\item Write the equation of the regression line for predicting travel 
time.
\item Interpret the slope and the intercept in this context.
\item Calculate $R^2$ of the regression line for predicting travel 
time from distance traveled for the Coast Starlight, and interpret 
$R^2$ in the context of the application.
\item The distance between Santa Barbara and Los Angeles is 103 
miles. Use the model to estimate the time it takes for the Starlight 
to travel between these two cities.
\item It actually takes the Coast Starlight about 168 mins to travel 
from Santa Barbara to Los Angeles. Calculate the residual and explain 
the meaning of this residual value.
\item Suppose Amtrak is considering adding a stop to the Coast 
Starlight 500 miles away from Los Angeles. Would it be appropriate to 
use this linear model to predict the travel time from Los Angeles to 
this point? 
\end{parts}
}{}

% 24

\eoce{\qt{Body measurements, Part III\label{body_measurements_shoulder_height_reg}}
Exercise~\ref{body_measurements_shoulder_height_corr_units} introduces 
data on shoulder girth and height of a group of individuals. The 
mean shoulder girth is 107.20 cm with a standard deviation of 
10.37 cm. The mean height is 171.14 cm with a standard deviation 
of 9.41 cm. The correlation between height and shoulder girth is 0.67.
\begin{parts}
\item Write the equation of the regression line for predicting height.
\item Interpret the slope and the intercept in this context.
\item Calculate $R^2$ of the regression line for predicting height 
from shoulder girth, and interpret it in the context of the 
application.
\item A randomly selected student from your class has a shoulder 
girth of 100 cm. Predict the height of this student using the model.
\item The student from part~(d) is 160 cm tall. Calculate the 
residual, and explain what this residual means.
\item A one year old has a shoulder girth of 56 cm. Would it be 
appropriate to use this linear model to predict the height of this 
child?
\end{parts}
}{}

\D{\newpage}

% 25

\eoce{\qt{Murders and poverty, Part I\label{murders_poverty_reg}} The following 
regression output is for predicting annual murders per million from 
percentage living in poverty in a random sample of 20 metropolitan 
areas.\\[2mm]
\begin{minipage}[c]{0.54\textwidth}
{\footnotesize
\begin{tabular}{rrrrr}
    \hline
            & Estimate  & Std. Error    & t value   & Pr($>$$|$t$|$) \\ 
    \hline
(Intercept) & -29.901   & 7.789         & -3.839    & 0.001 \\ 
poverty\%   & 2.559     & 0.390         & 6.562     & 0.000 \\ 
   \hline
\end{tabular} \\
$s = 5.512 \hfill R^2 = 70.52\% \hfill R^2_{adj} = 68.89\%$ 
}
\begin{parts}
\item Write out the linear model.
\item Interpret the intercept.
\item Interpret the slope.
\item Interpret $R^2$.
\item Calculate the correlation coefficient.
\end{parts}
\end{minipage}
\begin{minipage}[c]{0.02\textwidth}
$\:$\\
\end{minipage}
\begin{minipage}[c]{0.41\textwidth}
\FigureFullPath[A scatterplot is shown with 20 points. The horizontal axis is "Percent in Poverty" and has values ranging from 14\% to 26\%. The vertical axis is "Annual Murders per Million" with values ranging from about 5 to 40. There are 6 points with poverty below 18\%, and the Murder Rate for these values ranges from 5 to 13, with one exception of a point at about 17\% with a murder rate of about 25. There are 9 points with a poverty rate of 18\% to 22\%, and the murder rate for these points largely range from 14 to 25, with one exception of a point at about 21\% poverty and a murder rate of 35. There are 5 points where poverty is larger than 22\%, and these have murder rates ranging from 25 to 40.]{}{ch_regr_simple_linear/figures/eoce/murders_poverty_reg/murders_poverty.pdf}
\end{minipage}
}{}

% 26

\eoce{\qt{Cats, Part I\label{cat_body_heart_reg}} The following regression output is 
for predicting the heart weight (in g) of cats from their body weight 
(in kg). The coefficients are estimated using a dataset of 144 
domestic cats.\\[2mm]
\begin{minipage}[c]{0.54\textwidth}
{\footnotesize
\begin{tabular}{rrrrr}
    \hline
            & Estimate  & Std. Error    & t value   & Pr($>$$|$t$|$) \\ 
    \hline
(Intercept) & -0.357    & 0.692         & -0.515    & 0.607 \\ 
body wt     & 4.034     & 0.250         & 16.119    & 0.000 \\ 
    \hline
\end{tabular} \\
$s = 1.452 \hfill R^2 = 64.66\% \hfill R^2_{adj} = 64.41\%$ 
}
\begin{parts}
\item Write out the linear model.
\item Interpret the intercept.
\item Interpret the slope.
\item Interpret $R^2$.
\item Calculate the correlation coefficient.
\end{parts}
\end{minipage}
\begin{minipage}[c]{0.02\textwidth}
$\:$\\
\end{minipage}
\begin{minipage}[c]{0.41\textwidth}
\FigureFullPath[A scatterplot is shown with about 150 points. The horizontal axis is "Body weight, in kilograms" and has values ranging from 2 to 4. The vertical axis is "Heart weight, in grams" with values ranging from about 5 to 20. About 25\% of the data has a body weight below 2.5 kilograms, and these have heart weights mostly ranging from 7  to 11 grams. About 35\% of the data has body weights between 2.5 and 3 kilograms, and the heart weight for these values mostly ranges from 8 to 12 grams. About 30\% of the data has body weights between 3 and 3.5 kilograms, and the heart weight for these values mostly ranges from 11 to 15 grams. About 10\% of the data has body weights above 3.5 kilograms, and the heart weight for these values mostly ranges from 12 to 17 grams.]{}{ch_regr_simple_linear/figures/eoce/cat_body_heart_reg/cat_body_heart.pdf}
\end{minipage}
}{}
}








%__________________
\section{Types of outliers in linear regression}
\label{typesOfOutliersInLinearRegression}

In this section, we identify criteria for determining which
outliers are important and influential.
Outliers in regression are observations that fall far from
the cloud of points.
These points are especially important because they can have
a strong influence on the least squares line. 

\begin{examplewrap}
\begin{nexample}{There are six plots shown in
    Figure~\ref{outlierPlots} along with the least squares
    line and residual plots.
    For~each scatterplot and residual plot pair,
    identify the outliers and note how they influence
    the least squares line.
    Recall that an outlier is any point that doesn't appear
    to belong with the vast majority of the other points.}
  \label{outlierPlotsExample}%
  \begin{itemize}
  %\setlength{\itemsep}{0mm}
  \item[(1)]
      There is one outlier far from the other points,
      though it only appears to slightly influence the~line.
  \item[(2)]
      There is one outlier on the right, though it is quite
      close to the least squares line, which suggests it
      wasn't very influential.
  \item[(3)]
      There is one point far away from the cloud, and this
      outlier appears to pull the least squares line up on
      the right;
      examine how the line around the primary cloud doesn't
      appear to fit very~well.
  \item[(4)]
      There is a primary cloud and then a small secondary
      cloud of four outliers.
      The secondary cloud appears to be influencing the line
      somewhat strongly, making the least square line fit
      poorly almost everywhere.
      There might be an interesting explanation for the dual
      clouds, which is something that could be investigated.
  \item[(5)]
      There is no obvious trend in the main cloud of points
      and the outlier on the right appears to largely control
      the slope of the least squares line.
  \item[(6)]
      There is one outlier far from the cloud.
      However, it falls quite close to the least squares line
      and does not appear to be very influential.
  \end{itemize}
\end{nexample}
\end{examplewrap}

\begin{figure}
  \centering
  \Figure[Six scatterplots, each with a least squares line and residual plot. All data sets have at least one outlier. (1) A clear positive upward trend is evident in the points with a regression line overlaying these points, but one point is shown deviating substantially from the line about one-third of the way from the left side of the plot and far below the other points. (2) A slight downward trend is evident in the points on the left half of the plot with a regression line overlaying these points and extending to a single point on the far right of the plot that is also very close to the regression line. (3) A positive upward trend is evident for points shown on the left two-thirds of the plot with a regression line overlaying these points, but a single point is shown on the far right and lying substantially above the line. This one point appears to be "pulling" the regression line up on the right, making the line fit the rest of the data less well. (4) Most of the data is shown in the left two-thirds of the plot with a clear downward, linear trend. A cluster of 4 points is shown on the far right but deviating notably above the trend of the other points. The regression line fit to the data shows it largely "trying" to fit the bulk of the data on the left but being "pulled" upward on the right towards the cluster of points deviating from the linear trend. (5) A large cluster of points is shown on the far bottom-left, and there is no apparent trend in this large cluster. A single point is shown on the far upper-right. A regression line is fit to the data with a line extending from the cluster on the bottom-left and trending upwards near the single point on the upper right. (6) A clear downward trend is evident in the points on the right two-thirds of the plot with a regression line overlaying these points and extending to a single point on the far left of the plot that is also very close to the regression line.]{}{outlierPlots}
  \caption{Six plots, each with a least squares line and
      residual plot. All data sets have at least one outlier.}
\label{outlierPlots}
\end{figure}

Examine the residual plots in Figure~\ref{outlierPlots}.
You will probably find that there is some trend in the main
clouds of~(3) and~(4).
In these cases, the outliers influenced the slope of the
least squares lines.
In~(5), data with no clear trend were assigned a line with
a large trend simply due to one outlier (!).
 
\begin{onebox}{Leverage}
  Points that fall horizontally away from the center of the
  cloud tend to pull harder on the line, so we call them points
  with \term{high leverage}.\index{leverage}
\end{onebox}

Points that fall horizontally far from the line are points
of high leverage;
these points can strongly influence the slope of the least
squares line.
If one of these high leverage points does appear to actually
invoke its influence on the slope of the line --
as in cases~(3), (4), and (5) of Example~\ref{outlierPlotsExample}
-- then we call it an \term{influential point}.
Usually we can say a point is influential if, had we fitted
the line without it, the influential point would have been
unusually far from the least squares line.

It is tempting to remove outliers.
Don't do this without a very good reason.
Models that ignore exceptional (and interesting) cases often
perform poorly.
For instance, if a financial firm ignored the largest market
swings -- the ``outliers'' --  they would soon go bankrupt
by making poorly thought-out investments.


{\exercisesheader{}

% 27

\eoce{\qt{Outliers, Part I\label{outliers_1}} Identify the outliers in the 
scatterplots shown below, and determine what type of outliers they are. 
Explain your reasoning.
\begin{center}
\FigureFullPath[Most of the data is shown in the left third of the plot with a clear downward, linear trend extending from from the upper-left corner of the plot and to the bottom of the plot only a third of the way from the left side of the plot. A single point is shown on the bottom-right of the plot. A regression line is fit to the data, but it does not fit the bulk of the data well: On the furthest left portion, the line is below the points, crosses over the trend of the bulk of the data, then lies above the remainder of the bulk of the data. If it were shown fully, it would extend well below the single point on the bottom-right.]{0.32}{ch_regr_simple_linear/figures/eoce/outliers_1/outliers_1_influential}
\FigureFullPath[A clear downward trend is evident in the points on the left third of the plot with a regression line overlaying these points and extending to a single point on the far bottom right of the plot that is also almost exactly on the regression line.]{0.32}{ch_regr_simple_linear/figures/eoce/outliers_1/outliers_2_leverage}
\FigureFullPath[A downward trend is evident in the bulk of the points with an overlaid regression line. A single point is shown far above the regression line at the center-top of the plot.]{0.32}{ch_regr_simple_linear/figures/eoce/outliers_1/outliers_3_outlier}
\end{center}
}{}

% 28

\eoce{\qt{Outliers, Part II\label{outliers_2}} Identify the outliers in the scatterplots 
shown below and determine what type of outliers they are. Explain 
your reasoning.
\begin{center}
\FigureFullPath[Most of the data is shown in the right half of the plot with a clear upward, linear trend extending from from the bottom-center and extending to the upper-right corner of the plot. A single point is shown on the upper-left of the plot. A regression line is fit to the data, but it does not fit the bulk of the data well: Focusing first on the bulk of points at the bottom center of the plot, the regression line is well above these points, crosses over the trend of the bulk of the data about 25\% from the right of the plot, then lies below the remainder of the bulk of the data in the upper-right. If it were shown fully, the regression line would extend well below the single point on the upper-left.]{0.32}{ch_regr_simple_linear/figures/eoce/outliers_2/outliers_1_influential}
\FigureFullPath[A clear upward trend is evident in the points on the right half of the plot with a regression line approximately overlaying these points and extending towards a single point on the far bottom left of the plot, but the regression line is notably higher than this single point, which would have by far the largest residual (in absolute value) of all other points shown in the plot. Close inspection of the regression line fit over the bulk of points, it appears to be partially misfitting that data, "pulled" down on the left side.]{0.32}{ch_regr_simple_linear/figures/eoce/outliers_2/outliers_2_influential}
\FigureFullPath[An upper trend is evident in the bulk of the points with an overlaid regression line. A single point is shown far above the regression line at the center-top of the plot.]{0.32}{ch_regr_simple_linear/figures/eoce/outliers_2/outliers_3_outlier}
\end{center}
}{}

% 29

\eoce{\qt{Urban homeowners, Part I\label{urban_homeowners_outlier}} The 
scatterplot below shows the percent of families who own their 
home vs. the percent of the population living in urban areas.
\footfullcite{data:urbanOwner} There are 52 observations, each 
corresponding to a state in the US. Puerto Rico and District of 
Columbia are also included.

\noindent\begin{minipage}[c]{0.5\textwidth}
\begin{parts}
\item Describe the relationship between the percent of families who 
own their home and the percent of the population living in urban areas.
\item The outlier at the bottom right corner is District of Columbia, 
where 100\% of the population is considered urban. What type of an outlier 
is this observation?
\end{parts}
\end{minipage}
\begin{minipage}[c]{0.05\textwidth}
$\:$\\
\end{minipage}
\begin{minipage}[c]{0.4\textwidth}
\FigureFullPath[A scatterplot is shown with about 50 points. The horizontal axis is for "Percent Urban Population" and has values ranging from 40\% to 100\%. The vertical axis is for "Percent Own Their Home" with values ranging from about 40\% to about 75\%. About 10 points have Urban Population with values smaller than 60\%, and these have Homeownership rates between 65\% and 75\%, with most of those points above 70\%. About 20 points have Urban Population with values between 60\% and 70\%, and these have Homeownership rates between 62\% and 75\%. About 20 points have Urban Population with values greater than 70\%, and these have Homeownership rates between 55\% and 73\%, with one exception of a point with 100\% urban population that has a homeownership rate of about 43\%.]{0.95}{ch_regr_simple_linear/figures/eoce/urban_homeowners_outlier/urban_homeowners_outlier} \vspace{-3mm}
\end{minipage}
}{}

% 30

\eoce{\qt{Crawling babies, Part II\label{crawling_babies_outlier}} 
Exercise~\ref{crawling_babies_corr_units} introduces 
data on the average monthly temperature during the month babies first 
try to crawl (about 6 months after birth) and the average first 
crawling age for babies born in a given month. A scatterplot of these 
two variables reveals a potential outlying month when the average 
temperature is about 53\degree F and average crawling age is about 
28.5 weeks. Does this point have high leverage? Is it an influential 
point?
}{}
}








%__________________
\section{Inference for linear regression}
\label{inferenceForLinearRegression}

In this section, we discuss uncertainty in the estimates
of the slope and y-intercept for a regression line.
Just as we identified standard errors for point estimates
in previous chapters, we first discuss standard errors for
these new estimates.


\subsection{Midterm elections and unemployment}

\index{data!midterm elections|(}

Elections for members of the United States House
of Representatives occur every two years, coinciding
every four years with the U.S. Presidential election.
The set of House elections occurring during the middle
of a Presidential term are called
\indexthis{midterm elections}{midterm election}.
In America's two-party system, one political theory
suggests the higher the unemployment rate, the worse
the President's party will do in the midterm elections.

To assess the validity of this claim, we can compile
historical data and look for a connection.
We consider every midterm election from 1898 to 2018,
with the exception of those elections during the Great
Depression.
Figure~\ref{unemploymentAndChangeInHouse} shows these data
and the least-squares regression line: \vspace{-2mm}
\begin{align*}
&\text{\% change in House seats for President's party}  \\
&\qquad\qquad= -7.36 - 0.89 \times \text{(unemployment rate)}
\end{align*}
We consider the percent change in the number of seats
of the President's party (e.g. percent change in the number
of seats for Republicans in 2018) against the unemployment
rate.

Examining the data, there are no clear deviations from
linearity, the constant variance condition,
or substantial outliers.
While the data are collected sequentially, a separate analysis
was used to check for any apparent correlation between successive
observations;
no such correlation was found.

\begin{figure}[h]
  \centering
  \Figure[A scatterplot is shown for the percent change in House seats for the President's party in each midterm election from 1898 to 2018 plotted against the unemployment rate. The two points for the Great Depression have been removed, and a least squares regression line has been fit to the data with a slightly downward trend. The horizontal axis is for "Unemployment Rate" with values ranging from about 3\% to 12\%. The vertical axis is for "Percent Change in Seats of the President's Party in the House of Representatives" with values ranging from about -30\% to positive 10\%. The bulk of the observations have Unemployment Rate between 3\% and 8\%, and these have the percent change in seats ranging from about -27\% to positive 4\% without any discernible trend. There are four observations with unemployment rate above 8\%, and these have the percent change in seats ranging from -25\% to -9\%. Each point in the scatterplot is also labeled as "Democrat" in blue or "Republican" in red, though this doesn't reveal any additional pattern.]{}{unemploymentAndChangeInHouse}
  \caption{The percent change in House seats for the
      President's party in each midterm election from 1898 to 2018
      plotted against the unemployment rate.
      The two points for the Great Depression have been
      removed, and a least squares regression line has been
      fit to the data.}
  \label{unemploymentAndChangeInHouse}
\end{figure}

\begin{exercisewrap}
\begin{nexercise}
The data for the Great Depression (1934 and 1938) were removed
because the unemployment rate was 21\% and 18\%, respectively.
Do you agree that they should be removed for this investigation?
Why or why not?\footnotemark{}
\end{nexercise}
\end{exercisewrap}
\footnotetext{We will provide two considerations.
  Each of these points would have very high leverage on any
  least-squares regression line, and years with such high
  unemployment may not help us understand what would happen
  in other years where the unemployment is only modestly high.
  On the other hand, these are exceptional cases, and we would
  be discarding important information if we exclude them from
  a final analysis.}

There is a negative slope in the line shown in
Figure~\ref{unemploymentAndChangeInHouse}.
However, this slope (and the y-intercept) are only estimates
of the parameter values.
We might wonder, is this convincing evidence that the ``true''
linear model has a negative slope?
That is, do the data provide strong evidence that the political
theory is accurate, where the unemployment rate is a useful
predictor of the midterm election?
We can frame this investigation into a statistical hypothesis
test:
\begin{itemize}
\item[$H_0$:]
    $\beta_1 = 0$.
    The true linear model has slope zero.
\item[$H_A$:]
    $\beta_1 \neq 0$.
    The true linear model has a slope different than zero.
    The unemployment is predictive of whether the President's
    party wins or loses seats in the House of Representatives.
\end{itemize}
We would reject $H_0$ in favor of $H_A$ if the data provide
strong evidence that the true slope parameter is different
than zero.
To assess the hypotheses, we identify a standard error
for the estimate, compute an appropriate test statistic,
and identify the p-value.


\subsection{Understanding regression output from software}
\label{testStatisticForTheSlope}

\newcommand{\midtermshouseDF}{27}

Just like other point estimates we have seen before,
we can compute a standard error and test statistic for $b_1$.
We will generally label the test statistic using a $T$,
since it follows the $t$-distribution.

We will rely on statistical software to compute the standard
error and leave the explanation of how this standard error
is determined to a second or third statistics course.
Figure~\ref{midtermUnempRegTable} shows software output for
the least squares regression line in
Figure~\ref{unemploymentAndChangeInHouse}.
The row labeled \emph{unemp} includes the point estimate
and other hypothesis test information for the slope,
which is the coefficient of the unemployment variable.

\begin{figure}[ht]
\centering
\begin{tabular}{rrrrr}
  \hline
  \vspace{-3.7mm} & & & & \\
  & Estimate & Std. Error & t value & Pr($>$$|$t$|$) \\ 
  \hline
  \vspace{-3.6mm} & & & & \\
  (Intercept) & -7.3644 & 5.1553 & -1.43 & 0.1646 \\ 
  unemp & -0.8897 & 0.8350 & -1.07 & 0.2961 \\ 
  \hline
  \multicolumn{5}{r}{$df=\midtermshouseDF{}$} \\
\end{tabular}
\caption{Output from statistical software for the regression
    line modeling the midterm election losses for the
    President's party as a response to unemployment.}
\label{midtermUnempRegTable}
\end{figure}

\begin{examplewrap}
\begin{nexample}{What do the first and second columns
    of Figure~\ref{midtermUnempRegTable} represent?}
  The entries in the first column represent the least
  squares estimates, $b_0$ and $b_1$, and the values in
  the second column correspond to the standard errors
  of each estimate.
  Using the estimates, we could write the equation
  for the least square regression line as
  \begin{align*}
  \hat{y} = -7.3644 - 0.8897 x
  \end{align*}
  where $\hat{y}$ in this case represents the predicted
  change in the number of seats for the president's party,
  and $x$ represents the unemployment rate.
\end{nexample}
\end{examplewrap}

\D{\newpage}

We previously used a $t$-test statistic for hypothesis testing
in the context of numerical data.
Regression is very similar.
In the hypotheses we consider, the null value for the slope is~0,
so we can compute the test statistic using the T (or Z) score
formula:
\begin{align*}
T
  = \frac{\text{estimate} - \text{null value}}{\text{SE}}
  = \frac{-0.8897 - 0}{0.8350}
  = -1.07
\end{align*}
This corresponds to the third column of
Figure~\ref{midtermUnempRegTable}.

%\begin{figure}[h]
%  \centering
%  \Figure{0.82}{pValueMidtermUnemp}
%  \caption{The distribution shown here is the sampling distribution for $b_1$, if the null hypothesis was true. The shaded tail represents the p-value for the hypothesis test evaluating whether there is convincing evidence that higher unemployment corresponds to a greater loss of House seats for the President's party during a midterm election.}
%  \label{pValueMidtermUnemp}
%\end{figure}

\begin{examplewrap}
\begin{nexample}{Use the table in
    Figure~\ref{midtermUnempRegTable}
    to determine the p-value for the hypothesis test.}
  The last column of the table gives the p-value for
  the two-sided hypothesis test for the coefficient of
  the unemployment rate: 0.2961.
  That is, the data do not provide convincing evidence
  that a higher unemployment rate has any correspondence
  with smaller or larger losses for the President's party
  in the House of Representatives in midterm elections.
\end{nexample}
\end{examplewrap}

\index{data!midterm elections|)}

\begin{onebox}{Inference for regression}
  We usually rely on statistical software to identify point
  estimates, standard errors, test statistics, and p-values
  in practice.
  However, be aware that software will not generally
  check whether the method is appropriate, meaning we must
  still verify conditions are met.
\end{onebox}

\begin{examplewrap}
\begin{nexample}{Examine Figure~\vref{elmhurstScatterWLSROnly},
    which relates the Elmhurst College aid and student family
    income.
    How sure are you that the slope is statistically
    significantly different from zero?
    That is, do you think a formal hypothesis test would reject
    the claim that the true slope of the line should be zero?}
  \label{overallAidIncomeInfAssessOfRegrLineSlope}%
  While the relationship between the variables is not perfect,
  there is an evident decreasing trend in the data.
  This suggests the hypothesis test will reject the null claim
  that the slope is zero.
\end{nexample}
\end{examplewrap}

\begin{exercisewrap}
\begin{nexercise}
Figure~\ref{rOutputForIncomeAidLSRLineInInferenceSection}
shows statistical software output from fitting the least
squares regression line shown in
Figure~\ref{elmhurstScatterWLSROnly}.
Use this output to formally evaluate the following
hypotheses.\footnotemark{}
\begin{itemize}
\setlength{\itemsep}{0mm}
\item[$H_0$:]
    The true coefficient for family income is zero.
\item[$H_A$:]
    The true coefficient for family income is not zero.
\end{itemize}
\end{nexercise}
\end{exercisewrap}
\footnotetext{We look in the second row corresponding
  to the family income variable.
  We see the point estimate of the slope of the line is -0.0431,
  the standard error of this estimate is 0.0108, and the $t$-test
  statistic is $T = -3.98$.
  The p-value corresponds exactly to the two-sided test we are
  interested in: 0.0002.
  The p-value is so small that we reject the null hypothesis
  and conclude that family income and financial aid at Elmhurst
  College for freshman entering in the year 2011 are negatively
  correlated and the true slope parameter is indeed less than~0,
  just as we believed in
  Example~\ref{overallAidIncomeInfAssessOfRegrLineSlope}.}

\begin{figure}[ht]
\centering
\begin{tabular}{rrrrr}
  \hline
  \vspace{-3.7mm} & & & & \\
 & Estimate & Std. Error & t value & Pr($>$$|$t$|$) \\ 
  \hline
  \vspace{-3.6mm} & & & & \\
(Intercept) & 24319.3 & 1291.5 & 18.83 & $<$0.0001 \\ 
family\us{}income & -0.0431 & 0.0108 & -3.98 & 0.0002 \\ 
   \hline
   \multicolumn{5}{r}{$df=48$} \\
\end{tabular}
\caption{Summary of least squares fit for the Elmhurst
    College data, where we are predicting the gift aid
    by the university based on the family income of
    students.}
\label{rOutputForIncomeAidLSRLineInInferenceSection}
\end{figure}


\newpage

\subsection{Confidence interval for a coefficient}

\index{confidence interval!regression|(}%

Similar to how we can conduct a hypothesis test for
a model coefficient using regression output, we can also
construct a confidence interval for that coefficient.

\begin{examplewrap}
\begin{nexample}{
    Compute the 95\% confidence interval for the
    \var{family\us{}income} coefficient using the
    regression output from
    Table~\ref{rOutputForIncomeAidLSRLineInInferenceSection}.}
  The point estimate is -0.0431 and the standard error is
  $SE = 0.0108$.
  When constructing a confidence interval for a model
  coefficient, we generally use a $t$-distribution.
  The degrees of freedom for the distribution are noted in
  the regression output, $df = 48$, allowing us to identify
  $t_{48}^{\star} = 2.01$ for use in the confidence interval.

  We can now construct the confidence interval in the usual way:
  \begin{align*}
  \text{point estimate} \pm t_{48}^{\star} \times SE
    \qquad\to\qquad -0.0431 \pm 2.01 \times 0.0108
    \qquad\to\qquad (-0.0648, -0.0214)
  \end{align*}
  We are 95\% confident that with each dollar increase in
  \var{family\us{}income}, the university's gift aid is
  predicted to decrease on average by \$0.0214 to \$0.0648.
\end{nexample}
\end{examplewrap}

\begin{onebox}{Confidence intervals for coefficients}
  Confidence intervals for model coefficients can be computed
  using the $t$-distribution:
  \begin{align*}
  b_i \ \pm\ t_{df}^{\star} \times SE_{b_{i}}
  \end{align*}
  where $t_{df}^{\star}$ is the appropriate $t$-value
  corresponding to the confidence level with the
  model's degrees of freedom.
\end{onebox}

On the topic of intervals in this book, we've focused exclusively
on confidence intervals for model parameters.
However, there are other types of intervals that may be
of interest, including
prediction intervals\index{prediction interval}
for a response value
and also
confidence intervals for a
mean response value\index{mean response value}
in the context of regression.
These two interval types are introduced in an online extra
that you may download at
\begin{center}
\oiRedirect{stat_extra_linear_regression_supp}
    {www.openintro.org/d?file=stat\_extra\_linear\_regression\_supp}
\end{center}

\index{confidence interval!regression|)}%
\index{regression|)}


{\exercisesheader{}

\noindent%
In the following exercises, visually check the conditions
for fitting a least squares regression line.
However, you do not need to report these conditions in
your solutions.\\[6mm]

% 31

\eoce{\qt{Body measurements, Part IV\label{body_measurements_weight_height_inf}} 
The scatterplot and least squares summary below show the relationship 
between weight measured in kilograms and height measured in centimeters 
of 507 physically active individuals.

\noindent\begin{minipage}[c]{0.4\textwidth}
\begin{center}
\FigureFullPath[A scatterplot is shown with around 500 points. The horizontal axis is for "Height, in centimeters" and takes values between about 150 to 200 centimeters. The vertical axis is for "Weight, in kilograms" and takes values between about 40 to 120 centimeters. For heights smaller than about 160 centimeters, weights mostly range between 45 and 70 kilograms. For heights between 160 and 175 centimeters, weights mostly range between 55 and 80 kilograms. For heights between 175 and 185 centimeters, weights mostly range between 65 and 90 kilograms. For heights between 185 and 195 centimeters, where there are fewer points, weights mostly range between 80 and 95 kilograms. There are two points with heights at about than 196cm, and these have weights of about 85 and 95 kilograms.]{}{ch_regr_simple_linear/figures/eoce/body_measurements_weight_height_inf/body_measurements_weight_height}
\end{center}
\end{minipage}
\begin{minipage}[c]{0.6\textwidth}
{\scriptsize
\begin{center}
\begin{tabular}{rrrrr}
    \hline
            & Estimate  & Std. Error    & t value   & Pr($>$$|$t$|$) \\ 
    \hline
(Intercept) & -105.0113 & 7.5394        & -13.93    & 0.0000 \\ 
height      & 1.0176    & 0.0440        & 23.13     & 0.0000 \\
    \hline
\end{tabular}
\end{center}
}
\end{minipage}
\begin{parts}
\item Describe the relationship between height and weight.
\item Write the equation of the regression line. Interpret the slope 
and intercept in context.
\item Do the data provide strong evidence that an increase in height 
is associated with an increase in weight? State the null and alternative 
hypotheses, report the p-value, and state your conclusion.
\item The correlation coefficient for height and weight is 0.72. 
Calculate $R^2$ and interpret it in context.
\end{parts}
}{}

% 32

\eoce{\qt{Beer and blood alcohol content\label{beer_blood_alcohol_inf}} 
Many people believe that gender, 
weight, drinking habits, and many other factors are much more important 
in predicting blood alcohol content (BAC) than simply considering the 
number of drinks a person consumed. Here we examine data from sixteen 
student volunteers at Ohio State University who each drank a randomly 
assigned number of cans of beer. These students were evenly divided 
between men and women, and they differed in weight and drinking habits. 
Thirty minutes later, a police officer measured their blood alcohol 
content (BAC) in grams of alcohol per deciliter of blood.
\footfullcite{Malkevitc+Lesser:2008} The scatterplot and regression 
table summarize the findings.

\noindent\begin{minipage}[c]{0.4\textwidth}
\begin{center}
\FigureFullPath[A scatterplot is shown with around 15 points. The horizontal axis is for "Cans of beer" and takes values between about 1 and 9. The vertical axis is for "Blood Alcohol Concentration (BAC), in grams per deciliter" and takes values between about 0.01 to 0.2 centimeters. The point at 1 can of beer is at 0.01 BAC, lower than any other values. For the four points at 2 and 3 cans of beer, BAC ranges from 0.02 to 0.07. For the six points at 4 and 5 cans of beer, BAC ranges from 0.05 to 0.10. Two points are at 7 cans of beer and have BAC of 0.09 and 0.10. There is a single point for 8 cans of beer, which has a BAC of 0.12, and one last point at 9 cans of beer, which has a BAC of about 0.19.]{}{ch_regr_simple_linear/figures/eoce/beer_blood_alcohol_inf/beer_blood_alcohol}
\end{center}
\end{minipage}
\begin{minipage}[c]{0.6\textwidth}
{\scriptsize
\begin{center}
\begin{tabular}{rrrrr}
    \hline
            & Estimate  & Std. Error    & t value   & Pr($>$$|$t$|$) \\ 
    \hline
(Intercept) & -0.0127   & 0.0126        & -1.00     & 0.3320 \\ 
beers       & 0.0180    & 0.0024        & 7.48      & 0.0000 \\ 
    \hline
\end{tabular}
\end{center}
}
\end{minipage}
\begin{parts}
\item Describe the relationship between the number of cans of beer 
and BAC.
\item Write the equation of the regression line. Interpret the slope 
and intercept in context.
\item Do the data provide strong evidence that drinking more cans of 
beer is associated with an increase in blood alcohol? State the null 
and alternative hypotheses, report the p-value, and state your 
conclusion.
\item The correlation coefficient for number of cans of beer and BAC 
is 0.89. Calculate $R^2$ and interpret it in context.
\item Suppose we visit a bar, ask people how many drinks they have had, 
and also take their BAC. Do you think the relationship between number 
of drinks and BAC would be as strong as the relationship found in the 
Ohio State study?
\end{parts}
}{}

\D{\newpage}

% 33

\eoce{\qt{Husbands and wives, Part II\label{husbands_wives_height_inf}} The 
scatterplot below summarizes husbands' and wives' heights in a random 
sample of 170 married couples in Britain, where both partners' ages are 
below 65 years. Summary output of the least squares fit for predicting 
wife's height from husband's height is also provided in the table.

\noindent\begin{minipage}[c]{0.4\textwidth}
\begin{center}
\FigureFullPath[A scatterplot is shown with around 200 points. The horizontal axis is for "Husband's height, in inches" and takes values between 60 and 75 inches. The vertical axis is for "Wife's height, in inches" and takes values between 55 and 70 inches. For the approximately fifteen husband heights smaller than 65 inches, wife heights are mostly between 59 and 65 inches. For the approximately 100 husband heights between 65 and 70 inches, wife heights are mostly between 59 and 66 inches. For the approximately 30 husband heights taller than 70 inches, wife heights are mostly between 62 and 67 inches.]{}{ch_regr_simple_linear/figures/eoce/husbands_wives_height_inf_2s/husbands_wives_height_inf_2s}
\end{center}
\end{minipage}
\begin{minipage}[c]{0.6\textwidth}
{\scriptsize
\begin{center}
\begin{tabular}{rrrrr}
    \hline
                    & Estimate  & Std. Error    & t value   & Pr($>$$|$t$|$) \\ 
    \hline
(Intercept)         & 43.5755   & 4.6842        & 9.30      & 0.0000 \\ 
height\_\hspace{0.3mm}husband   & 0.2863    & 0.0686        & 4.17      & 0.0000 \\ 
    \hline
\end{tabular}
\end{center}
}
\end{minipage}
\begin{parts}
\item Is there strong evidence that taller men marry taller women? 
State the hypotheses and include any information used to conduct the test.
\item Write the equation of the regression line for predicting wife's 
height from husband's height.
\item Interpret the slope and intercept in the context of the application.
\item Given that $R^2 = 0.09$, what is the correlation of heights 
in this data set?
\item You meet a married man from Britain who is 5'9" (69 inches). 
What would you predict his wife's height to be? How reliable is this 
prediction?
\item You meet another married man from Britain who is 6'7" (79 inches). 
Would it be wise to use the same linear model to predict his wife's 
height? Why or why not?
\end{parts}
}{}

% 34

\eoce{\qt{Urban homeowners, Part II\label{urban_homeowners_cond}}
Exercise~\ref{urban_homeowners_outlier} gives a scatterplot displaying the 
relationship between the percent of families that own their home and 
the percent of the population living in urban areas. Below is a 
similar scatterplot, excluding District of Columbia, as well as the 
residuals plot. There were 51 cases.

\noindent\begin{minipage}[c]{0.45\textwidth}
{\raggedright\begin{parts}
\item For these data, $R^2=0.28$. What is the correlation? How can 
you tell if it is positive or negative?
\item Examine the residual plot. What do you observe? Is a simple 
least squares fit appropriate for these data?
\end{parts}\vspace{15mm}}
\end{minipage}
\begin{minipage}[c]{0.1\textwidth}
$\:$ \\
\end{minipage}
\begin{minipage}[c]{0.43\textwidth}
\begin{center}
\FigureFullPath[A scatterplot is shown. The horizontal axis represents "Husband's Age (in years)" with values ranging from about 20 to 65. The vertical axis represents "Wife's Age (in years)" with values ranging from about 18 to 65. When husband age is between 20 and 30, wife age mostly ranges from 18 to about 30. When husband age is between 30 and 40, wife age mostly ranges from 23 to about 40. When husband age is between 40 and 50, wife age mostly ranges from 35 to about 50. When husband age is between 50 and 60, wife age mostly ranges from 45 to about 60. When husband age is larger than 60, wife age mostly ranges from 55 to about 65.]{}{ch_regr_simple_linear/figures/eoce/urban_homeowners_cond/urban_homeowners_cond}
\end{center}
\end{minipage}
}{}

\D{\newpage}

% 35

\eoce{\qt{Murders and poverty, Part II\label{murders_poverty_inf}}
Exercise~\ref{murders_poverty_reg} presents regression output from a model 
for predicting annual murders per million from percentage living in 
poverty based on a random sample of 20 metropolitan areas. The model 
output is also provided below.
\begin{center}
\begin{tabular}{rrrrr}
    \hline
            & Estimate  & Std. Error    & t value   & Pr($>$$|$t$|$) \\ 
    \hline
(Intercept) & -29.901   & 7.789         & -3.839    & 0.001 \\ 
poverty\%   & 2.559     & 0.390         & 6.562     & 0.000 \\ 
    \hline
\end{tabular}
\[ s = 5.512 \qquad R^2 = 70.52\% \qquad R^2_{adj} = 68.89\% \]
\end{center}
\begin{parts}
\item What are the hypotheses for evaluating whether poverty percentage 
is a significant predictor of murder rate?
\item State the conclusion of the hypothesis test from part (a) in 
context of the data.
\item Calculate a 95\% confidence interval for the slope of poverty 
percentage, and interpret it in context of the data.
\item Do your results from the hypothesis test and the confidence 
interval agree? Explain.
\end{parts}
}{}

% 36

\eoce{\qt{Babies\label{babies_head_gestation_inf}} Is the gestational age 
(time between conception and birth) of a low birth-weight baby useful 
in predicting head circumference at birth? Twenty-five low birth-weight 
babies were studied at a Harvard teaching hospital; the investigators 
calculated the regression of head circumference (measured in centimeters) 
against gestational age (measured in weeks). The estimated regression 
line is
\[ \widehat{head~circumference} = 3.91 + 0.78 \times gestational~age \]
\begin{parts}
\item What is the predicted head circumference for a baby whose 
gestational age is 28 weeks?
\item The standard error for the coefficient of gestational age is 0.
35, which is associated with $df=23$. Does the model provide strong 
evidence that gestational age is significantly associated with head 
circumference?
\end{parts} 
}{}
}
