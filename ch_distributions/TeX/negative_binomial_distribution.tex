\exercisesheader{}

% 27

\eoce{\qt{Rolling a die\label{roll_die}} Calculate the 
following probabilities and indicate which probability distribution model is 
appropriate in each case. You roll a fair die 5 times. What is the 
probability of rolling
\begin{parts}
\item the first 6 on the fifth roll?
\item exactly three 6s?
\item the third 6 on the fifth roll?
\end{parts}
}{}

% 28

\eoce{\qt{Playing darts\label{play_darts}} Calculate the following probabilities 
and indicate which probability distribution model is appropriate in each 
case. A very good darts player can hit the bull's eye (red circle in the 
center of the dart board) 65\% of the time. What is the probability that he
\begin{parts}
\item hits the bullseye for the $10^{th}$ time on the $15^{th}$ try?
\item hits the bullseye 10 times in 15 tries?
\item hits the first bullseye on the third try?
\end{parts}
}{}

% 29

\eoce{\qt{Sampling at school\label{sampling_at_school}} For a sociology class 
project you are asked to conduct a survey on 20 students at your school. You 
decide to stand outside of your dorm's cafeteria and conduct the survey on a 
random sample of 20 students leaving the cafeteria after dinner one evening. 
Your dorm is comprised of 45\% males and 55\% females.
\begin{parts}
\item Which probability model is most appropriate for calculating the 
probability that the $4^{th}$ person you survey is the $2^{nd}$ female? 
Explain.
\item Compute the probability from part (a).
\item The three possible scenarios that lead to $4^{th}$ person you survey 
being the $2^{nd}$ female are
\[ \{M, M, F, F\}, \{M, F, M, F\}, \{F, M, M, F\} \]
One common feature among these scenarios is that the last trial is always 
female. In the first three trials there are 2 males and 1 female. Use the 
binomial coefficient to confirm that there are 3 ways of ordering 2 males and 
1 female. 
\item Use the findings presented in part (c) to explain why the formula for 
the coefficient for the negative binomial is ${n-1 \choose k-1}$ while the 
formula for the binomial coefficient is ${n \choose k}$.
\end{parts}
}{}

% 30

\eoce{\qt{Serving in volleyball\label{serving_volleyball}} A not-so-skilled 
volleyball player has a 15\% chance of making the serve, which involves 
hitting the ball so it passes over the net on a trajectory such that it will 
land in the opposing team's court. Suppose that her serves are independent of 
each other.
\begin{parts}
\item What is the probability that on the $10^{th}$ try she will make her 
$3^{rd}$ successful serve?
\item Suppose she has made two successful serves in nine attempts. What is 
the probability that her $10^{th}$ serve will be successful?
\item Even though parts (a) and (b) discuss the same scenario, the 
probabilities you calculated should be different. Can you explain the reason 
for this discrepancy?
\end{parts}
}{}
