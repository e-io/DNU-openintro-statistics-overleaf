\reviewexercisesheader{}

% 27

\eoce{\qt{Make-up exam\label{makeup_exam}} In a class of 25 students, 24 of them took an exam 
in class and 1 student took a make-up exam the following day. The professor graded the 
first batch of 24 exams and found an average score of 74 points with a standard 
deviation of 8.9 points. The student who took the make-up the following day scored 64 
points on the exam.
\begin{parts}
\item Does the new student's score increase or decrease the average score?
\item What is the new average?
\item Does the new student's score increase or decrease the standard deviation of the 
scores?
\end{parts}
}{}

% 28

\eoce{\qt{Infant mortality\label{infant_mortality}} The infant mortality rate is defined as 
the number of infant deaths per 1,000 live births. This rate is often used as an 
indicator of the level of health in a country. The relative frequency histogram below 
shows the distribution of estimated infant death rates for 224 countries for which such 
data were available in 2014. 
\footfullcite{data:ciaFactbook}

\noindent\begin{minipage}[c]{0.43\textwidth}
\begin{parts}
\item Estimate Q1, the median, and Q3 from the histogram.
\item Would you expect the mean of this data set to be smaller or larger than the 
median? Explain your reasoning.
\end{parts} \vfill \
\end{minipage}
\begin{minipage}[c]{0.52\textwidth}
\hfill%
\Figures[A histogram is shown for the variable "Infant Mortality (per 1000 live births)" with axis range of 0 to 120. The histogram vertical axis is for "Fraction of Countries" and runs from 0 to 0.4. The bins are as follows: the 0 to 10 bin has a height of 0.38, 10 to 20 has a height of 0.22, 20 to 30 a height of 0.11, 30 to 40 a height of 0.06, 40 to 50 a height of 0.07, 50 to 60 a height of 0.08, 60 to 70 a height of 0.04, 70 to 80 a height of 0.03, 80 to 90 a height of 0.01, 90 to 100 a height of 0.02, and 100 to 110 a height of 0.01.]{0.85}{eoce/infant_mortality_rel_freq}{infant_mortality_rel_freq_hist}
\end{minipage}
}{}

% 29

\eoce{\qt{TV watchers\label{dist_shape_TV_watchers}} Students in an AP Statistics class 
were asked how many hours of television they watch per week (including online 
streaming). This sample yielded an average of 4.71 hours, with a standard 
deviation of 4.18 hours. Is the distribution of number of hours students watch 
television weekly symmetric? If not, what shape would you expect this distribution 
to have? Explain your reasoning.
}{}

% 30

\eoce{\qt{A new statistic\label{new_stat}} The statistic $\frac{\bar{x}}{median}$ can 
be used as a measure of skewness. Suppose we have a distribution where all 
observations are greater than 0, $x_i > 0$. What is the expected shape of 
the distribution under the following conditions? Explain your reasoning.
\begin{parts}
\item $\frac{\bar{x}}{median} = 1$
\item $\frac{\bar{x}}{median} < 1$
\item $\frac{\bar{x}}{median} > 1$
\end{parts}
}{}

% 31

\eoce{\qt{Oscar winners\label{oscar_winners}} The first Oscar awards for best actor 
and best actress were given out in 1929. The histograms below show the age 
distribution for all of the best actor and best actress winners from 1929 to 
2018. Summary statistics for these distributions are also provided. Compare the 
distributions of ages of best actor and actress winners.\footfullcite{data:oscars} \\
\begin{minipage}[c]{0.72\textwidth}
\begin{center}
\Figures[Two histograms are shown, one for "Best Actress" and a second for "Best Actor", where values for the histogram range from 15 to 85. The heights of the bins or the Best Actress histogram are as follows: the bin of 15 to 25 has a height of 9, the 25 to 35 bin has a height of 50, 35 to 45 a height of 19, 45 to 55 a height of 6, 55 to 65 a height of 8, 65 to 75 a height of 1, and 75 to 85 a height of 1. The heights of the bins or the Best Actress histogram are as follows: the bin of 15 to 25 has a height of 0, the 25 to 35 bin has a height of 14, 35 to 45 a height of 45, 45 to 55 a height of 23, 55 to 65 a height of 11, 65 to 75 a height of 0, and 75 to 85 a height of 1.]{0.95}{eoce/oscar_winners}{oscars_winners_hist}
\end{center}
\end{minipage}
\begin{minipage}[c]{0.27\textwidth}
{\small
\begin{tabular}{l c}
\hline
        & Best Actress  \\
\hline
Mean    & 36.2      \\
SD      & 11.9      \\
n       & 92        \\  
        & \\
        & \\
        & \\
        & \\
        & \\
\hline
        & Best Actor \\
\hline
Mean    & 43.8 \\
SD      & 8.83 \\
n       & 92
\end{tabular}
}
\end{minipage}
}{}

% 32

\eoce{\qt{Exam scores\label{dist_shape_exam_scores}} The average on a history exam 
(scored out of 100 points) was 85, with a standard deviation of 15. Is the 
distribution of the scores on this exam symmetric? If not, what shape would 
you expect this distribution to have? Explain your reasoning.
}{}

% 33

\eoce{\qt{Stats scores\label{stats_scores_box}} Below are the final exam scores of twenty 
introductory statistics students.
\begin{center}
57, 66, 69, 71, 72, 73, 74, 77, 78, 78, 79, 79, 81, 81, 82, 83, 83, 88, 89, 94
\end{center}
Create a box plot of the distribution of these scores. The five number summary provided below may be useful.
\begin{center}
\renewcommand\arraystretch{1.5}
\begin{tabular}{ccccc}
Min & Q1    & Q2 (Median)   & Q3    & Max \\
\hline
57  & 72.5  & 78.5          & 82.5  & 94 \\
\end{tabular}
\end{center}
}{}

% 34

\eoce{\qt{Marathon winners\label{marathon_winners}} The histogram and box plots below show the distribution of finishing times in hours for male and female winners of the New York Marathon between 1970 and 1999.
\begin{center}
\Figures[Two plots are shown, one that is a histogram and one that is a box plot, where the range of data for each is from 2.0 to 3.2. The bins for the histogram are as follows: the 2.0 to 2.2 bin has a height of 21, bin 2.2 to 2.4 a height of 6, 2.4 to 2.6 a height of 25, 2.6 to 2.8 a height of 3, 2.8 to 3.0 a height of 2, and 3.0 to 3.2 a height of 2. The box plot shows the box spanning 2.2 to 2.5, with the median line centered at 2.4. The whiskers extend from about 2.15 to 2.75. There are four points marked beyond the upper whisker at 2.9, 3.0, 3.10, and 3.15.]{0.56}{eoce/marathon_winners}{marathon_winners_hist_box}
\end{center}
\begin{parts}
\item What features of the distribution are apparent in the histogram and not the box plot? What features are apparent in the box plot but not in the histogram?
\item What may be the reason for the bimodal distribution? Explain.
\item Compare the distribution of marathon times for men and women based on the box plot shown below.
\begin{center}
\Figures[A side-by-side box plot is shown for marathon run times, one box plot for men and one for women. The axis for the run times spans from 2.0 to 3.2. All values described as follows are estimates. For the men box plot, the box spans 2.16 to 2.22 with the median line at 2.19. The whiskers span to 2.12 up to 2.27. There are 6 points above the upper whisker at 2.32, 2.36, 2.38, 2.44, 2.46, and 2.50. For the women box plot, the box spans from 2.44 to 2.52, with a median value of 2.46. The whiskers span from 2.41 to 2.57. There are 6 points above the upper whisker: 2.72, 2.78, 2.9, 2.92, 3.12, and 3.15.]{0.56}{eoce/marathon_winners}{marathon_winners_gender_box}
\end{center}
\item The time series plot shown below is another way to look at these data. Describe what is visible in this plot but not in the others.
\end{parts}
\begin{center}
\Figures[A time series plot is shown, which in this case gives the appearance of a scatterplot. The horizontal variable is for year, which runs from 1970 to 2000, and the vertical variable is "Marathon times", which runs from 2.0 to 3.2 hours. There are two colors of points, one for men and one for women, and there is one point for men and one for women for each year. The points start at about 2.5 for men in 1970 and 2.9 for women in 1971. The points bounce around for a few years and then decline in 1975 or 1976 to 2.2 for men and 2.7 for women. The values for women decreases for a few more years to about 2.5. For the remainder of the years, the values fluctuate up or down 0.1 hours from year to year but are stable until 1999, which is the last data points provided.]{0.6}{eoce/marathon_winners}{marathon_winners_time_series} \\
\end{center}
}{}
